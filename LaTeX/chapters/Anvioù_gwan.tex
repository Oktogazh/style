\begin{LTR}
\begin{otherlanguage}{breton}

Pa geñverier ar mod ma tro an anvioù-gwan e brezhoneg hag e galleg e
kaver perzhioù heñvel ha disheñvel. En div yezh e c'hall
servijout an anvioù-gwan da a zoareañ pe da resisaat un anv-kadarn.
Alies e c'hallont ivez merkañ stadoù
(\href{https://arbres.iker.cnrs.fr/index.php?title=Les_adjectifs}{Arbres}).
E brezhoneg hag e galleg e c'hall an anv-gwan
c'hoari roll an \href{/wiki/Doareen}{doareen} (da
skouer, \emph{skuizh eo ar plac'h}) pe ar
\href{/wiki/Stagenn}{stagenn} (\emph{ur plac'h skuizh}).

E galleg e kenglot an anvioù-gwan penn-da-benn er reizh hag en niver
gant an anv-kadarn a gemmont. E brezhoneg, avat, e vez merket
kenglotadur ar reizh hag an niver gant kemmadurioù.

\subsection{Plas an anv-gwan}\label{Plas_an_anv-gwan}

{{{[}}\href{/wiki/Anvio\%C3\%B9-gwan?action=edit&section=1}{{modifier}}{{]}}}

Hervez \href{/wiki/Even_1987}{Even (1987, 122)} e veze lakaet an
anv-gwan dirak an anv-kadarn aliesoc'h er
c'hrennvrezhoneg eget e brezhoneg modern. Peurliesañ e
laker an anv-gwan goude an anv-kadarn e brezhoneg a-vremañ. Koulskoude,
e degouezhioù zo, e c'haller lakaat un nebeud anezho
dirak an anv:

Il avait soif et son premier soin fut d'entrer chez la
veuve Le Stum, qui, dans sa propre chambre, dont elle fermait seulement
le lit, vendait à boire.

Queffélec 1970, 47

Sec'hed en doa hag e gentañ preder a reas dezhañ mont
tre e ti an intañvez ar Stumm hag a werzhe boeson en he
c'hambr ma ne rae nemet klozañ ar gwele.

Beyer 2016, 29

War
\href{https://arbres.iker.cnrs.fr/index.php?title=Adjectifs_ant\%C3\%A9pos\%C3\%A9s_au_nom}{ARBRES}
e kaver ul listenn glok eus an anvioù-gwan a vez lakaet dirak an anv:

\begin{itemize}
\tightlist
\item
  niveroù pegementiñ, niveroù petvediñ ha niveroù kevrennañ evel
  \emph{hanter};
\item
  anvioù-gwan pe anvioù prizañ, evel \emph{kaezh, paour-kaezh, lastez-,
  c'hoant, gwashat};
\item
  gerioù tabou implijet evel anvioù-gwan kreñvaat evel \emph{ur sapre
  den};
\item
  gerioù amresis evel \emph{holl} ha \emph{nep};
\item
  anvioù-gwan unsilabennek pe divsilabennek troet da rakgerioù, en o
  zouez \emph{arall-, berr-, bihan-,}
\item
  bras-, brizh-, dister-, dreist-, drouk-, fall-/fals-, gouez-, gwen-,
  heñvel-, hir-, kamm-, kozh-,
\item
  krak-, krenn-, izel-, nevez-, pell-, pounner-, reizh-, skañv-, tomm-,
  uhel-\emph{.}
\end{itemize}

Hervez \href{/wiki/Kervella_1995}{Kervella (1995, 88)}, an anvioù-gwan
doareañ en derez-plaen (gwall, kozh, brizh, krak, dister...) a vez graet
ganto evel gant gwir rak-gerioù. Koulskoude ne c'hoarvez
kemmadur ebet goude un anv-gwan en derez-uhel:

Dans la visite que la plus belle fille de l'île lui
avait faite et dans les phrases âpres et suaves qu'elle
lui avait lancées, il reconnaissait, de toute évidence, une tentation.

Queffélec 1970, 54

Ar weladenn graet dezhañ gant kaerañ plac'h an enez,
koulz hag ar c'homzoù garv ha c'hwek
taolet dezhañ, a anaveze evel un temptadur anat.

Beyer 2016, 35

\subsection[Eztaoler ur c'heñveriadur gant
anvioù-gwan-doareañ]{\texorpdfstring{\protect\hypertarget{Eztaoler_ur_c.27he.C3.B1veriadur_gant_anvio.C3.B9-gwan-doarea.C3.B1}{}{}Eztaoler
ur c'heñveriadur gant
anvioù-gwan-doareañ}{Eztaoler ur c'heñveriadur gant anvioù-gwan-doareañ}}\label{Eztaoler_ur_cux27heuxf1veriadur_gant_anviouxf9-gwan-doareauxf1}

{{{[}}\href{/wiki/Anvio\%C3\%B9-gwan?action=edit&section=2}{{modifier}}{{]}}}

An anvioù-gwan-doareañ a c'haller implijout ivez evit
eztaoler ur c'heñveriadur, da skouer gant \emph{par}
(da):

Ils se mettraient en face l'un de
l'autre comme de jeunes animaux noirs
qu'ils étaient, deux créatures de l'île,
un homme, une femme, la voile et le mât d'une famille.

Queffélec 1970, 54

En em lakaat a rafent tal-ouzh-tal, par da zaou loen yaouank du a oa
anezho, daou grouadur eus an enez, ur gwaz, ur vaouez, gouel ha gwern un
tiegezh.

Beyer 2016, 34

\subsection{Anv-gwan hep anv ouzh e
heul}\label{Anv-gwan_hep_anv_ouzh_e_heul}

{{{[}}\href{/wiki/Anvio\%C3\%B9-gwan?action=edit&section=3}{{modifier}}{{]}}}

E degouezhioù ma n'eus anv ebet ouzh an anv-gwan e vez
graet gantañ evel gant un anv-kadarn gwirion:

Unan kastizet gant Doue, ha gant Doue e vefe-eñ kastizet d'e dro.

Queffélec 1970, 188

Un puni de Dieu qui allait le punir à son tour.

Beyer 2016, 138

Ar bravañ zo e vo dilugernet ouzhoc'h, ma tougit kañv
din gant koefoù plaen ha tavañjer du.

Drezen 2012, 326

Le plus joli est qu'on vous regardera de travers, si
vous portez mon deuil, avec des coiffes pleines et un tablier noir.

Drezen 2002, 205

\end{otherlanguage}
\end{LTR}
