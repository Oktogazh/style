\chapter{Anvioù-gwan}

Pa geñverier ar mod ma tro an anvioù-gwan e brezhoneg hag e galleg e
kaver perzhioù heñvel ha disheñvel. En div yezh e c\textquotesingle hall
servijout an anvioù-gwan da a zoareañ pe da resisaat un anv-kadarn.
Alies e c\textquotesingle hallont ivez merkañ stadoù
(\href{https://arbres.iker.cnrs.fr/index.php?title=Les_adjectifs}{Arbres}).
E brezhoneg hag e galleg e c\textquotesingle hall an anv-gwan
c\textquotesingle hoari roll an \url{doareen} (da skouer, \emph{skuizh
eo ar plac\textquotesingle h}) pe ar \url{stagenn} (\emph{ur
plac\textquotesingle h skuizh}).

E galleg e kenglot an anvioù-gwan penn-da-benn er reizh hag en niver
gant an anv-kadarn a gemmont. E brezhoneg, avat, e vez merket
kenglotadur ar reizh hag an niver gant kemmadurioù.

\section{Plas an anv-gwan}\label{plas_an_anv_gwan}

Hervez \href{Even_1987}{Even (1987, 122)} e veze lakaet an anv-gwan
dirak an anv-kadarn aliesoc\textquotesingle h er
c\textquotesingle hrennvrezhoneg eget e brezhoneg modern. Peurliesañ e
laker an anv-gwan goude an anv-kadarn e brezhoneg a-vremañ. Koulskoude,
e degouezhioù zo, e c\textquotesingle haller lakaat un nebeud anezho
dirak an anv:

\begin{longtable}[]{|p{0.45\textwidth}|p{0.45\textwidth}}

Il avait soif et son premier soin fut d\textquotesingle entrer chez la
veuve Le Stum, qui, dans sa propre chambre, dont elle fermait seulement
le lit, vendait à boire. & Sec\textquotesingle hed en doa hag e gentañ
preder a reas dezhañ mont tre e ti an intañvez ar Stumm hag a werzhe
boeson en he c\textquotesingle hambr ma ne rae nemet klozañ ar gwele. \\
Queffélec 1970, 47 & Beyer 2016, 29 \\
\end{longtable}

War
\href{https://arbres.iker.cnrs.fr/index.php?title=Adjectifs_ant\%C3\%A9pos\%C3\%A9s_au_nom}{ARBRES}
e kaver ul listenn glok eus an anvioù-gwan a vez lakaet dirak an anv:

\begin{itemize}
\tightlist
\item
  niveroù pegementiñ, niveroù petvediñ ha niveroù kevrennañ evel
  \emph{hanter};
\item
  anvioù-gwan pe anvioù prizañ, evel \emph{kaezh, paour-kaezh, lastez-,
  c\textquotesingle hoant, gwashat};
\item
  gerioù tabou implijet evel anvioù-gwan kreñvaat evel \emph{ur sapre
  den};
\item
  gerioù amresis evel \emph{holl} ha \emph{nep};
\item
  anvioù-gwan unsilabennek pe divsilabennek troet da rakgerioù, en o
  zouez \textquotesingle\textquotesingle arall-, berr-, bihan-,
\item
  bras-, brizh-, dister-, dreist-, drouk-, fall-/fals-, gouez-, gwen-,
  heñvel-, hir-, kamm-, kozh-,
\item
  krak-, krenn-, izel-, nevez-, pell-, pounner-, reizh-, skañv-, tomm-,
  uhel-\textquotesingle\textquotesingle.
\end{itemize}

Hervez \href{Kervella_1995}{Kervella (1995, 88)}, an anvioù-gwan doareañ
en derez-plaen (gwall, kozh, brizh, krak, dister...) a vez graet ganto
evel gant gwir rak-gerioù. Koulskoude ne c\textquotesingle hoarvez
kemmadur ebet goude un anv-gwan en derez-uhel:

\begin{longtable}[]{|p{0.45\textwidth}|p{0.45\textwidth}}

Dans la visite que la plus belle fille de l\textquotesingle île lui
avait faite et dans les phrases âpres et suaves qu\textquotesingle elle
lui avait lancées, il reconnaissait, de toute évidence, une tentation. &
Ar weladenn graet dezhañ gant kaerañ plac\textquotesingle h an enez,
koulz hag ar c\textquotesingle homzoù garv ha c\textquotesingle hwek
taolet dezhañ, a anaveze evel un temptadur anat. \\
Queffélec 1970, 54 & Beyer 2016, 35 \\
\end{longtable}

\section{Eztaoler ur c\textquotesingle heñveriadur gant
anvioù-gwan-doareañ}\label{eztaoler_ur_cheuxf1veriadur_gant_anviouxf9_gwan_doareauxf1}

An anvioù-gwan-doareañ a c\textquotesingle haller implijout ivez evit
eztaoler ur c\textquotesingle heñveriadur, da skouer gant \emph{par}
(da):

\begin{longtable}[]{|p{0.45\textwidth}|p{0.45\textwidth}}

Ils se mettraient en face l\textquotesingle un de
l\textquotesingle autre comme de jeunes animaux noirs
qu\textquotesingle ils étaient, deux créatures de l\textquotesingle île,
un homme, une femme, la voile et le mât d\textquotesingle une famille. &
En em lakaat a rafent tal-ouzh-tal, par da zaou loen yaouank du a oa
anezho, daou grouadur eus an enez, ur gwaz, ur vaouez, gouel ha gwern un
tiegezh. \\
Queffélec 1970, 54 & Beyer 2016, 34 \\
\end{longtable}

\section{Anv-gwan hep anv ouzh e
heul}\label{anv_gwan_hep_anv_ouzh_e_heul}

E degouezhioù ma n\textquotesingle eus anv ebet ouzh an anv-gwan e vez
graet gantañ evel gant un anv-kadarn gwirion:

\begin{longtable}[]{|p{0.45\textwidth}|p{0.45\textwidth}}

Unan kastizet gant Doue, ha gant Doue e vefe-eñ kastizet d'e dro. & Un
puni de Dieu qui allait le punir à son tour. \\
Queffélec 1970, 188 & Beyer 2016, 138 \\
\end{longtable}

\begin{longtable}[]{|p{0.45\textwidth}|p{0.45\textwidth}}

Ar bravañ zo e vo dilugernet ouzhoc\textquotesingle h, ma tougit kañv
din gant koefoù plaen ha tavañjer du. & Le plus joli est
qu\textquotesingle on vous regardera de travers, si vous portez mon
deuil, avec des coiffes pleines et un tablier noir. \\
Drezen 2012, 326 & Drezen 2002, 205 \\
\end{longtable}
