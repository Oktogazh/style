\chapter{Anvioù-gwan}

Pa geñverier ar mod ma tro an anvioù-gwan e brezhoneg hag e galleg e
kaver perzhioù heñvel ha disheñvel. En div yezh e c\textquotesingle hall
servijout an anvioù-gwan da a zoareañ pe da resisaat un anv-kadarn.
Alies e c\textquotesingle hallont ivez merkañ stadoù
(\href{https://arbres.iker.cnrs.fr/index.php?title=Les_adjectifs}{Arbres}).
E brezhoneg hag e galleg e c\textquotesingle hall an anv-gwan
c\textquotesingle hoari roll an \url{doareen} (da skouer, \emph{skuizh
eo ar plac\textquotesingle h}) pe ar \url{stagenn} (\emph{ur
plac\textquotesingle h skuizh}).

E galleg e kenglot an anvioù-gwan penn-da-benn er reizh hag en niver
gant an anv-kadarn a gemmont. E brezhoneg, avat, e vez merket
kenglotadur ar reizh hag an niver gant kemmadurioù.

\section{Plas an anv-gwan}\label{plas_an_anv_gwan}

Hervez \href{Even_1987}{Even (1987, 122)} e veze lakaet an anv-gwan
dirak an anv-kadarn aliesoc\textquotesingle h er
c\textquotesingle hrennvrezhoneg eget e brezhoneg modern. Peurliesañ e
laker an anv-gwan goude an anv-kadarn e brezhoneg a-vremañ. Koulskoude,
e degouezhioù zo, e c\textquotesingle haller lakaat un nebeud anezho
dirak an anv:

War
\href{https://arbres.iker.cnrs.fr/index.php?title=Adjectifs_ant\%C3\%A9pos\%C3\%A9s_au_nom}{ARBRES}
e kaver ul listenn glok eus an anvioù-gwan a vez lakaet dirak an anv:

\begin{itemize}
\tightlist
\item
  niveroù pegementiñ, niveroù petvediñ ha niveroù kevrennañ evel
  \emph{hanter};
\item
  anvioù-gwan pe anvioù prizañ, evel \emph{kaezh, paour-kaezh, lastez-,
  c\textquotesingle hoant, gwashat};
\item
  gerioù tabou implijet evel anvioù-gwan kreñvaat evel \emph{ur sapre
  den};
\item
  gerioù amresis evel \emph{holl} ha \emph{nep};
\item
  anvioù-gwan unsilabennek pe divsilabennek troet da rakgerioù, en o
  zouez \textquotesingle\textquotesingle arall-, berr-, bihan-,
\item
  bras-, brizh-, dister-, dreist-, drouk-, fall-/fals-, gouez-, gwen-,
  heñvel-, hir-, kamm-, kozh-,
\item
  krak-, krenn-, izel-, nevez-, pell-, pounner-, reizh-, skañv-, tomm-,
  uhel-\textquotesingle\textquotesingle.
\end{itemize}

Hervez \href{Kervella_1995}{Kervella (1995, 88)}, an anvioù-gwan doareañ
en derez-plaen (gwall, kozh, brizh, krak, dister...) a vez graet ganto
evel gant gwir rak-gerioù. Koulskoude ne c\textquotesingle hoarvez
kemmadur ebet goude un anv-gwan en derez-uhel:

\section{Eztaoler ur c\textquotesingle heñveriadur gant
anvioù-gwan-doareañ}\label{eztaoler_ur_cheuxf1veriadur_gant_anviouxf9_gwan_doareauxf1}

An anvioù-gwan-doareañ a c\textquotesingle haller implijout ivez evit
eztaoler ur c\textquotesingle heñveriadur, da skouer gant \emph{par}
(da):

\section{Anv-gwan hep anv ouzh e
heul}\label{anv_gwan_hep_anv_ouzh_e_heul}

E degouezhioù ma n\textquotesingle eus anv ebet ouzh an anv-gwan e vez
graet gantañ evel gant un anv-kadarn gwirion:
