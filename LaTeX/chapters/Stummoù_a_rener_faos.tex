\chapter{Stummoù a rener faos}

"A-wechoù e c\textquotesingle haller kaout e penn ul lavarenn gerioù hag
o deus an neuz da vezañ rener pe renadenn eeuun d\textquotesingle ar
verb, hag en deus koulskoude ur rener pe ur renadenn eeun all"
(\href{Kervella_1995}{Kervella 1995}, 393):

\begin{longtable}[]{p{0.5\textwidth}p{0.5\textwidth}}

Ha gwelet e vo ar pezh a vo gwelet, rak Katell a oa droug enni. & Et
l\textquotesingle on verrait ce qu\textquotesingle on verrait, car
Katell était bien en colère. \\
Drezen 2012, 67 & Drezen 2002, 42 \\
\end{longtable}

\section{Terminologiezh}\label{terminologiezh}

Ar ger-diaraok n\textquotesingle eo na rener na renadenn eeun
(\href{Kervella_1995}{Kervella 1995}, 420-421)

\section{Implijoù a stummoù a rener faos en droidigezh
vrezhonek}\label{implijouxf9_a_stummouxf9_a_rener_faos_en_droidigezh_vrezhonek}

\subsection{Deskrivañ an neuz}\label{deskrivauxf1_an_neuz}

\begin{longtable}[]{p{0.5\textwidth}p{0.5\textwidth}}

Quand Thomas avait vu pour la première fois ce prêtre {[}\ldots{]}, il
s\textquotesingle était échappé de l\textquotesingle église pour avertir
ses parents que l\textquotesingle ile tenait un prêtre différent des
autres et qui ne s\textquotesingle en irait pas~: -- Celui-là sera
enterré dans le cimetière. & Kentañ gwech en doa Tomaz gwelet ar beleg
{[}\ldots{]} e oa tec\textquotesingle het eus an iliz da gemenn
d\textquotesingle e dud he doa tapet an enez kaout ur beleg disheñvel
diouzh ar re all ha na \textquotesingle z afe ket kuit~: -- Henne vo
intiarret er vered. \\
Queffélec 1970, 34 & Beyer 2016, 14 \\
\end{longtable}

\begin{longtable}[]{p{0.5\textwidth}p{0.5\textwidth}}

Brun, de petite taille et la voix monotone, il avait seulement une
poitrine mince et, parfois, un regard craintif, mais il vivait en mer
autant que les autres. & Duard e oa, bihan e vent hag unton e vouezh,
hag a-wechoù en deveze ur sell aonik. Met kement hag ar re all e veve
war vor \\
Queffélec 1970, 39 & Beyer 2016, 39 \\
\end{longtable}

\subsection{Treiñ lavarennoù
kenurzhiet}\label{treiuxf1_lavarennouxf9_kenurzhiet}

\begin{longtable}[]{p{0.5\textwidth}p{0.5\textwidth}}

Une brume tombe, mouillée comme une grève après la marée, une brume qui
sent le sable et le sel. & Ha neuze morlusenn o tont, ken gleb hag un
aod goude al lanv, ul lusenn frond an traezh hag an holen ganti. \\
Queffélec 1970, 6 & Beyer 2016, 8 \\
\end{longtable}

\begin{longtable}[]{p{0.5\textwidth}p{0.5\textwidth}}

On entendait gronder la bise comme un chien qu\textquotesingle un autre
chien irrite. & Klevet e veze an avel-viz o krozal evel ur
c\textquotesingle hi savet droug ennañ ouzh ur c\textquotesingle hi
all. \\
Queffélec 1970, 6 & Beyer 2016, 9 \\
\end{longtable}

\begin{longtable}[]{p{0.5\textwidth}p{0.5\textwidth}}

Edo he mab war he seulioù~: un den yaouank hir, mistr, du e zaoulagad,
du e vlev. & Son fils arrivait sur ses talons~: c\textquotesingle était
un jeune homme élancé, fin, noir d\textquotesingle yeux et de
cheveux. \\
Drezen 2012, 21 & Drezen 2002, 11 \\
\end{longtable}
