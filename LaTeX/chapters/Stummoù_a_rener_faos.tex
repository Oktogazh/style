\chapter{Stummoù a rener faos}

"A-wechoù e c\textquotesingle haller kaout e penn ul lavarenn gerioù hag
o deus an neuz da vezañ rener pe renadenn eeuun d\textquotesingle ar
verb, hag en deus koulskoude ur rener pe ur renadenn eeun all"
(\href{Kervella_1995}{Kervella 1995}, 393):

\section{Terminologiezh}\label{terminologiezh}

Ar ger-diaraok n\textquotesingle eo na rener na renadenn eeun
(\href{Kervella_1995}{Kervella 1995}, 420-421)

\section{Implijoù a stummoù a rener faos en droidigezh
vrezhonek}\label{implijouxf9_a_stummouxf9_a_rener_faos_en_droidigezh_vrezhonek}

\subsection{Deskrivañ an neuz}\label{deskrivauxf1_an_neuz}

\subsection{Treiñ lavarennoù
kenurzhiet}\label{treiuxf1_lavarennouxf9_kenurzhiet}
