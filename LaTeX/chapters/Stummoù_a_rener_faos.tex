\begin{LTR}
\begin{otherlanguage}{breton}

"A-wechoù e c'haller kaout e penn ul lavarenn gerioù hag
o deus an neuz da vezañ rener pe renadenn eeuun d'ar
verb, hag en deus koulskoude ur rener pe ur renadenn eeun all"
(\href{/wiki/Kervella_1995}{Kervella 1995}, 393):

Ha gwelet e vo ar pezh a vo gwelet, rak Katell a oa droug enni.

Drezen 2012, 67

Et l'on verrait ce qu'on verrait, car
Katell était bien en colère.

Drezen 2002, 42

\section{Terminologiezh}\label{Terminologiezh}

{{{[}}\href{/wiki/Stummo\%C3\%B9_a_rener_faos?action=edit&section=1}{{modifier}}{{]}}}

Ar ger-diaraok n'eo na rener na renadenn eeun
(\href{/wiki/Kervella_1995}{Kervella 1995}, 420-421)

\subsection[Implijoù a stummoù a rener faos en droidigezh
vrezhonek]{\texorpdfstring{\protect\hypertarget{Implijo.C3.B9_a_stummo.C3.B9_a_rener_faos_en_droidigezh_vrezhonek}{}{}Implijoù
a stummoù a rener faos en droidigezh
vrezhonek}{Implijoù a stummoù a rener faos en droidigezh vrezhonek}}\label{Implijouxf9_a_stummouxf9_a_rener_faos_en_droidigezh_vrezhonek}

{{{[}}\href{/wiki/Stummo\%C3\%B9_a_rener_faos?action=edit&section=2}{{modifier}}{{]}}}

\subsubsection[Deskrivañ an
neuz]{\texorpdfstring{\protect\hypertarget{Deskriva.C3.B1_an_neuz}{}{}Deskrivañ
an neuz}{Deskrivañ an neuz}}\label{Deskrivauxf1_an_neuz}

{{{[}}\href{/wiki/Stummo\%C3\%B9_a_rener_faos?action=edit&section=3}{{modifier}}{{]}}}

Quand Thomas avait vu pour la première fois ce prêtre {[}\ldots{]}, il
s'était échappé de l'église pour avertir
ses parents que l'ile tenait un prêtre différent des
autres et qui ne s'en irait pas~: -- Celui-là sera
enterré dans le cimetière.

Queffélec 1970, 34

Kentañ gwech en doa Tomaz gwelet ar beleg {[}\ldots{]} e oa
tec'het eus an iliz da gemenn d'e dud he
doa tapet an enez kaout ur beleg disheñvel diouzh ar re all ha na
'z afe ket kuit~: -- Henne vo intiarret er vered.

Beyer 2016, 14

Brun, de petite taille et la voix monotone, il avait seulement une
poitrine mince et, parfois, un regard craintif, mais il vivait en mer
autant que les autres.

Queffélec 1970, 39

Duard e oa, bihan e vent hag unton e vouezh, hag a-wechoù en deveze ur
sell aonik. Met kement hag ar re all e veve war vor

Beyer 2016, 39

\subsubsection[Treiñ lavarennoù
kenurzhiet]{\texorpdfstring{\protect\hypertarget{Trei.C3.B1_lavarenno.C3.B9_kenurzhiet}{}{}Treiñ
lavarennoù
kenurzhiet}{Treiñ lavarennoù kenurzhiet}}\label{Treiuxf1_lavarennouxf9_kenurzhiet}

{{{[}}\href{/wiki/Stummo\%C3\%B9_a_rener_faos?action=edit&section=4}{{modifier}}{{]}}}

Une brume tombe, mouillée comme une grève après la marée, une brume qui
sent le sable et le sel.

Queffélec 1970, 6

Ha neuze morlusenn o tont, ken gleb hag un aod goude al lanv, ul lusenn
frond an traezh hag an holen ganti.

Beyer 2016, 8

On entendait gronder la bise comme un chien qu'un autre
chien irrite.

Queffélec 1970, 6

Klevet e veze an avel-viz o krozal evel ur c'hi savet
droug ennañ ouzh ur c'hi all.

Beyer 2016, 9

Edo he mab war he seulioù~: un den yaouank hir, mistr, du e zaoulagad,
du e vlev.

Drezen 2012, 21

Son fils arrivait sur ses talons~: c'était un jeune
homme élancé, fin, noir d'yeux et de cheveux.

Drezen 2002, 11

\end{otherlanguage}
\end{LTR}
