\chapter{Kinnig ar raktres}
{{{[}}\href{/wiki/Kinnig_ar_raktres?action=edit&section=0}{{modifier}}{{]}}}
\section[Petra eo
STYLE?]{\texorpdfstring{\protect\hypertarget{Petra_eo_STYLE.3F}{}{}Petra
eo STYLE?}{Petra eo STYLE?}}\label{Petra_eo_STYLEux3f}

{{{[}}\href{/wiki/Kinnig_ar_raktres?action=edit&section=1}{{modifier}}{{]}}}

Un akronim brezhonek eo STYLE: \textbf{S}tudi an \textbf{T}roidigezhioù
er \textbf{Y}ezh \textbf{Le}nnegel. Un diaz eo diwar ma enklaskoù war
studi ar stil hag an droidigezh e brezhoneg kroget e 2023 da-geñver ma
c'hounskrid mestroniezh e skol-veur Roazhon 2. Spi am
eus e servijfe ar benveg-mañ d'ar gumuniezh vrezhonek,
en o zoues studierien, troerien ha klaskerien.

Evit keñveriañ framm ar galleg hag ar brezhoneg e tielfennan ur
c'horpus troidigezhioù, ha lakaat a ran war wel ar
patromoù yezhel a gavan ennañ. Evit deskrivañ ar skouerioù tennet eus ar
c'horpus e implijan oberennoù war ar stil hag ar
yezhadur, dreist-holl \href{/wiki/Jouitteau_2009-2024}{Jouitteau
2009-2024} ha \href{/wiki/Kervella_1947}{Kervella 1947} e-touez
mammennoù all. An hentenn enklask-mañ, anvet studi ar stil dre
geñveriañ, a voe kinniget gant
\href{/wiki/Darbelnet_\%26_Vinay_1993}{Darbelnet \& Vinay 1993} diwar o
c'hentelioù war an droidigezh etre ar galleg hag ar saozneg.
Diwezhatoc'h e astennas
\href{/wiki/Rottet_\%26_Moris_2018}{Rottet \& Moris 2018} ar
vetodologiezh-se da studi ar saozneg hag ar c'hembraeg.

Evit ma c'horpus troidigezhioù kentañ e tibabis daou levr:
\href{/wiki/Queff\%C3\%A9lec_1970}{Queffélec 1970} skrivet e galleg ha
troet e brezhoneg gant \href{/wiki/Beyer_2016}{Beyer 2016}, ha
\href{/wiki/Drezen_2012}{Drezen 2012} skrivet e brezhoneg ha troet e
galleg gant ar skrivagner e-unan. En tu all d'al
labourioù-se e tennis un nebeud skourioù ouzhpenn eus
\href{/wiki/Le_Dimna_2005}{Le Dimna 2005} diazezet war skridoù liesseurt
Youenn Drezen. En amzer da zont e vefe dedennus pinvidikaat an
dielfennadur kentañ gant testennoù liesseurt all evit kadarnaat ar
patromoù kavet er c'horpus diazez ha diskuliañ al
liesseurted stil e brezhoneg. Dre geñveriañ an troidigezhioù en daou du
e c'hall an dielfennadur chom dibar, rak diskouez a ra
patromoù yezhoniel a gaver anezho en div yezh, kentoc'h
eget doareoù un aozer resis.

Evit lakaat ma labour aes da gaout ha da furchal ennañ e tibabis ur
stumm niverel dre implijout Media Wiki. A-drugarez d'ar
framm-mañ e vo aesoc'h ledanaat ar
c'horpus troidigezhioù en tu all d'an
daou levr kentañ, skignañ ar raktres ha lakaat tud all da genlabourat
warnañ.

Erfin, stumm niverel al labour embannet dindan un aotre CC-BY a aotreo
ivez tennañ ur c'horpus troidigezhioù kenstur evit
pinvidikaat korpus kriz ar brezhoneg ha stummañ ostilhoù IA (intant
atrifisiel), ar pezh a zo a-bouez bras evit ar yezhoù minorelaet
(\href{/wiki/Jouitteau_2023d}{Jouitteau 2023d}).
