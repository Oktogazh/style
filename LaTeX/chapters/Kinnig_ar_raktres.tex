\chapter{Kinnig ar raktres}

\section{Petra eo STYLE?}\label{petra_eo_style}

Un akronim brezhonek eo STYLE: \textbf{S}tudi an \textbf{T}roidigezhioù
er \textbf{Y}ezh \textbf{Le}nnegel. Un diaz eo diwar ma enklaskoù war
studi ar stil hag an droidigezh e brezhoneg kroget e 2023 da-geñver ma
c\textquotesingle hounskrid mestroniezh e skol-veur Roazhon 2. Spi am
eus e servijfe ar benveg-mañ d\textquotesingle ar gumuniezh vrezhonek,
en o zoues studierien, troerien ha klaskerien.

Evit keñveriañ framm ar galleg hag ar brezhoneg e tielfennan ur
c\textquotesingle horpus troidigezhioù, ha lakaat a ran war wel ar
patromoù yezhel a gavan ennañ. Evit deskrivañ ar skouerioù tennet eus ar
c\textquotesingle horpus e implijan oberennoù war ar stil hag ar
yezhadur, dreist-holl \href{Jouitteau_2009-2024}{Jouitteau 2009-2024} ha
\href{Kervella_1947}{Kervella 1947} e-touez mammennoù all. An hentenn
enklask-mañ, anvet studi ar stil dre geñveriañ, a voe kinniget gant
\href{Darbelnet_&_Vinay_1993}{Darbelnet \& Vinay 1993} diwar o
c'hentelioù war an droidigezh etre ar galleg hag ar saozneg.
Diwezhatoc\textquotesingle h e astennas
\href{Rottet_&_Moris_2018}{Rottet \& Moris 2018} ar vetodologiezh-se da
studi ar saozneg hag ar c\textquotesingle hembraeg.

Evit ma c'horpus troidigezhioù kentañ e tibabis daou levr:
\href{Queffélec_1970}{Queffélec 1970} skrivet e galleg ha troet e
brezhoneg gant \href{Beyer_2016}{Beyer 2016}, ha
\href{Drezen_2012}{Drezen 2012} skrivet e brezhoneg ha troet e galleg
gant ar skrivagner e-unan. En tu all d\textquotesingle al labourioù-se e
tennis un nebeud skourioù ouzhpenn eus \href{Le_Dimna_2005}{Le Dimna
2005} diazezet war skridoù liesseurt Youenn Drezen. En amzer da zont e
vefe dedennus pinvidikaat an dielfennadur kentañ gant testennoù
liesseurt all evit kadarnaat ar patromoù kavet er
c\textquotesingle horpus diazez ha diskuliañ al liesseurted stil e
brezhoneg. Dre geñveriañ an troidigezhioù en daou du e
c\textquotesingle hall an dielfennadur chom dibar, rak diskouez a ra
patromoù yezhoniel a gaver anezho en div yezh, kentoc\textquotesingle h
eget doareoù un aozer resis.

Evit lakaat ma labour aes da gaout ha da furchal ennañ e tibabis ur
stumm niverel dre implijout Media Wiki. A-drugarez d\textquotesingle ar
framm-mañ e vo aesoc\textquotesingle h ledanaat ar
c\textquotesingle horpus troidigezhioù en tu all d\textquotesingle an
daou levr kentañ, skignañ ar raktres ha lakaat tud all da genlabourat
warnañ.

Erfin, stumm niverel al labour embannet dindan un aotre CC-BY a aotreo
ivez tennañ ur c\textquotesingle horpus troidigezhioù kenstur evit
pinvidikaat korpus kriz ar brezhoneg ha stummañ ostilhoù IA (intant
atrifisiel), ar pezh a zo a-bouez bras evit ar yezhoù minorelaet
(\href{Jouitteau_2023d}{Jouitteau 2023d}).
