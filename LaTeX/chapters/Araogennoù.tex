\chapter{Araogennoù}

\section{Digoradur}\label{digoradur}

An araogenn, hervez Kervella (1995, 333), a zo «ur ger hag a vez lakaet dirak ur ger all --- anv-kadarn, anv-verb, anv-gwan pe adverb --- evit e liammañ e doare pe zoare ouzh ur ger all a zeu a-raok». Koulskoude e kaver skouerioù ma ne ra ket un araogenn al liamm gant ur ger all a zeu a-raok, evel pa grog ur frazenn gant un araogenn (\emph{Da skouer}, \emph{War a seblant}...). Un termenadur simploc\textquotesingle h a zeskrivfe an araogennoù evel ur rummad gerioù a dalvez da lec\textquotesingle hiañ, en amzer pe en un egor bennak (fizikel, arouezel, darempredel hag all) da nebeutañ un arguzenn a zo staget outi. Anvet e vez araogenn peogwir e teu an araogennoù a-raok an arguzenn a dalvez da lec\textquotesingle hiañ, koulskoude e c\textquotesingle haller kavout gerioù heñvel lec\textquotesingle hiet goude an arguzenn a zaveont outi (\emph{Sellet a rin goude}, \emph{Mont kuit}, \emph{Mont en-dro} hag all), ma c\textquotesingle haller o envel \emph{lerc\textquotesingle hennoù} neuze. Evel ma lavar Kervella, e c\textquotesingle hoarvez alies d\textquotesingle an araogennoù bezañ etre arguzennoù disheñvel, pa dalvez neuze da dermeniñ natur an darempred etrezo.

Evel ma voe meneget gant Rottet ha Moris (2018, 65), n\textquotesingle eo ket anat a-wechoù kavout an araogenn reizh er yezh tal, dreist-holl pa droer ur frazenn pe un droienn resis. Er pennad-mañ e vo kinniget un nebeud doareoù posupl da implijout ul lod eus araogennoù brezhonek diwar skouerioù tennet eus ar c\textquotesingle horpus troidigezhioù.

E degouezhioù zo e implijer un araogenn gevatal er yezh orin hag er yezh tal:

\begin{longtable}[]{|p{0.45\textwidth}|p{0.45\textwidth}}

Paol, emañ ho koan war an daol. & Paol, votre dîner est sur la table. \\ Drezen 2012, 13 & Drezen 2002, 19 \\ \end{longtable}

E degouezhioù all, avat, e implijer araogennoù disheñvel er yezh orin hag er yezh tal:

\begin{longtable}[]{|p{0.45\textwidth}|p{0.45\textwidth}}

Il a peur de la mer. & Aon en deus rak ar mor. \\ Queffélec 1970, 30 & Beyer 2016, 14 \\ \end{longtable}

\begin{longtable}[]{|p{0.45\textwidth}|p{0.45\textwidth}}

Ur c\textquotesingle harter yac\textquotesingle h, ha bev ouzhpenn, rak, ma n\textquotesingle o doa ket ar baotred \textbf{aon ouzh} o skeud, e vigoudenned a oa lipr o zeod. & C\textquotesingle était un quartier complètement assaini, et resté, quand même, vivant et «~nature~», car si les jeunes gens n\textquotesingle y ont pas \textbf{peur de} leur ombre, ses Bigoudenn ont le don inné de la réplique. \\ Drezen 2012, 249 & Drezen 2002, 155 \\ \end{longtable}

Bez\textquotesingle{} ez eus ivez degouezhioù ma vez un araogenn en ur yezh hag ezvezant en eben:

\begin{longtable}[]{|p{0.45\textwidth}|p{0.45\textwidth}}

An heol war e ziskar a lakae anat roufennoù he dremm dreut. & Le soleil déclinant accusait les rides de son maigre visage. \\ Drezen 2012, 21 & Drezen 2002, 11 \\ \end{longtable}

Er skouer roet amañ-dindan e implij ur yezh ar renadenn dra eeun, hag eben --- ar renadenn dra ameeun.

\begin{longtable}[]{|p{0.45\textwidth}|p{0.45\textwidth}}

Thomas heurta la table du poing. & Tomaz a skoas e veilh-dorn war an daol. \\ Queffélec 1970, 109 & Beyer 2016, 141 \\ \end{longtable}

\section{En}\label{en}

\subsection{Merkañ dalc\textquotesingle had gant an araogenn en}\label{merkauxf1_dalchad_gant_an_araogenn_en}

Gallout a ra servijout an araogenn \emph{en} da lakat pouez war an dalc\textquotesingle had, en degouezh-mañ resis danvez ul lestr:

\begin{longtable}[]{|p{0.45\textwidth}|p{0.45\textwidth}}

Et tout le monde raconte qu\textquotesingle il a trouvé une épave, un baril, de malaga ou de rhum, et croisé pour attendre la nuit: il grimperait chez lui en cachette et enterrerait le baril. & Ha padal e konte an holl en doa kavet ur peñse, ur varilh, malaga pe rom ennañ, hag en doa redet da c\textquotesingle hortoz an noz: da neuze en dije gellet pignat d\textquotesingle e lojeiz ha kuzhat ar varilhad. \\ Queffélec 1970, 6 & Beyer 2016, 8 \\ \end{longtable}

Disheñvel e vefe ster ar frazenn \emph{ur varilh malaga pe rom} hepken, rak talvezout a rafe e komzer diwar-benn al lestr end-eun ha neket e zalc\textquotesingle had.

Un doare all da dreuzkas mennozh al lestr leun a vefe skrivañ \emph{ur varilhad malaga pe rom}. Muioc\textquotesingle h a ditouroù diwar-benn ar stumm-se a zo kinniget amañ.

\section{Gant}\label{gant}

\subsection{Merkañ an dalc\textquotesingle h}\label{merkauxf1_an_dalch}

Liammet eo implij ar stumm \emph{gant} evit treiñ \emph{détenir} gant ster ar verb-se e galleg: «avoir entre les mains, à sa disposition, légalement ou illégalement quelque chose qui appartient à autrui» Geriadur CNRTL.

Ne dalvez ket \emph{détenir} bezañ perc\textquotesingle henn war un dra, met kentoc\textquotesingle h kaout un dra en e zalc\textquotesingle h. Ne c'haller ket implijout ar verb \emph{kaout} er ster-se.

\begin{longtable}[]{|p{0.45\textwidth}|p{0.45\textwidth}}

Ul lizher a oa gantañ en e zorn. & Il tenait une lettre à la main. \\ Drezen 2012, 169 & Drezen 2002, 106 \\ \end{longtable}

\begin{longtable}[]{|p{0.45\textwidth}|p{0.45\textwidth}}

Ils détiennent une science effrayante et, lui, pauvre terrien, il ne lutte pas contre eux. & Ganto ez eus ur skiant efreizhus hag ar paourkaezh douarad anezhañ ne zalc\textquotesingle h ket penn outo. \\ Queffélec 1970, 7 & Beyer 2016, 9 \\ \end{longtable}

\section{Stagelloù araogenn}\label{stagellouxf9_araogenn}

\section{En desped (da)}\label{en_desped_da}

Ar stagell araogenn \emph{en desped (da)} a glot gant \emph{cependant / malgré / en dépit} de e galleg.

\begin{longtable}[]{|p{0.45\textwidth}|p{0.45\textwidth}}

Il tergiversa cependant. & En desped da se e klaskas troiata. \\ Queffélec 1970, 53 & Beyer 2016, 33 \\ \end{longtable}

\begin{longtable}[]{|p{0.45\textwidth}|p{0.45\textwidth}}

Skedus e oa an oabl, en desped d\textquotesingle ur vogedennig tanav, adalek Benoded, neizhet er gwez, betek an Enez-Tudi, gwenn-kann war he zurumellig goemonek-rous. & Le ciel était éclatant, en dépit d\textquotesingle une buée transparente, depuis Benn-Odet, niché dans les arbres, jusqu\textquotesingle à l\textquotesingle île Tudi, toute blanche sur sa petite butte de goémons roux. \\ Drezen 2012, 138 & Drezen 2002, 85 \\ \end{longtable}

\section{\texorpdfstring{Araogenn + \emph{a} + verb}{Araogenn + a + verb}}\label{araogenn_a_verb}

Pa vez lakaet un araogenn + a + verb, \emph{a} a zo ur berradur evit \emph{ar pezh a}.

\subsection{war a}\label{war_a}

Implijout a c\textquotesingle haller an araogenn \textbf{\emph{war}} dirak ar rannig verb \textbf{\emph{a\textbf{\textquotesingle\textquotesingle{} gant ar ster}}war a pezh a lavar ar vemor}\textquotesingle\textquotesingle:

\begin{longtable}[]{|p{0.45\textwidth}|p{0.45\textwidth}}

La mémoire prétend que des alouettes, dans cet après-midi de juin, chantaient parmi les haubans du ciel. & War a lavar ar vemor e oa alc\textquotesingle hwedered, en endervezh-se a viz Mezheven, o kanañ e-touez skeulioù-gwern an oabl. \\ Queffélec 1970, 6 & Beyer 2016, 8 \\ \end{longtable}

\begin{longtable}[]{|p{0.45\textwidth}|p{0.45\textwidth}}

Hounnezh ne zeu ket alies d\textquotesingle an oferenn, war a gredan. & Cette jeune personne ne vient pas souvent à la messe, crois-je savoir. \\ Drezen 2012, 46 & Drezen 2002, 27 \\ \end{longtable}

Ar skouer-se a ziskouez penaos e teu un islavarenn-stag da vezañ un islavarenn-stag hep penn.

\section{Da lenn pelloc\textquotesingle h war ar gudenn}\label{da_lenn_pelloch_war_ar_gudenn}

Ar pennad-mañ a c\textquotesingle hallfed astenn gant an oberennoù da-heul:

\begin{itemize} \tightlist \item   Gourmelon, Yvon. 2012. \textquotesingle Div araogenn dirak ar memes   anv\textquotesingle, Notennoù yezhadur, Al Liamm (éd.), 63-66.   {[}adembannet e 2008. Al Liamm 368, 81-84{]}. \item   Gourmelon, Yvon. 2012. \textquotesingle Displegañ an araogennoù pe   chom hep ober ?\textquotesingle, Notennoù yezhadur, Al Liamm, 59-62. \item   Morin, Guillaume. 2024. Studiañ un nebeud araogennoù brezhonek diwar   skridoù eñvorennoù Jarl Priel, ms. master, u. Rennes II \item   Rottet, Kevin J. 2020. \textquotesingle Complex prepositions in   Breton\textquotesingle, Benjamin Fagard, José Pinto de Lima, Dejan   Stošić, \& Elena Smirnova (éds.), Complex Adpositions in European   Languages: A Micro-Typological Approach to Complex Nominal Relators,   Series Empirical Approaches to Language Typology (EALT) Vol 65,   Berlin: De Gruyter, 195-231. \end{itemize}

Rummad: Pennadoù 