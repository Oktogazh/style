\chapter{Araogennoù}

\section{Digoradur}\label{digoradur}

An \url{araogenn}, hervez \href{Kervella_1995}{Kervella (1995, 333)}, a
zo «ur ger hag a vez lakaet dirak ur ger all --- \url{anv-kadarn},
\url{anv-verb}, \url{anv-gwan} pe adverb --- evit e liammañ e doare pe
zoare ouzh ur ger all a zeu a-raok». Koulskoude e kaver skouerioù ma ne
ra ket un araogenn al liamm gant ur ger all a zeu a-raok, evel pa grog
ur frazenn gant un araogenn (\emph{Da skouer}, \emph{War a seblant}...).
Un termenadur simploc\textquotesingle h a zeskrivfe an araogennoù evel
ur rummad gerioù a dalvez da lec\textquotesingle hiañ, en amzer pe en un
egor bennak (fizikel, arouezel, darempredel hag all) da nebeutañ un
\href{https://arbres.iker.cnrs.fr/index.php?title=Argument}{arguzenn} a
zo staget outi. Anvet e vez araogenn peogwir e teu an araogennoù a-raok
an arguzenn a dalvez da lec\textquotesingle hiañ, koulskoude e
c\textquotesingle haller kavout gerioù heñvel lec\textquotesingle hiet
goude an arguzenn a zaveont outi (\emph{Sellet a rin goude}, \emph{Mont
kuit}, \emph{Mont en-dro} hag all), ma c\textquotesingle haller o envel
\href{https://arbres.iker.cnrs.fr/index.php?title=Postpositions}{\emph{lerc\textquotesingle hennoù}}
neuze. Evel ma lavar Kervella, e c\textquotesingle hoarvez alies
d\textquotesingle an araogennoù bezañ etre arguzennoù disheñvel, pa
dalvez neuze da dermeniñ natur an darempred etrezo.

Evel ma voe meneget gant \href{Rottet_&_Moris_2018}{Rottet ha Moris
(2018, 65)}, n\textquotesingle eo ket anat a-wechoù kavout an araogenn
reizh er yezh tal, dreist-holl pa droer ur frazenn pe un
\href{troienn}{droienn} resis. Er pennad-mañ e vo kinniget un nebeud
doareoù posupl da implijout ul lod eus araogennoù brezhonek diwar
skouerioù tennet eus ar c\textquotesingle horpus troidigezhioù.

E degouezhioù zo e implijer un araogenn gevatal er yezh orin hag er yezh
tal:

E degouezhioù all, avat, e implijer araogennoù disheñvel er yezh orin
hag er yezh tal:

Bez\textquotesingle{} ez eus ivez degouezhioù ma vez un araogenn en ur
yezh hag ezvezant en eben:

Er skouer roet amañ-dindan e implij ur yezh ar renadenn dra eeun, hag
eben --- ar renadenn dra ameeun.

\section{En}\label{en}

\subsection{Merkañ dalc\textquotesingle had gant an araogenn
en}\label{merkauxf1_dalchad_gant_an_araogenn_en}

Gallout a ra servijout an araogenn \emph{en} da lakat pouez war an
dalc\textquotesingle had, en degouezh-mañ resis danvez ul lestr:

Disheñvel e vefe ster ar frazenn \emph{ur varilh malaga pe rom} hepken,
rak talvezout a rafe e komzer diwar-benn al lestr end-eun ha neket e
zalc\textquotesingle had.

Un doare all da dreuzkas mennozh al lestr leun a vefe skrivañ \emph{ur
varilhad malaga pe rom}. Muioc\textquotesingle h a ditouroù diwar-benn
ar stumm-se a zo kinniget \href{Lostgerioù}{amañ}.

\section{Gant}\label{gant}

\subsection{Merkañ an
dalc\textquotesingle h}\label{merkauxf1_an_dalch}

Liammet eo implij ar stumm \emph{gant} evit treiñ \emph{détenir} gant
ster ar verb-se e galleg: «avoir entre les mains, à sa disposition,
légalement ou illégalement quelque chose qui appartient à autrui»
\href{https://www.cnrtl.fr/definition/d\%C3\%A9tenir}{Geriadur CNRTL}.

Ne dalvez ket \emph{détenir} bezañ perc\textquotesingle henn war un dra,
met kentoc\textquotesingle h kaout un dra en e zalc\textquotesingle h.
Ne c'haller ket implijout ar verb \emph{kaout} er ster-se.

\section{Stagelloù araogenn}\label{stagellouxf9_araogenn}

\section{En desped (da)}\label{en_desped_da}

Ar \href{stagell_araogenn}{stagell araogenn} \emph{en desped (da)} a
glot gant \emph{cependant / malgré / en dépit} de e galleg.

\section{\texorpdfstring{Araogenn + \emph{a} +
verb}{Araogenn + a + verb}}\label{araogenn_a_verb}

Pa vez lakaet un araogenn + a + verb, \emph{a} a zo ur berradur evit
\emph{ar pezh a}.

\subsection{war a}\label{war_a}

Implijout a c\textquotesingle haller an araogenn \textbf{\emph{war}}
dirak ar rannig verb
\textbf{\emph{a\textbf{\textquotesingle\textquotesingle{} gant ar
ster}}war a pezh a lavar ar vemor}\textquotesingle\textquotesingle:

Ar skouer-se a ziskouez penaos e teu un \url{islavarenn-stag} da vezañ
un \href{islavarenn-stag_hep_penn}{islavarenn-stag hep penn}.

\section{Da lenn pelloc\textquotesingle h war ar
gudenn}\label{da_lenn_pelloch_war_ar_gudenn}

Ar pennad-mañ a c\textquotesingle hallfed astenn gant an oberennoù
da-heul:

\begin{itemize}
\tightlist
\item
  Gourmelon, Yvon. 2012. \textquotesingle Div araogenn dirak ar memes
  anv\textquotesingle, Notennoù yezhadur, Al Liamm (éd.), 63-66.
  {[}adembannet e 2008. Al Liamm 368, 81-84{]}.
\item
  Gourmelon, Yvon. 2012. \textquotesingle Displegañ an araogennoù pe
  chom hep ober ?\textquotesingle, Notennoù yezhadur, Al Liamm, 59-62.
\item
  Morin, Guillaume. 2024. Studiañ un nebeud araogennoù brezhonek diwar
  skridoù eñvorennoù Jarl Priel, ms. master, u. Rennes II
\item
  Rottet, Kevin J. 2020. \textquotesingle Complex prepositions in
  Breton\textquotesingle, Benjamin Fagard, José Pinto de Lima, Dejan
  Stošić, \& Elena Smirnova (éds.), Complex Adpositions in European
  Languages: A Micro-Typological Approach to Complex Nominal Relators,
  Series Empirical Approaches to Language Typology (EALT) Vol 65,
  Berlin: De Gruyter, 195-231.
\end{itemize}

\href{Rummad:_Pennadoù}{Rummad: Pennadoù}
