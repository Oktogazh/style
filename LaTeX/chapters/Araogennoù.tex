\begin{LTR}
\begin{otherlanguage}{breton}

\section{Digoradur}\label{Digoradur}

{{{[}}\href{/wiki/Araogenno\%C3\%B9?action=edit&section=1}{{modifier}}{{]}}}

An \href{/wiki/Araogenn}{araogenn}, hervez
\href{/wiki/Kervella_1995}{Kervella (1995, 333)}, a zo «ur ger hag a vez
lakaet dirak ur ger all --- \href{/wiki/Anv-kadarn}{anv-kadarn},
\href{/wiki/Anv-verb}{anv-verb}, \href{/wiki/Anv-gwan}{anv-gwan} pe
adverb --- evit e liammañ e doare pe zoare ouzh ur ger all a zeu
a-raok». Koulskoude e kaver skouerioù ma ne ra ket un araogenn al liamm
gant ur ger all a zeu a-raok, evel pa grog ur frazenn gant un araogenn
(\emph{Da skouer}, \emph{War a seblant}...). Un termenadur
simploc'h a zeskrivfe an araogennoù evel ur rummad
gerioù a dalvez da lec'hiañ, en amzer pe en un egor
bennak (fizikel, arouezel, darempredel hag all) da nebeutañ un
\href{https://arbres.iker.cnrs.fr/index.php?title=Argument}{arguzenn} a
zo staget outi. Anvet e vez araogenn peogwir e teu an araogennoù a-raok
an arguzenn a dalvez da lec'hiañ, koulskoude e
c'haller kavout gerioù heñvel lec'hiet
goude an arguzenn a zaveont outi (\emph{Sellet a rin goude}, \emph{Mont
kuit}, \emph{Mont en-dro} hag all), ma c'haller o envel
\href{https://arbres.iker.cnrs.fr/index.php?title=Postpositions}{\emph{lerc'hennoù}}
neuze. Evel ma lavar Kervella, e c'hoarvez alies
d'an araogennoù bezañ etre arguzennoù disheñvel, pa
dalvez neuze da dermeniñ natur an darempred etrezo.

Evel ma voe meneget gant \href{/wiki/Rottet_\%26_Moris_2018}{Rottet ha
Moris (2018, 65)}, n'eo ket anat a-wechoù kavout an
araogenn reizh er yezh tal, dreist-holl pa droer ur frazenn pe un
\href{/wiki/Troienn}{droienn} resis. Er pennad-mañ e vo kinniget un
nebeud doareoù posupl da implijout ul lod eus araogennoù brezhonek diwar
skouerioù tennet eus ar c'horpus troidigezhioù.

E degouezhioù zo e implijer un araogenn gevatal er yezh orin hag er yezh
tal:

Paol, emañ ho koan war an daol.

Drezen 2012, 13

Paol, votre dîner est sur la table.

Drezen 2002, 19

E degouezhioù all, avat, e implijer araogennoù disheñvel er yezh orin
hag er yezh tal:

Il a peur de la mer.

Queffélec 1970, 30

Aon en deus rak ar mor.

Beyer 2016, 14

Ur c'harter yac'h, ha bev ouzhpenn, rak,
ma n'o doa ket ar baotred \textbf{aon ouzh} o skeud, e
vigoudenned a oa lipr o zeod.

Drezen 2012, 249

C'était un quartier complètement assaini, et resté,
quand même, vivant et «~nature~», car si les jeunes gens
n'y ont pas \textbf{peur de} leur ombre, ses Bigoudenn
ont le don inné de la réplique.

Drezen 2002, 155

Bez\textquotesingle{} ez eus ivez degouezhioù ma vez un araogenn en ur
yezh hag ezvezant en eben:

An heol war e ziskar a lakae anat roufennoù he dremm dreut.

Drezen 2012, 21

Le soleil déclinant accusait les rides de son maigre visage.

Drezen 2002, 11

Er skouer roet amañ-dindan e implij ur yezh ar renadenn dra eeun, hag
eben --- ar renadenn dra ameeun.

Thomas heurta la table du poing.

Queffélec 1970, 109

Tomaz a skoas e veilh-dorn war an daol.

Beyer 2016, 141

\subsection{En}\label{En}

{{{[}}\href{/wiki/Araogenno\%C3\%B9?action=edit&section=2}{{modifier}}{{]}}}

\subsubsection[Merkañ dalc'had gant an araogenn
en]{\texorpdfstring{\protect\hypertarget{Merka.C3.B1_dalc.27had_gant_an_araogenn_en}{}{}Merkañ
dalc'had gant an araogenn
en}{Merkañ dalc'had gant an araogenn en}}\label{Merkauxf1_dalcux27had_gant_an_araogenn_en}

{{{[}}\href{/wiki/Araogenno\%C3\%B9?action=edit&section=3}{{modifier}}{{]}}}

Gallout a ra servijout an araogenn \emph{en} da lakat pouez war an
dalc'had, en degouezh-mañ resis danvez ul lestr:

Et tout le monde raconte qu'il a trouvé une épave, un
baril, de malaga ou de rhum, et croisé pour attendre la nuit: il
grimperait chez lui en cachette et enterrerait le baril.

Queffélec 1970, 6

Ha padal e konte an holl en doa kavet ur peñse, ur varilh, malaga pe rom
ennañ, hag en doa redet da c'hortoz an noz: da neuze en
dije gellet pignat d'e lojeiz ha kuzhat ar varilhad.

Beyer 2016, 8

Disheñvel e vefe ster ar frazenn \emph{ur varilh malaga pe rom} hepken,
rak talvezout a rafe e komzer diwar-benn al lestr end-eun ha neket e
zalc'had.

Un doare all da dreuzkas mennozh al lestr leun a vefe skrivañ \emph{ur
varilhad malaga pe rom}. Muioc'h a ditouroù diwar-benn
ar stumm-se a zo kinniget \href{/wiki/Lostgerio\%C3\%B9}{amañ}.

\subsection{Gant}\label{Gant}

{{{[}}\href{/wiki/Araogenno\%C3\%B9?action=edit&section=4}{{modifier}}{{]}}}

\subsubsection[Merkañ an
dalc'h]{\texorpdfstring{\protect\hypertarget{Merka.C3.B1_an_dalc.27h}{}{}Merkañ
an
dalc'h}{Merkañ an dalc'h}}\label{Merkauxf1_an_dalcux27h}

{{{[}}\href{/wiki/Araogenno\%C3\%B9?action=edit&section=5}{{modifier}}{{]}}}

Liammet eo implij ar stumm \emph{gant} evit treiñ \emph{détenir} gant
ster ar verb-se e galleg: «avoir entre les mains, à sa disposition,
légalement ou illégalement quelque chose qui appartient à autrui»
\href{https://www.cnrtl.fr/definition/d\%C3\%A9tenir}{Geriadur CNRTL}.

Ne dalvez ket \emph{détenir} bezañ perc'henn war un dra,
met kentoc'h kaout un dra en e zalc'h.
Ne c'haller ket implijout ar verb \emph{kaout} er ster-se.

Ul lizher a oa gantañ en e zorn.

Drezen 2012, 169

Il tenait une lettre à la main.

Drezen 2002, 106

Ils détiennent une science effrayante et, lui, pauvre terrien, il ne
lutte pas contre eux.

Queffélec 1970, 7

Ganto ez eus ur skiant efreizhus hag ar paourkaezh douarad anezhañ ne
zalc'h ket penn outo.

Beyer 2016, 9

\section[Stagelloù
araogenn]{\texorpdfstring{\protect\hypertarget{Stagello.C3.B9_araogenn}{}{}Stagelloù
araogenn}{Stagelloù araogenn}}\label{Stagellouxf9_araogenn}

{{{[}}\href{/wiki/Araogenno\%C3\%B9?action=edit&section=6}{{modifier}}{{]}}}

\subsection[En desped
(da)]{\texorpdfstring{\protect\hypertarget{En_desped_.28da.29}{}{}En
desped (da)}{En desped (da)}}\label{En_desped_ux28daux29}

{{{[}}\href{/wiki/Araogenno\%C3\%B9?action=edit&section=7}{{modifier}}{{]}}}

Ar \href{/wiki/Stagell_araogenn}{stagell araogenn} \emph{en desped (da)}
a glot gant \emph{cependant / malgré / en dépit} de e galleg.

Il tergiversa cependant.

Queffélec 1970, 53

En desped da se e klaskas troiata.

Beyer 2016, 33

Skedus e oa an oabl, en desped d'ur vogedennig tanav,
adalek Benoded, neizhet er gwez, betek an Enez-Tudi, gwenn-kann war he
zurumellig goemonek-rous.

Drezen 2012, 138

Le ciel était éclatant, en dépit d'une buée
transparente, depuis Benn-Odet, niché dans les arbres,
jusqu'à l'île Tudi, toute blanche sur sa
petite butte de goémons roux.

Drezen 2002, 85

\subsection[Araogenn + \emph{a} +
verb]{\texorpdfstring{\protect\hypertarget{Araogenn_.2B_a_.2B_verb}{}{}Araogenn
+ \emph{a} + verb}{Araogenn + a + verb}}\label{Araogenn_+_a_+_verb}

{{{[}}\href{/wiki/Araogenno\%C3\%B9?action=edit&section=8}{{modifier}}{{]}}}

Pa vez lakaet un araogenn + a + verb, \emph{a} a zo ur berradur evit
\emph{ar pezh a}.

\subsubsection{war a}\label{war_a}

{{{[}}\href{/wiki/Araogenno\%C3\%B9?action=edit&section=9}{{modifier}}{{]}}}

Implijout a c'haller an araogenn \emph{\textbf{war}}
dirak ar rannig verb \emph{\textbf{a}} gant ar ster \emph{\textbf{war a
pezh a lavar ar vemor}}:

La mémoire prétend que des alouettes, dans cet après-midi de juin,
chantaient parmi les haubans du ciel.

Queffélec 1970, 6

War a lavar ar vemor e oa alc'hwedered, en endervezh-se
a viz Mezheven, o kanañ e-touez skeulioù-gwern an oabl.

Beyer 2016, 8

Hounnezh ne zeu ket alies d'an oferenn, war a gredan.

Drezen 2012, 46

Cette jeune personne ne vient pas souvent à la messe, crois-je savoir.

Drezen 2002, 27

Ar skouer-se a ziskouez penaos e teu un
\href{/wiki/Islavarenn-stag}{islavarenn-stag} da vezañ un
\href{/wiki/Islavarenn-stag_hep_penn}{islavarenn-stag hep penn}.

\section[Da lenn pelloc'h war ar
gudenn]{\texorpdfstring{\protect\hypertarget{Da_lenn_pelloc.27h_war_ar_gudenn}{}{}Da
lenn pelloc'h war ar
gudenn}{Da lenn pelloc'h war ar gudenn}}\label{Da_lenn_pellocux27h_war_ar_gudenn}

{{{[}}\href{/wiki/Araogenno\%C3\%B9?action=edit&section=10}{{modifier}}{{]}}}

Ar pennad-mañ a c'hallfed astenn gant an oberennoù
da-heul:

\begin{itemize}
\tightlist
\item
  Gourmelon, Yvon. 2012. 'Div araogenn dirak ar memes
  anv\textquotesingle, Notennoù yezhadur, Al Liamm (éd.), 63-66.
  {[}adembannet e 2008. Al Liamm 368, 81-84{]}.
\item
  Gourmelon, Yvon. 2012. 'Displegañ an araogennoù pe
  chom hep ober~?\textquotesingle, Notennoù yezhadur, Al Liamm, 59-62.
\item
  Morin, Guillaume. 2024. Studiañ un nebeud araogennoù brezhonek diwar
  skridoù eñvorennoù Jarl Priel, ms. master, u. Rennes II
\item
  Rottet, Kevin J. 2020. 'Complex prepositions in
  Breton\textquotesingle, Benjamin Fagard, José Pinto de Lima, Dejan
  Stošić, \& Elena Smirnova (éds.), Complex Adpositions in European
  Languages: A Micro-Typological Approach to Complex Nominal Relators,
  Series Empirical Approaches to Language Typology (EALT) Vol 65,
  Berlin: De Gruyter, 195-231.
\end{itemize}

\end{otherlanguage}
\end{LTR}
