\chapter{Verboù ha stummoù damskoazellañ e brezhoneg}

\section{Petra eo ar vodelezh ?}\label{petra_eo_ar_vodelezh}

Termeniñ a ra Hacquard (\href{Hacquard_2009}{2009}, 1) ar vodelezh evel
perzh ar ster a denn d\textquotesingle ar martezeadennoù hag ezhommoù pe
redioù, da lavaret eo stadoù an traoù en tu all d\textquotesingle ar
saviad a-vremañ. Daoust ma ne c\textquotesingle hoarvez biken an
darvoudoù hipotetek-se e c\textquotesingle haller komz diwar o fenn en
ur implijout gerioù damskoazellañ.

Gallout a reer komz diwar-benn ar pezh a zo ret/arabat ober evit doujañ
ouzh al lezenn, ar pezh a fello d\textquotesingle an nen ober goude
bezañ echuet e studioù, ar pezh a rafed ma c\textquotesingle hounezfed
d\textquotesingle al loto pe ar pezh a c\textquotesingle hallje
c\textquotesingle hoarvezout ma ne vije ket Breizh aloubet gant
Bro-C\textquotesingle hall.

Hervez \href{Jouitteau_2009-2024}{Jouitteau} e vez eztaolet gant stummoù
damskoazellañ mennozhioù evel ar posublded pe an ezhomm. Gallout a ra
merkañ ar verboù damskoazellañ un derez a brobablentez ivez. Er ster-se
e servijont evel merkerioù epistemek hag a ziskouez live fiziañs ar
c\textquotesingle homzer e-keñver e brezeg. An epistemegezh a
c\textquotesingle haller seveniñ er yezh gant meur a elfenn all ouzhpenn
ar verboù damskoazellañ, en o zouez adverboù evel
\textquotesingle marteze\textquotesingle. E-touez elfennoù damskoazellañ
e brezhoneg e kaver ar verboù (rankout, dleout, faotañ, mennout,
fellout, plijout, gallout), predikadoù damskoazellañ a rummadoù
disheñvel (anvioù-gwan evel \textquotesingle dav\textquotesingle{} pe
\textquotesingle ral\textquotesingle, anvioù-gwan-verb evel
\textquotesingle darbet\textquotesingle, pe an anv damskoazellañ
\textquotesingle ret\textquotesingle) ha stummoù damskoazellañ evel
\textquotesingle na vezañ evit\textquotesingle{}
(\href{https://arbres.iker.cnrs.fr/index.php?title=Constructions_modales}{ARBRES
: Constructions modales}).

\section{Verboù damskoazellañ}\label{verbouxf9_damskoazellauxf1}

\subsection{Gallout}\label{gallout}

Liesster hag amster eo ar verb \emph{gallout} e brezhoneg, evel e meur a
yezh all. Gallout a reer kompren ar frazenn \emph{Il peut venir} e daou
ster disheñvel: \emph{Posubl eo dezhañ dont} pe \emph{Marteze e teuio}.

E degouezhioù liesseurt e vez implijet ar verb-se :

\paragraph{Evit diskouez ur varregezh
fizikel}\label{evit_diskouez_ur_varregezh_fizikel}

\begin{longtable}[]{@{}ll@{}}
\toprule\noalign{}
\endhead
\bottomrule\noalign{}
\endlastfoot
Personne n'avait pu rien voir ni rien entendre.... & Gant den ebet ne oa
bet gellet klevout na gwelout tra... \\
Queffélec 1970, 208 & Beyer 2016, 188 \\
\end{longtable}

\paragraph{Evit treuzkas un ezhomm pe ur
redi}\label{evit_treuzkas_un_ezhomm_pe_ur_redi}

\begin{longtable}[]{@{}ll@{}}
\toprule\noalign{}
\endhead
\bottomrule\noalign{}
\endlastfoot
Le recteur ne tendit pas la main et le pêcheur n'eut d'autre ressource
que de les poser sur une table. & Ar person ne astennas ket e zorn ha ne
c'hellas ar pesketaer nemet o lakaat war an daol. \\
Queffélec 1970, 69 & Beyer 2016, 50 \\
\end{longtable}

\begin{longtable}[]{@{}ll@{}}
\toprule\noalign{}
\endhead
\bottomrule\noalign{}
\endlastfoot
Les jours où l'état de la mer interdisait aux barques de sortir, les
hommes attrapaient les oiseaux. & En deizioù ma ne c'helle ket ar wazed
mont war vor abalamour d'e stad e tapent laboused. \\
Queffélec 1970, 125 & Beyer 2016, 103 \\
\end{longtable}

\paragraph{Evit eztaoler asurded pe
bosublentez}\label{evit_eztaoler_asurded_pe_bosublentez}

\begin{longtable}[]{@{}ll@{}}
\toprule\noalign{}
\endhead
\bottomrule\noalign{}
\endlastfoot
Certainement Dieu n'avait pu se tromper et le sens de cette histoire se
cachait quelque part, mais non dans les commentaires. & A dra sur n'en
doa ket gellet Doue faziañ ha kuzhet e oa ster an istor-se tu bennak met
ne oa ket en addisplegoù. \\
Queffélec 1970, 177 & Beyer 2016, 157 \\
\end{longtable}

\begin{longtable}[]{@{}ll@{}}
\toprule\noalign{}
\endhead
\bottomrule\noalign{}
\endlastfoot
Il n\textquotesingle est pas possible~que des aubes glorieuses, se
déployant dans le fond du ciel, aient éclairé ce morceau de récif. & Ne
c\textquotesingle hell ket bezañ bet~ruzelloù-mintin o sklêrijennañ an
tamm penn-karreg-se en o dispak glorius e don an oabl. \\
Queffélec 1970, 5 & Beyer 2016, 7 \\
\end{longtable}

\begin{longtable}[]{@{}ll@{}}
\toprule\noalign{}
\endhead
\bottomrule\noalign{}
\endlastfoot
«~C'est possible, répondaient les femmes, mais le bon Dieu nous punit
trop.~» & «~E c'hell bezañ,~» a eilgerie ar maouezed «~met re e vezomp
kastizet gant an aotrou Doue~». \\
Queffélec 1970, 40 & Beyer 2016, 20 \\
\end{longtable}

\paragraph{Eztaoler ur merzhadur
(perception)}\label{eztaoler_ur_merzhadur_perception}

\begin{longtable}[]{@{}ll@{}}
\toprule\noalign{}
\endhead
\bottomrule\noalign{}
\endlastfoot
Avec cela, parfois, les journées étaient claires, un ciel vif heurtait
les fenêtres, quelques moments encore et il semblait qu'on apercevrait
le vent, qui ployait l'air et frappait la mer selon la brusquerie d'une
houssine. & Dont a rae an oabl lemm da skeiñ ouzh ar prenestri, ul
lajadig c'hoazh hag e c'helled soñjal damwelout an avel o plegañ an aer,
o vazhata ar mor gant taerder ur gelennenn. \\
Queffélec 1970, 59 & Beyer 2016, 39 \\
\end{longtable}

\paragraph{Implijet e vez ar stumm dic\textquotesingle hour a verb
gallout e brezhoneg evit treiñ verboù raganv
gallek}\label{implijet_e_vez_ar_stumm_dichour_a_verb_gallout_e_brezhoneg_evit_treiuxf1_verbouxf9_raganv_gallek}

\begin{longtable}[]{@{}ll@{}}
\toprule\noalign{}
\endhead
\bottomrule\noalign{}
\endlastfoot
Ils étaient ici comme dans un navire, et un navire n'a point de prêtre,
mais un jour le navire aborde en pays chrétien et, du pont, des clochers
se contemplent sur le rivage. & Evel en ul lestr e oant hag e-bourzh ul
lestr n'eus ket a veleg, met un deiz e teu al lestr da zouarañ en ur vro
gristen ha diwar ar pont e c'heller arvestiñ ouzh ar kloc'hdioù war an
aod. \\
Queffélec 1970, 61 & Beyer 2016, 41 \\
\end{longtable}

\begin{longtable}[]{@{}ll@{}}
\toprule\noalign{}
\endhead
\bottomrule\noalign{}
\endlastfoot
Le sable, d'un blanc faible et un peu jaunâtre, se devinait quelque
temps et se perdait contre les souples murailles de l'ombre qui se
dressaient dans la nuit vague. & Ur predig e c'helled damwelout an
traezh, gwenn peñver hag arvelen un disterañ, ha goude ez ae d'en em
goll ouzh mogerennoù gwevn an deñvalijenn a save en noz dispis. \\
Queffélec 1970, 64 & Beyer 2016, 45 \\
\end{longtable}

\subsection{Rankout}\label{rankout}

\paragraph{Mennozh ar rank}\label{mennozh_ar_rank}

\begin{longtable}[]{@{}ll@{}}
\toprule\noalign{}
\endhead
\bottomrule\noalign{}
\endlastfoot
Noblesse oblige. & Diouzh e renk e rank an unan ober. \\
Queffélec 1970, 127 & Beyer 2016, 104 \\
\end{longtable}

\paragraph{An ezhomm pe ar redi}\label{an_ezhomm_pe_ar_redi}

\begin{longtable}[]{@{}ll@{}}
\toprule\noalign{}
\endhead
\bottomrule\noalign{}
\endlastfoot
L'île, comme une place forte, lance une sortie à chaque marée basse~;
elle récupère ses douves, elle pille le camp ennemi~; vient le flux, et
l'île se retranche dans son enceinte, dont les assiégeants battent les
murs~: ils ont perdu la plus grande partie de leurs troupes et de leur
énergie sur toutes les pentes, barrées d'écueils, qu'il a fallu
franchir. & Par d'ur c'hreñvlec'h e ra an enez un disailhadenn da vare
pep izelvor; azpiaouañ a ra he douvezioù ha preizhañ kamp an enebour;
dont a ra ar chal hag an enez neuze d'en em gaeañ e gwarez he gourizad
tra ma vez ar sezizerien o tagañ he mogerioù: kollet o deus ar braz eus
o bagadoù hag o holl startijenn war an holl savioù sparlet gant
pennoù-kerreg o deus ranket treuziñ. \\
Queffélec 1970, 137 & Beyer 2016, 115 \\
\end{longtable}

\paragraph{Ur raktres pe un obererezh gortozet digant unan
bennak}\label{ur_raktres_pe_un_obererezh_gortozet_digant_unan_bennak}

\begin{longtable}[]{@{}ll@{}}
\toprule\noalign{}
\endhead
\bottomrule\noalign{}
\endlastfoot
Elle avait dû se retirer avant que la barque fût halée sur la dune,
mais, ce matin, toute seule, elle s'était rendue dans la famille du
pêcheur y prendre de ses nouvelles et le remercier. & Ranket he doa mont
kuit a-raok na vefe bet sachet ar bark war an tevenn met diouzh ar
mintin e oa bet, en hec'h-unanpenn, e ti ar pesketaer yaouank da gaout
keloù ha d'e drugarekaat. \\
Queffélec 1970, 151 & Beyer 2016, 130-131 \\
\end{longtable}

\begin{longtable}[]{@{}ll@{}}
\toprule\noalign{}
\endhead
\bottomrule\noalign{}
\endlastfoot
Thomas les rencontrait à chaque séjour sur le continent et, s'il devait
coucher à Audierne, il les réunissait le soir chez le sabotier. & Da bep
chomadenn war an douar bras en em gave Tomaz ganto ha pa ranke kousket e
Gwaien e vode anezho da noz e ti ar botaouer-koad. \\
Queffélec 1970, 176 & Beyer 2016, 156 \\
\end{longtable}

\paragraph{Ur gefridi hiniennel pe un dlead
moral}\label{ur_gefridi_hiniennel_pe_un_dlead_moral}

\begin{longtable}[]{@{}ll@{}}
\toprule\noalign{}
\endhead
\bottomrule\noalign{}
\endlastfoot
Jules dut promettre de se rendre à Saint-Tugen en pèlerinage quand il
serait guéri. & Rankout a rafe Jul mont da birc'hirinaj da Sant-Tujen pa
vefe pare. \\
Queffélec 1970, & Beyer 2016, 140 \\
\end{longtable}

\begin{longtable}[]{@{}ll@{}}
\toprule\noalign{}
\endhead
\bottomrule\noalign{}
\endlastfoot
Quand il s'éloigna, le fils Boulch n'estimait pas moins son recteur
qu'auparavant, il réfléchissait néanmoins qu'il se rencontrait avec une
question du catéchisme~: doiton l'obéissance à ses supérieurs,
lorsqu'ils désobéissent à la loi de Dieu~? & Pa bellaas ne oa ket aet
war vihanaat an istim en doa ar mab Boulc'h evit e berson met en desped
da se e soñje e oa o talañ ouzh ur goulenn katekiz: daoust ha rankout a
ra an unan sentiñ d'e superiored pa zisentont ouzh lezenn Doue? \\
Queffélec 1970, 218 & Beyer 2016, 198 \\
\end{longtable}

\paragraph{Penaos e vez treuzkaset mennozh an dlead e
galleg}\label{penaos_e_vez_treuzkaset_mennozh_an_dlead_e_galleg}

Er yezh seven, e-touez degouezhioù all, degas a ra ar verb \emph{devoir}
e galleg mennozh an dlead dreist-holl. Gwanaet eo bet implij ar verb
\emph{devoir} evit ezteurel ur redi, dreist-holl en amzer-vremañ hag en
amzer dremenet ledan : war a-seblant e implijer
aliesoc\textquotesingle h stummoù evel \emph{il faut}, \emph{il fallait}
er yezh voutin \href{Darbelnet_&_Vinay_1993}{Darbelnet \& Vinay 1993}
(138-139).

Pa implijer\emph{devoir} en amzer dremenet ledan e galleg ez eus un
amsterder etre mennozh an dle hag hini ar ratozh: \emph{vous deviez
rentrer hier} = \emph{raktreset e oa e tistrofec\textquotesingle h
dec\textquotesingle h}; \emph{dleet e oa deoc\textquotesingle h distreiñ
dec\textquotesingle h}.

\emph{Devoir} lakaet en amzer-vremañ a c\textquotesingle hall merkañ un
obererezh en amzer-da-zont : \emph{Je dois le voir demain}.

Eztaoler a ra mennozh an dle ar verb \emph{devoir} en amzer-da-zont hag
en amzerioù tremenet all estreget an amzer dremenet ledan.

E degouezhioù zo, eztaolet e vez mennozh ar brobablentez pa implijer
\emph{devoir} en amzerioù tremenet.

Ar gallus hag an dic\textquotesingle hallus a stou davet un dlead moral.

\paragraph{\texorpdfstring{Mennozh an ezhomm a c\textquotesingle haller
degas e brezhoneg gant an araogenn \emph{da} e-touez elfennoù all
evel-henn}{Mennozh an ezhomm a c\textquotesingle haller degas e brezhoneg gant an araogenn da e-touez elfennoù all evel-henn}}\label{mennozh_an_ezhomm_a_challer_degas_e_brezhoneg_gant_an_araogenn_da_e_touez_elfennouxf9_all_evel_henn}

\begin{longtable}[]{@{}ll@{}}
\toprule\noalign{}
\endhead
\bottomrule\noalign{}
\endlastfoot
Au-delà des cailloux et des sueurs, des peines que les mains prennent et
des fardeaux qu'il faut traîner. & En tu all d'ar mein ha d'an
dour-c'hwez, d'ar poanioù da sammañ a-zornadoù, d'ar bec'hioù da
stlejañ. \\
Queffélec 1970, 212 & Beyer 2016, 192 \\
\end{longtable}

\subsection{Dleout}\label{dleout}

\paragraph{\texorpdfstring{Un obererezh a zo ret hervez ar pezh a zere
(\href{https://niverel.brezhoneg.bzh/fr/meurgorf/?page=1&term=dleout&search\%20type=me}{Meurgorf})}{Un obererezh a zo ret hervez ar pezh a zere (Meurgorf)}}\label{un_obererezh_a_zo_ret_hervez_ar_pezh_a_zere_meurgorf}

\begin{longtable}[]{@{}ll@{}}
\toprule\noalign{}
\endhead
\bottomrule\noalign{}
\endlastfoot
Chaque îlien eût dû se tenir aux aguets. & Dleet e oa da bep enezour
chom war api. \\
Queffélec 1970, 122 & Beyer 2016, 100 \\
\end{longtable}

\begin{longtable}[]{@{}ll@{}}
\toprule\noalign{}
\endhead
\bottomrule\noalign{}
\endlastfoot
Tu dois bien exécuter cette besogne-là. & Dleet eo dit kas al labour-se
da benn. \\
Queffélec 1970, 240 & Beyer 2016, 220 \\
\end{longtable}

\paragraph{Disklêriañ kerse diwar-benn un dra a zlefe bezañ graet met
n\textquotesingle eo ket bet
graet}\label{diskluxeariauxf1_kerse_diwar_benn_un_dra_a_zlefe_bezauxf1_graet_met_neo_ket_bet_graet}

\begin{longtable}[]{@{}ll@{}}
\toprule\noalign{}
\endhead
\bottomrule\noalign{}
\endlastfoot
Elle aurait dû cent fois prévenir Thomas et solliciter son conseil ---
ne jouait-il pas ici le même rôle qu'un prêtre~? & Dleet e oa dezhi
kelaouiñ Tomaz ha goulenn e ali --- daoust ha ne oa ket amañ perzh ur
beleg gantañ~? \\
Queffélec 1970, 175 & Beyer 2016, 155 \\
\end{longtable}

\paragraph{Deskrivañ un obererezh kaset da benn en un doare dereat, evel
m\textquotesingle eo
gortozet}\label{deskrivauxf1_un_obererezh_kaset_da_benn_en_un_doare_dereat_evel_meo_gortozet}

\begin{longtable}[]{@{}ll@{}}
\toprule\noalign{}
\endhead
\bottomrule\noalign{}
\endlastfoot
Après leur passage les femmes sortaient par les portes de derrière et
couraient chez des amies dans l'espérance de précéder les messagères~;
on se manquait souvent ou bien, à l'improviste, on se trouvait nez à nez
l'une avec l'autre, et chacune déjà dûment avertie. & Goude ma oant bet
ez ae ar maouezed all er-maez dre an dorioù dreñv ha d'ar red e ti o
mignonezed, gant ar spi da erruout a-raok ar c'hannadezed; alies e veze
c'hwitet war ar re all pe neuze e veze emgav fri-ouzh-fri, ha pep hini
bet kelaouet evel m'eo dleet. \\
Queffélec 1970, 82 & Beyer 2016, 61 \\
\end{longtable}

\begin{longtable}[]{@{}ll@{}}
\toprule\noalign{}
\endhead
\bottomrule\noalign{}
\endlastfoot
Les parents, pour le principe, s'apprêtaient à veiller quelques heures.
& Evel m'eo dleet e priente ar gerent da veilhat un toulladig
eurvezhioù. \\
Queffélec 1970, 143 & Beyer 2016, 122 \\
\end{longtable}

\paragraph{Diskouez un dellid pe un dra endalc\textquotesingle het da
vezañ
roet}\label{diskouez_un_dellid_pe_un_dra_endalchet_da_vezauxf1_roet}

\begin{longtable}[]{@{}ll@{}}
\toprule\noalign{}
\endhead
\bottomrule\noalign{}
\endlastfoot
Un homme triste méritait quelques égards, il ne possède pas le lot qui
lui revient. & Dellezout a rae un den trist e vefe damantet outañ, un
disterig paneveken, rak e tiouer al lod a zo dleet dezhañ. \\
Queffélec 1970, 112 & Beyer 2016, 90 \\
\end{longtable}

\begin{longtable}[]{@{}ll@{}}
\toprule\noalign{}
\endhead
\bottomrule\noalign{}
\endlastfoot
Il sentait que Thomas méritait la même obéissance qu'un prêtre. &
Santout a rae e oa dleet sentidigezh da Domaz evel ma vez d'ur beleg. \\
Queffélec 1970, 191 & Beyer 2016, 172 \\
\end{longtable}

\paragraph{Ober ur vartezeadenn da-geñver un darvoud pe ur
fed}\label{ober_ur_vartezeadenn_da_geuxf1ver_un_darvoud_pe_ur_fed}

\begin{longtable}[]{@{}ll@{}}
\toprule\noalign{}
\endhead
\bottomrule\noalign{}
\endlastfoot
Il fallait que ce Guillerm possédât une longue lignée d'ancêtres bêtes
et méchants. & Bez' e tlee ar Gwilherm-se bezañ eus un hir a lignezad
hendadoù drouk ha sot. \\
Queffélec 1970, 230 & Beyer 2016, 210 \\
\end{longtable}

\section{Anvioù-gwan
damskoazellañ}\label{anviouxf9_gwan_damskoazellauxf1}

\subsection{Dav}\label{dav}

Un anv-gwan damskoazellañ eo \emph{dav} ; merkañ a ra un ezhomm posupl
pe un dlead moral
(\href{https://arbres.iker.cnrs.fr/index.php?title=Dav}{ARBRES : Dav})

\paragraph{\texorpdfstring{Implijet e vez evit treiñ ar stumm gallek
\emph{il
faut}}{Implijet e vez evit treiñ ar stumm gallek il faut}}\label{implijet_e_vez_evit_treiuxf1_ar_stumm_gallek_il_faut}

\begin{longtable}[]{@{}ll@{}}
\toprule\noalign{}
\endhead
\bottomrule\noalign{}
\endlastfoot
Ce n\textquotesingle étaient pas des prêtres qu\textquotesingle il
fallait leur envoyer, de bons prêtres de Quimper qui leur parleraient
breton, mais des missionnaires espagnols et des hommes
d\textquotesingle armes. & N\textquotesingle eo ket beleien e oa dav kas
dezho, beleien vat a Gemper hag a \textquotesingle z afe e brezhoneg
outo met misionerien spagnolat ha tud armet. \\
Queffélec 1970, 46 & Beyer 2016, 26 \\
\end{longtable}

\begin{longtable}[]{@{}ll@{}}
\toprule\noalign{}
\endhead
\bottomrule\noalign{}
\endlastfoot
Il fallait que le recteur écrivît à l'évêque une lettre comme quoi ça ne
pouvait plus durer comme quoi l'île de Sein, un village chrétien, avait
besoin d'un prêtre. & Dav e oa d'ar beleg skrivañ ul lizher d'an eskob
da lavarout ne c'helle ket an traoù padout evel-se, hag he doa ezhomm
enez Sun, kêriadenn gristen, da gaout ur beleg. \\
Queffélec 1970, 68 & Beyer 2016, 48 \\
\end{longtable}

\begin{longtable}[]{@{}ll@{}}
\toprule\noalign{}
\endhead
\bottomrule\noalign{}
\endlastfoot
Il faut tout recommencer. & Dav eo adober pep tra. \\
Queffélec 1970, 143 & Beyer 2016, 121 \\
\end{longtable}

\paragraph{Gallout a ra ezteurel ur c\textquotesingle hlozadur poellek
(logical conclusion) pe ur vartezeadenn greñv diazezet war fedoù
kinniget}\label{gallout_a_ra_ezteurel_ur_chlozadur_poellek_logical_conclusion_pe_ur_vartezeadenn_greuxf1v_diazezet_war_fedouxf9_kinniget}

\begin{longtable}[]{@{}ll@{}}
\toprule\noalign{}
\endhead
\bottomrule\noalign{}
\endlastfoot
Mais prendre leurs vêtements à des naufragés... Il fallait vraiment que
les hommes fussent bien méchants. & Met tapout o dilhad digant
peñseidi... Dav e vefe d'an dud bezañ drouk da vat. \\
Queffélec 1970, 105 & Beyer 2016, 83 \\
\end{longtable}

\section{Stummoù damskoazellañ
all}\label{stummouxf9_damskoazellauxf1_all}

\subsection{\texorpdfstring{\emph{Ret}}{Ret}}\label{ret}

Ret a zegas mennozh an dlead moral, ar redi lezennel pe an ezhomm
boutin. Pa vez liammet gant ar gopulenn, e teu da vezañ ur stumm
damskoazellañ
(\href{https://arbres.iker.cnrs.fr/index.php?title=Ret}{ARBRES : Ret}).

\begin{longtable}[]{@{}ll@{}}
\toprule\noalign{}
\endhead
\bottomrule\noalign{}
\endlastfoot
L'oiseau, après un délai obligatoire d'hésitations, des saccades timides
qui se changeaient en saccades orgueilleuses, mordait goulûment sur
l'appât et se perçait le bec. & Goude ur pennadig arvar ret ha frapadoù
abaf a droe e frapadoù lorc'hus e kroge lontek al labous er vouedenn hag
e toulle e veg. \\
Queffélec 1970, 125 & Beyer 2016, 103 \\
\end{longtable}
