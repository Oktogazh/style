\chapter{Verboù ha stummoù damskoazellañ e brezhoneg}
\subsection[Petra eo ar
vodelezh~?]{\texorpdfstring{\protect\hypertarget{Petra_eo_ar_vodelezh_.3F}{}{}Petra
eo ar
vodelezh~?}{Petra eo ar vodelezh~?}}\label{Petra_eo_ar_vodelezh_ux3f}

{{{[}}\href{/wiki/Verbo\%C3\%B9_ha_stummo\%C3\%B9_damskoazella\%C3\%B1_e_brezhoneg?action=edit&section=1}{{modifier}}{{]}}}

Termeniñ a ra Hacquard (\href{/wiki/Hacquard_2009}{2009}, 1) ar vodelezh
evel perzh ar ster a denn d'ar martezeadennoù hag
ezhommoù pe redioù, da lavaret eo stadoù an traoù en tu all
d'ar saviad a-vremañ. Daoust ma ne
c'hoarvez biken an darvoudoù hipotetek-se e
c'haller komz diwar o fenn en ur implijout gerioù
damskoazellañ.

Gallout a reer komz diwar-benn ar pezh a zo ret/arabat ober evit doujañ
ouzh al lezenn, ar pezh a fello d'an nen ober goude
bezañ echuet e studioù, ar pezh a rafed ma c'hounezfed
d'al loto pe ar pezh a c'hallje
c'hoarvezout ma ne vije ket Breizh aloubet gant
Bro-C'hall.

Hervez \href{/wiki/Jouitteau_2009-2024}{Jouitteau} e vez eztaolet gant
stummoù damskoazellañ mennozhioù evel ar posublded pe an ezhomm. Gallout
a ra merkañ ar verboù damskoazellañ un derez a brobablentez ivez. Er
ster-se e servijont evel merkerioù epistemek hag a ziskouez live fiziañs
ar c'homzer e-keñver e brezeg. An epistemegezh a
c'haller seveniñ er yezh gant meur a elfenn all ouzhpenn
ar verboù damskoazellañ, en o zouez adverboù evel
'marteze\textquotesingle. E-touez elfennoù damskoazellañ
e brezhoneg e kaver ar verboù (rankout, dleout, faotañ, mennout,
fellout, plijout, gallout), predikadoù damskoazellañ a rummadoù
disheñvel (anvioù-gwan evel 'dav\textquotesingle{} pe
'ral\textquotesingle, anvioù-gwan-verb evel
'darbet\textquotesingle, pe an anv damskoazellañ
'ret\textquotesingle) ha stummoù damskoazellañ evel
'na vezañ evit\textquotesingle{}
(\href{https://arbres.iker.cnrs.fr/index.php?title=Constructions_modales}{ARBRES~:
Constructions modales}).

\subsection[Verboù
damskoazellañ]{\texorpdfstring{\protect\hypertarget{Verbo.C3.B9_damskoazella.C3.B1}{}{}Verboù
damskoazellañ}{Verboù damskoazellañ}}\label{Verbouxf9_damskoazellauxf1}

{{{[}}\href{/wiki/Verbo\%C3\%B9_ha_stummo\%C3\%B9_damskoazella\%C3\%B1_e_brezhoneg?action=edit&section=2}{{modifier}}{{]}}}

\subsubsection{Gallout}\label{Gallout}

{{{[}}\href{/wiki/Verbo\%C3\%B9_ha_stummo\%C3\%B9_damskoazella\%C3\%B1_e_brezhoneg?action=edit&section=3}{{modifier}}{{]}}}

Liesster hag amster eo ar verb \emph{gallout} e brezhoneg, evel e meur a
yezh all. Gallout a reer kompren ar frazenn \emph{Il peut venir} e daou
ster disheñvel: \emph{Posubl eo dezhañ dont} pe \emph{Marteze e teuio}.

E degouezhioù liesseurt e vez implijet ar verb-se~:

\paragraph{Evit diskouez ur varregezh
fizikel}\label{Evit_diskouez_ur_varregezh_fizikel}

{{{[}}\href{/wiki/Verbo\%C3\%B9_ha_stummo\%C3\%B9_damskoazella\%C3\%B1_e_brezhoneg?action=edit&section=4}{{modifier}}{{]}}}

Personne n'avait pu rien voir ni rien entendre....

Queffélec 1970, 208

Gant den ebet ne oa bet gellet klevout na gwelout tra...

Beyer 2016, 188

\paragraph{Evit treuzkas un ezhomm pe ur
redi}\label{Evit_treuzkas_un_ezhomm_pe_ur_redi}

{{{[}}\href{/wiki/Verbo\%C3\%B9_ha_stummo\%C3\%B9_damskoazella\%C3\%B1_e_brezhoneg?action=edit&section=5}{{modifier}}{{]}}}

Le recteur ne tendit pas la main et le pêcheur n'eut d'autre ressource
que de les poser sur une table.

Queffélec 1970, 69

Ar person ne astennas ket e zorn ha ne c'hellas ar pesketaer nemet o
lakaat war an daol.

Beyer 2016, 50

Les jours où l'état de la mer interdisait aux barques de sortir, les
hommes attrapaient les oiseaux.

Queffélec 1970, 125

En deizioù ma ne c'helle ket ar wazed mont war vor abalamour d'e stad e
tapent laboused.

Beyer 2016, 103

\paragraph{Evit eztaoler asurded pe
bosublentez}\label{Evit_eztaoler_asurded_pe_bosublentez}

{{{[}}\href{/wiki/Verbo\%C3\%B9_ha_stummo\%C3\%B9_damskoazella\%C3\%B1_e_brezhoneg?action=edit&section=6}{{modifier}}{{]}}}

Certainement Dieu n'avait pu se tromper et le sens de cette histoire se
cachait quelque part, mais non dans les commentaires.

Queffélec 1970, 177

A dra sur n'en doa ket gellet Doue faziañ ha kuzhet e oa ster an
istor-se tu bennak met ne oa ket en addisplegoù.

Beyer 2016, 157

Il n'est pas possible~que des aubes glorieuses, se
déployant dans le fond du ciel, aient éclairé ce morceau de récif.

Queffélec 1970, 5

Ne c'hell ket bezañ bet~ruzelloù-mintin o sklêrijennañ
an tamm penn-karreg-se en o dispak glorius e don an oabl.

Beyer 2016, 7

«~C'est possible, répondaient les femmes, mais le bon Dieu nous punit
trop.~»

Queffélec 1970, 40

«~E c'hell bezañ,~» a eilgerie ar maouezed «~met re e vezomp kastizet
gant an aotrou Doue~».

Beyer 2016, 20

\paragraph[Eztaoler ur merzhadur
(perception)]{\texorpdfstring{\protect\hypertarget{Eztaoler_ur_merzhadur_.28perception.29}{}{}Eztaoler
ur merzhadur
(perception)}{Eztaoler ur merzhadur (perception)}}\label{Eztaoler_ur_merzhadur_ux28perceptionux29}

{{{[}}\href{/wiki/Verbo\%C3\%B9_ha_stummo\%C3\%B9_damskoazella\%C3\%B1_e_brezhoneg?action=edit&section=7}{{modifier}}{{]}}}

Avec cela, parfois, les journées étaient claires, un ciel vif heurtait
les fenêtres, quelques moments encore et il semblait qu'on apercevrait
le vent, qui ployait l'air et frappait la mer selon la brusquerie d'une
houssine.

Queffélec 1970, 59

Dont a rae an oabl lemm da skeiñ ouzh ar prenestri, ul lajadig c'hoazh
hag e c'helled soñjal damwelout an avel o plegañ an aer, o vazhata ar
mor gant taerder ur gelennenn.

Beyer 2016, 39

\paragraph[Implijet e vez ar stumm dic'hour a verb
gallout e brezhoneg evit treiñ verboù raganv
gallek]{\texorpdfstring{\protect\hypertarget{Implijet_e_vez_ar_stumm_dic.27hour_a_verb_gallout_e_brezhoneg_evit_trei.C3.B1_verbo.C3.B9_raganv_gallek}{}{}Implijet
e vez ar stumm dic'hour a verb gallout e brezhoneg evit
treiñ verboù raganv
gallek}{Implijet e vez ar stumm dic'hour a verb gallout e brezhoneg evit treiñ verboù raganv gallek}}\label{Implijet_e_vez_ar_stumm_dicux27hour_a_verb_gallout_e_brezhoneg_evit_treiuxf1_verbouxf9_raganv_gallek}

{{{[}}\href{/wiki/Verbo\%C3\%B9_ha_stummo\%C3\%B9_damskoazella\%C3\%B1_e_brezhoneg?action=edit&section=8}{{modifier}}{{]}}}

Ils étaient ici comme dans un navire, et un navire n'a point de prêtre,
mais un jour le navire aborde en pays chrétien et, du pont, des clochers
se contemplent sur le rivage.

Queffélec 1970, 61

Evel en ul lestr e oant hag e-bourzh ul lestr n'eus ket a veleg, met un
deiz e teu al lestr da zouarañ en ur vro gristen ha diwar ar pont e
c'heller arvestiñ ouzh ar kloc'hdioù war an aod.

Beyer 2016, 41

Le sable, d'un blanc faible et un peu jaunâtre, se devinait quelque
temps et se perdait contre les souples murailles de l'ombre qui se
dressaient dans la nuit vague.

Queffélec 1970, 64

Ur predig e c'helled damwelout an traezh, gwenn peñver hag arvelen un
disterañ, ha goude ez ae d'en em goll ouzh mogerennoù gwevn an
deñvalijenn a save en noz dispis.

Beyer 2016, 45

\subsubsection{Rankout}\label{Rankout}

{{{[}}\href{/wiki/Verbo\%C3\%B9_ha_stummo\%C3\%B9_damskoazella\%C3\%B1_e_brezhoneg?action=edit&section=9}{{modifier}}{{]}}}

\paragraph{Mennozh ar rank}\label{Mennozh_ar_rank}

{{{[}}\href{/wiki/Verbo\%C3\%B9_ha_stummo\%C3\%B9_damskoazella\%C3\%B1_e_brezhoneg?action=edit&section=10}{{modifier}}{{]}}}

Noblesse oblige.

Queffélec 1970, 127

Diouzh e renk e rank an unan ober.

Beyer 2016, 104

\paragraph{An ezhomm pe ar redi}\label{An_ezhomm_pe_ar_redi}

{{{[}}\href{/wiki/Verbo\%C3\%B9_ha_stummo\%C3\%B9_damskoazella\%C3\%B1_e_brezhoneg?action=edit&section=11}{{modifier}}{{]}}}

L'île, comme une place forte, lance une sortie à chaque marée basse~;
elle récupère ses douves, elle pille le camp ennemi~; vient le flux, et
l'île se retranche dans son enceinte, dont les assiégeants battent les
murs~: ils ont perdu la plus grande partie de leurs troupes et de leur
énergie sur toutes les pentes, barrées d'écueils, qu'il a fallu
franchir.

Queffélec 1970, 137

Par d'ur c'hreñvlec'h e ra an enez un disailhadenn da vare pep izelvor;
azpiaouañ a ra he douvezioù ha preizhañ kamp an enebour; dont a ra ar
chal hag an enez neuze d'en em gaeañ e gwarez he gourizad tra ma vez ar
sezizerien o tagañ he mogerioù: kollet o deus ar braz eus o bagadoù hag
o holl startijenn war an holl savioù sparlet gant pennoù-kerreg o deus
ranket treuziñ.

Beyer 2016, 115

\paragraph{Ur raktres pe un obererezh gortozet digant unan
bennak}\label{Ur_raktres_pe_un_obererezh_gortozet_digant_unan_bennak}

{{{[}}\href{/wiki/Verbo\%C3\%B9_ha_stummo\%C3\%B9_damskoazella\%C3\%B1_e_brezhoneg?action=edit&section=12}{{modifier}}{{]}}}

Elle avait dû se retirer avant que la barque fût halée sur la dune,
mais, ce matin, toute seule, elle s'était rendue dans la famille du
pêcheur y prendre de ses nouvelles et le remercier.

Queffélec 1970, 151

Ranket he doa mont kuit a-raok na vefe bet sachet ar bark war an tevenn
met diouzh ar mintin e oa bet, en hec'h-unanpenn, e ti ar pesketaer
yaouank da gaout keloù ha d'e drugarekaat.

Beyer 2016, 130-131

Thomas les rencontrait à chaque séjour sur le continent et, s'il devait
coucher à Audierne, il les réunissait le soir chez le sabotier.

Queffélec 1970, 176

Da bep chomadenn war an douar bras en em gave Tomaz ganto ha pa ranke
kousket e Gwaien e vode anezho da noz e ti ar botaouer-koad.

Beyer 2016, 156

\paragraph{Ur gefridi hiniennel pe un dlead
moral}\label{Ur_gefridi_hiniennel_pe_un_dlead_moral}

{{{[}}\href{/wiki/Verbo\%C3\%B9_ha_stummo\%C3\%B9_damskoazella\%C3\%B1_e_brezhoneg?action=edit&section=13}{{modifier}}{{]}}}

Jules dut promettre de se rendre à Saint-Tugen en pèlerinage quand il
serait guéri.

Queffélec 1970,

Rankout a rafe Jul mont da birc'hirinaj da Sant-Tujen pa vefe pare.

Beyer 2016, 140

Quand il s'éloigna, le fils Boulch n'estimait pas moins son recteur
qu'auparavant, il réfléchissait néanmoins qu'il se rencontrait avec une
question du catéchisme~: doiton l'obéissance à ses supérieurs,
lorsqu'ils désobéissent à la loi de Dieu~?

Queffélec 1970, 218

Pa bellaas ne oa ket aet war vihanaat an istim en doa ar mab Boulc'h
evit e berson met en desped da se e soñje e oa o talañ ouzh ur goulenn
katekiz: daoust ha rankout a ra an unan sentiñ d'e superiored pa
zisentont ouzh lezenn Doue?

Beyer 2016, 198

\paragraph{Penaos e vez treuzkaset mennozh an dlead e
galleg}\label{Penaos_e_vez_treuzkaset_mennozh_an_dlead_e_galleg}

{{{[}}\href{/wiki/Verbo\%C3\%B9_ha_stummo\%C3\%B9_damskoazella\%C3\%B1_e_brezhoneg?action=edit&section=14}{{modifier}}{{]}}}

Er yezh seven, e-touez degouezhioù all, degas a ra ar verb \emph{devoir}
e galleg mennozh an dlead dreist-holl. Gwanaet eo bet implij ar verb
\emph{devoir} evit ezteurel ur redi, dreist-holl en amzer-vremañ hag en
amzer dremenet ledan~: war a-seblant e implijer
aliesoc'h stummoù evel \emph{il faut}, \emph{il fallait}
er yezh voutin \href{/wiki/Darbelnet_\%26_Vinay_1993}{Darbelnet \& Vinay
1993} (138-139).

Pa implijer\emph{devoir} en amzer dremenet ledan e galleg ez eus un
amsterder etre mennozh an dle hag hini ar ratozh: \emph{vous deviez
rentrer hier} = \emph{raktreset e oa e tistrofec'h
dec'h}; \emph{dleet e oa deoc'h distreiñ
dec'h} .

\emph{Devoir} lakaet en amzer-vremañ a c'hall merkañ un
obererezh en amzer-da-zont~: \emph{Je dois le voir demain}.

Eztaoler a ra mennozh an dle ar verb \emph{devoir} en amzer-da-zont hag
en amzerioù tremenet all estreget an amzer dremenet ledan.

E degouezhioù zo, eztaolet e vez mennozh ar brobablentez pa implijer
\emph{devoir} en amzerioù tremenet.

Ar gallus hag an dic'hallus a stou davet un dlead moral.

\paragraph[Mennozh an ezhomm a c'haller degas e
brezhoneg gant an araogenn \emph{da} e-touez elfennoù all
evel-henn]{\texorpdfstring{\protect\hypertarget{Mennozh_an_ezhomm_a_c.27haller_degas_e_brezhoneg_gant_an_araogenn_da_e-touez_elfenno.C3.B9_all_evel-henn}{}{}Mennozh
an ezhomm a c'haller degas e brezhoneg gant an araogenn
\emph{da} e-touez elfennoù all
evel-henn}{Mennozh an ezhomm a c'haller degas e brezhoneg gant an araogenn da e-touez elfennoù all evel-henn}}\label{Mennozh_an_ezhomm_a_cux27haller_degas_e_brezhoneg_gant_an_araogenn_da_e-touez_elfennouxf9_all_evel-henn}

{{{[}}\href{/wiki/Verbo\%C3\%B9_ha_stummo\%C3\%B9_damskoazella\%C3\%B1_e_brezhoneg?action=edit&section=15}{{modifier}}{{]}}}

Au-delà des cailloux et des sueurs, des peines que les mains prennent et
des fardeaux qu'il faut traîner.

Queffélec 1970, 212

En tu all d'ar mein ha d'an dour-c'hwez, d'ar poanioù da sammañ
a-zornadoù, d'ar bec'hioù da stlejañ.

Beyer 2016, 192

\subsubsection{Dleout}\label{Dleout}

{{{[}}\href{/wiki/Verbo\%C3\%B9_ha_stummo\%C3\%B9_damskoazella\%C3\%B1_e_brezhoneg?action=edit&section=16}{{modifier}}{{]}}}

\paragraph[Un obererezh a zo ret hervez ar pezh a zere
(\href{https://niverel.brezhoneg.bzh/fr/meurgorf/?page=1&term=dleout&search\%20type=me}{Meurgorf})]{\texorpdfstring{\protect\hypertarget{Un_obererezh_a_zo_ret_hervez_ar_pezh_a_zere_.28Meurgorf.29}{}{}Un
obererezh a zo ret hervez ar pezh a zere
(\href{https://niverel.brezhoneg.bzh/fr/meurgorf/?page=1&term=dleout&search\%20type=me}{Meurgorf})}{Un obererezh a zo ret hervez ar pezh a zere (Meurgorf)}}\label{Un_obererezh_a_zo_ret_hervez_ar_pezh_a_zere_ux28Meurgorfux29}

{{{[}}\href{/wiki/Verbo\%C3\%B9_ha_stummo\%C3\%B9_damskoazella\%C3\%B1_e_brezhoneg?action=edit&section=17}{{modifier}}{{]}}}

Chaque îlien eût dû se tenir aux aguets.

Queffélec 1970, 122

Dleet e oa da bep enezour chom war api.

Beyer 2016, 100

Tu dois bien exécuter cette besogne-là.

Queffélec 1970, 240

Dleet eo dit kas al labour-se da benn.

Beyer 2016, 220

\paragraph[Disklêriañ kerse diwar-benn un dra a zlefe bezañ graet met
n'eo ket bet
graet]{\texorpdfstring{\protect\hypertarget{Diskl.C3.AAria.C3.B1_kerse_diwar-benn_un_dra_a_zlefe_beza.C3.B1_graet_met_n.27eo_ket_bet_graet}{}{}Disklêriañ
kerse diwar-benn un dra a zlefe bezañ graet met n'eo ket
bet
graet}{Disklêriañ kerse diwar-benn un dra a zlefe bezañ graet met n'eo ket bet graet}}\label{Diskluxeariauxf1_kerse_diwar-benn_un_dra_a_zlefe_bezauxf1_graet_met_nux27eo_ket_bet_graet}

{{{[}}\href{/wiki/Verbo\%C3\%B9_ha_stummo\%C3\%B9_damskoazella\%C3\%B1_e_brezhoneg?action=edit&section=18}{{modifier}}{{]}}}

Elle aurait dû cent fois prévenir Thomas et solliciter son conseil ---
ne jouait-il pas ici le même rôle qu'un prêtre~?

Queffélec 1970, 175

Dleet e oa dezhi kelaouiñ Tomaz ha goulenn e ali --- daoust ha ne oa ket
amañ perzh ur beleg gantañ~?

Beyer 2016, 155

\paragraph[Deskrivañ un obererezh kaset da benn en un doare dereat, evel
m'eo
gortozet]{\texorpdfstring{\protect\hypertarget{Deskriva.C3.B1_un_obererezh_kaset_da_benn_en_un_doare_dereat.2C_evel_m.27eo_gortozet}{}{}Deskrivañ
un obererezh kaset da benn en un doare dereat, evel m'eo
gortozet}{Deskrivañ un obererezh kaset da benn en un doare dereat, evel m'eo gortozet}}\label{Deskrivauxf1_un_obererezh_kaset_da_benn_en_un_doare_dereatux2c_evel_mux27eo_gortozet}

{{{[}}\href{/wiki/Verbo\%C3\%B9_ha_stummo\%C3\%B9_damskoazella\%C3\%B1_e_brezhoneg?action=edit&section=19}{{modifier}}{{]}}}

Après leur passage les femmes sortaient par les portes de derrière et
couraient chez des amies dans l'espérance de précéder les messagères~;
on se manquait souvent ou bien, à l'improviste, on se trouvait nez à nez
l'une avec l'autre, et chacune déjà dûment avertie.

Queffélec 1970, 82

Goude ma oant bet ez ae ar maouezed all er-maez dre an dorioù dreñv ha
d'ar red e ti o mignonezed, gant ar spi da erruout a-raok ar
c'hannadezed; alies e veze c'hwitet war ar re all pe neuze e veze emgav
fri-ouzh-fri, ha pep hini bet kelaouet evel m'eo dleet.

Beyer 2016, 61

Les parents, pour le principe, s'apprêtaient à veiller quelques heures.

Queffélec 1970, 143

Evel m'eo dleet e priente ar gerent da veilhat un toulladig eurvezhioù.

Beyer 2016, 122

\paragraph[Diskouez un dellid pe un dra endalc'het da
vezañ
roet]{\texorpdfstring{\protect\hypertarget{Diskouez_un_dellid_pe_un_dra_endalc.27het_da_veza.C3.B1_roet}{}{}Diskouez
un dellid pe un dra endalc'het da vezañ
roet}{Diskouez un dellid pe un dra endalc'het da vezañ roet}}\label{Diskouez_un_dellid_pe_un_dra_endalcux27het_da_vezauxf1_roet}

{{{[}}\href{/wiki/Verbo\%C3\%B9_ha_stummo\%C3\%B9_damskoazella\%C3\%B1_e_brezhoneg?action=edit&section=20}{{modifier}}{{]}}}

Un homme triste méritait quelques égards, il ne possède pas le lot qui
lui revient.

Queffélec 1970, 112

Dellezout a rae un den trist e vefe damantet outañ, un disterig
paneveken, rak e tiouer al lod a zo dleet dezhañ.

Beyer 2016, 90

Il sentait que Thomas méritait la même obéissance qu'un prêtre.

Queffélec 1970, 191

Santout a rae e oa dleet sentidigezh da Domaz evel ma vez d'ur beleg.

Beyer 2016, 172

\paragraph[Ober ur vartezeadenn da-geñver un darvoud pe ur
fed]{\texorpdfstring{\protect\hypertarget{Ober_ur_vartezeadenn_da-ge.C3.B1ver_un_darvoud_pe_ur_fed}{}{}Ober
ur vartezeadenn da-geñver un darvoud pe ur
fed}{Ober ur vartezeadenn da-geñver un darvoud pe ur fed}}\label{Ober_ur_vartezeadenn_da-geuxf1ver_un_darvoud_pe_ur_fed}

{{{[}}\href{/wiki/Verbo\%C3\%B9_ha_stummo\%C3\%B9_damskoazella\%C3\%B1_e_brezhoneg?action=edit&section=21}{{modifier}}{{]}}}

Il fallait que ce Guillerm possédât une longue lignée d'ancêtres bêtes
et méchants.

Queffélec 1970, 230

Bez' e tlee ar Gwilherm-se bezañ eus un hir a lignezad hendadoù drouk ha
sot.

Beyer 2016, 210

\subsection[Anvioù-gwan
damskoazellañ]{\texorpdfstring{\protect\hypertarget{Anvio.C3.B9-gwan_damskoazella.C3.B1}{}{}Anvioù-gwan
damskoazellañ}{Anvioù-gwan damskoazellañ}}\label{Anviouxf9-gwan_damskoazellauxf1}

{{{[}}\href{/wiki/Verbo\%C3\%B9_ha_stummo\%C3\%B9_damskoazella\%C3\%B1_e_brezhoneg?action=edit&section=22}{{modifier}}{{]}}}

\subsubsection{Dav}\label{Dav}

{{{[}}\href{/wiki/Verbo\%C3\%B9_ha_stummo\%C3\%B9_damskoazella\%C3\%B1_e_brezhoneg?action=edit&section=23}{{modifier}}{{]}}}

Un anv-gwan damskoazellañ eo \emph{dav}~; merkañ a ra un ezhomm posupl
pe un dlead moral
(\href{https://arbres.iker.cnrs.fr/index.php?title=Dav}{ARBRES~: Dav})

\paragraph[Implijet e vez evit treiñ ar stumm gallek \emph{il
faut}]{\texorpdfstring{\protect\hypertarget{Implijet_e_vez_evit_trei.C3.B1_ar_stumm_gallek_il_faut}{}{}Implijet
e vez evit treiñ ar stumm gallek \emph{il
faut}}{Implijet e vez evit treiñ ar stumm gallek il faut}}\label{Implijet_e_vez_evit_treiuxf1_ar_stumm_gallek_il_faut}

{{{[}}\href{/wiki/Verbo\%C3\%B9_ha_stummo\%C3\%B9_damskoazella\%C3\%B1_e_brezhoneg?action=edit&section=24}{{modifier}}{{]}}}

Ce n'étaient pas des prêtres qu'il
fallait leur envoyer, de bons prêtres de Quimper qui leur parleraient
breton, mais des missionnaires espagnols et des hommes
d'armes.

Queffélec 1970, 46

N'eo ket beleien e oa dav kas dezho, beleien vat a
Gemper hag a 'z afe e brezhoneg outo met misionerien
spagnolat ha tud armet.

Beyer 2016, 26

Il fallait que le recteur écrivît à l'évêque une lettre comme quoi ça ne
pouvait plus durer comme quoi l'île de Sein, un village chrétien, avait
besoin d'un prêtre.

Queffélec 1970, 68

Dav e oa d'ar beleg skrivañ ul lizher d'an eskob da lavarout ne c'helle
ket an traoù padout evel-se, hag he doa ezhomm enez Sun, kêriadenn
gristen, da gaout ur beleg.

Beyer 2016, 48

Il faut tout recommencer.

Queffélec 1970, 143

Dav eo adober pep tra.

Beyer 2016, 121

\paragraph[Gallout a ra ezteurel ur c'hlozadur poellek
(logical conclusion) pe ur vartezeadenn greñv diazezet war fedoù
kinniget]{\texorpdfstring{\protect\hypertarget{Gallout_a_ra_ezteurel_ur_c.27hlozadur_poellek_.28logical_conclusion.29_pe_ur_vartezeadenn_gre.C3.B1v_diazezet_war_fedo.C3.B9_kinniget}{}{}Gallout
a ra ezteurel ur c'hlozadur poellek (logical conclusion)
pe ur vartezeadenn greñv diazezet war fedoù
kinniget}{Gallout a ra ezteurel ur c'hlozadur poellek (logical conclusion) pe ur vartezeadenn greñv diazezet war fedoù kinniget}}\label{Gallout_a_ra_ezteurel_ur_cux27hlozadur_poellek_ux28logical_conclusionux29_pe_ur_vartezeadenn_greuxf1v_diazezet_war_fedouxf9_kinniget}

{{{[}}\href{/wiki/Verbo\%C3\%B9_ha_stummo\%C3\%B9_damskoazella\%C3\%B1_e_brezhoneg?action=edit&section=25}{{modifier}}{{]}}}

Mais prendre leurs vêtements à des naufragés... Il fallait vraiment que
les hommes fussent bien méchants.

Queffélec 1970, 105

Met tapout o dilhad digant peñseidi... Dav e vefe d'an dud bezañ drouk
da vat.

Beyer 2016, 83

\subsection[Stummoù damskoazellañ
all]{\texorpdfstring{\protect\hypertarget{Stummo.C3.B9_damskoazella.C3.B1_all}{}{}Stummoù
damskoazellañ
all}{Stummoù damskoazellañ all}}\label{Stummouxf9_damskoazellauxf1_all}

{{{[}}\href{/wiki/Verbo\%C3\%B9_ha_stummo\%C3\%B9_damskoazella\%C3\%B1_e_brezhoneg?action=edit&section=26}{{modifier}}{{]}}}

\subsubsection{\texorpdfstring{\emph{Ret}}{Ret}}\label{Ret}

{{{[}}\href{/wiki/Verbo\%C3\%B9_ha_stummo\%C3\%B9_damskoazella\%C3\%B1_e_brezhoneg?action=edit&section=27}{{modifier}}{{]}}}

Ret a zegas mennozh an dlead moral, ar redi lezennel pe an ezhomm
boutin. Pa vez liammet gant ar gopulenn, e teu da vezañ ur stumm
damskoazellañ
(\href{https://arbres.iker.cnrs.fr/index.php?title=Ret}{ARBRES~: Ret}).

L'oiseau, après un délai obligatoire d'hésitations, des saccades timides
qui se changeaient en saccades orgueilleuses, mordait goulûment sur
l'appât et se perçait le bec.

Queffélec 1970, 125

Goude ur pennadig arvar ret ha frapadoù abaf a droe e frapadoù lorc'hus
e kroge lontek al labous er vouedenn hag e toulle e veg.

Beyer 2016, 103
