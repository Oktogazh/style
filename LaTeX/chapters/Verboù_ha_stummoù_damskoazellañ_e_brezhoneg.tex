\chapter{Verboù ha stummoù damskoazellañ e brezhoneg}

\section{Petra eo ar vodelezh ?}\label{petra_eo_ar_vodelezh}

Termeniñ a ra Hacquard (\href{Hacquard_2009}{2009}, 1) ar vodelezh evel
perzh ar ster a denn d\textquotesingle ar martezeadennoù hag ezhommoù pe
redioù, da lavaret eo stadoù an traoù en tu all d\textquotesingle ar
saviad a-vremañ. Daoust ma ne c\textquotesingle hoarvez biken an
darvoudoù hipotetek-se e c\textquotesingle haller komz diwar o fenn en
ur implijout gerioù damskoazellañ.

Gallout a reer komz diwar-benn ar pezh a zo ret/arabat ober evit doujañ
ouzh al lezenn, ar pezh a fello d\textquotesingle an nen ober goude
bezañ echuet e studioù, ar pezh a rafed ma c\textquotesingle hounezfed
d\textquotesingle al loto pe ar pezh a c\textquotesingle hallje
c\textquotesingle hoarvezout ma ne vije ket Breizh aloubet gant
Bro-C\textquotesingle hall.

Hervez \href{Jouitteau_2009-2024}{Jouitteau} e vez eztaolet gant stummoù
damskoazellañ mennozhioù evel ar posublded pe an ezhomm. Gallout a ra
merkañ ar verboù damskoazellañ un derez a brobablentez ivez. Er ster-se
e servijont evel merkerioù epistemek hag a ziskouez live fiziañs ar
c\textquotesingle homzer e-keñver e brezeg. An epistemegezh a
c\textquotesingle haller seveniñ er yezh gant meur a elfenn all ouzhpenn
ar verboù damskoazellañ, en o zouez adverboù evel
\textquotesingle marteze\textquotesingle. E-touez elfennoù damskoazellañ
e brezhoneg e kaver ar verboù (rankout, dleout, faotañ, mennout,
fellout, plijout, gallout), predikadoù damskoazellañ a rummadoù
disheñvel (anvioù-gwan evel \textquotesingle dav\textquotesingle{} pe
\textquotesingle ral\textquotesingle, anvioù-gwan-verb evel
\textquotesingle darbet\textquotesingle, pe an anv damskoazellañ
\textquotesingle ret\textquotesingle) ha stummoù damskoazellañ evel
\textquotesingle na vezañ evit\textquotesingle{}
(\href{https://arbres.iker.cnrs.fr/index.php?title=Constructions_modales}{ARBRES
: Constructions modales}).

\section{Verboù damskoazellañ}\label{verbouxf9_damskoazellauxf1}

\subsection{Gallout}\label{gallout}

Liesster hag amster eo ar verb \emph{gallout} e brezhoneg, evel e meur a
yezh all. Gallout a reer kompren ar frazenn \emph{Il peut venir} e daou
ster disheñvel: \emph{Posubl eo dezhañ dont} pe \emph{Marteze e teuio}.

E degouezhioù liesseurt e vez implijet ar verb-se :

\paragraph{Evit diskouez ur varregezh
fizikel}\label{evit_diskouez_ur_varregezh_fizikel}

\paragraph{Evit treuzkas un ezhomm pe ur
redi}\label{evit_treuzkas_un_ezhomm_pe_ur_redi}

\paragraph{Evit eztaoler asurded pe
bosublentez}\label{evit_eztaoler_asurded_pe_bosublentez}

\paragraph{Eztaoler ur merzhadur
(perception)}\label{eztaoler_ur_merzhadur_perception}

\paragraph{Implijet e vez ar stumm dic\textquotesingle hour a verb
gallout e brezhoneg evit treiñ verboù raganv
gallek}\label{implijet_e_vez_ar_stumm_dichour_a_verb_gallout_e_brezhoneg_evit_treiuxf1_verbouxf9_raganv_gallek}

\subsection{Rankout}\label{rankout}

\paragraph{Mennozh ar rank}\label{mennozh_ar_rank}

\paragraph{An ezhomm pe ar redi}\label{an_ezhomm_pe_ar_redi}

\paragraph{Ur raktres pe un obererezh gortozet digant unan
bennak}\label{ur_raktres_pe_un_obererezh_gortozet_digant_unan_bennak}

\paragraph{Ur gefridi hiniennel pe un dlead
moral}\label{ur_gefridi_hiniennel_pe_un_dlead_moral}

\paragraph{Penaos e vez treuzkaset mennozh an dlead e
galleg}\label{penaos_e_vez_treuzkaset_mennozh_an_dlead_e_galleg}

Er yezh seven, e-touez degouezhioù all, degas a ra ar verb \emph{devoir}
e galleg mennozh an dlead dreist-holl. Gwanaet eo bet implij ar verb
\emph{devoir} evit ezteurel ur redi, dreist-holl en amzer-vremañ hag en
amzer dremenet ledan : war a-seblant e implijer
aliesoc\textquotesingle h stummoù evel \emph{il faut}, \emph{il fallait}
er yezh voutin \href{Darbelnet_&_Vinay_1993}{Darbelnet \& Vinay 1993}
(138-139).

Pa implijer\emph{devoir} en amzer dremenet ledan e galleg ez eus un
amsterder etre mennozh an dle hag hini ar ratozh: \emph{vous deviez
rentrer hier} = \emph{raktreset e oa e tistrofec\textquotesingle h
dec\textquotesingle h}; \emph{dleet e oa deoc\textquotesingle h distreiñ
dec\textquotesingle h}.

\emph{Devoir} lakaet en amzer-vremañ a c\textquotesingle hall merkañ un
obererezh en amzer-da-zont : \emph{Je dois le voir demain}.

Eztaoler a ra mennozh an dle ar verb \emph{devoir} en amzer-da-zont hag
en amzerioù tremenet all estreget an amzer dremenet ledan.

E degouezhioù zo, eztaolet e vez mennozh ar brobablentez pa implijer
\emph{devoir} en amzerioù tremenet.

Ar gallus hag an dic\textquotesingle hallus a stou davet un dlead moral.

\paragraph{\texorpdfstring{Mennozh an ezhomm a c\textquotesingle haller
degas e brezhoneg gant an araogenn \emph{da} e-touez elfennoù all
evel-henn}{Mennozh an ezhomm a c\textquotesingle haller degas e brezhoneg gant an araogenn da e-touez elfennoù all evel-henn}}\label{mennozh_an_ezhomm_a_challer_degas_e_brezhoneg_gant_an_araogenn_da_e_touez_elfennouxf9_all_evel_henn}

\subsection{Dleout}\label{dleout}

\paragraph{\texorpdfstring{Un obererezh a zo ret hervez ar pezh a zere
(\href{https://niverel.brezhoneg.bzh/fr/meurgorf/?page=1&term=dleout&search\%20type=me}{Meurgorf})}{Un obererezh a zo ret hervez ar pezh a zere (Meurgorf)}}\label{un_obererezh_a_zo_ret_hervez_ar_pezh_a_zere_meurgorf}

\paragraph{Disklêriañ kerse diwar-benn un dra a zlefe bezañ graet met
n\textquotesingle eo ket bet
graet}\label{diskluxeariauxf1_kerse_diwar_benn_un_dra_a_zlefe_bezauxf1_graet_met_neo_ket_bet_graet}

\paragraph{Deskrivañ un obererezh kaset da benn en un doare dereat, evel
m\textquotesingle eo
gortozet}\label{deskrivauxf1_un_obererezh_kaset_da_benn_en_un_doare_dereat_evel_meo_gortozet}

\paragraph{Diskouez un dellid pe un dra endalc\textquotesingle het da
vezañ
roet}\label{diskouez_un_dellid_pe_un_dra_endalchet_da_vezauxf1_roet}

\paragraph{Ober ur vartezeadenn da-geñver un darvoud pe ur
fed}\label{ober_ur_vartezeadenn_da_geuxf1ver_un_darvoud_pe_ur_fed}

\section{Anvioù-gwan
damskoazellañ}\label{anviouxf9_gwan_damskoazellauxf1}

\subsection{Dav}\label{dav}

Un anv-gwan damskoazellañ eo \emph{dav} ; merkañ a ra un ezhomm posupl
pe un dlead moral
(\href{https://arbres.iker.cnrs.fr/index.php?title=Dav}{ARBRES : Dav})

\paragraph{\texorpdfstring{Implijet e vez evit treiñ ar stumm gallek
\emph{il
faut}}{Implijet e vez evit treiñ ar stumm gallek il faut}}\label{implijet_e_vez_evit_treiuxf1_ar_stumm_gallek_il_faut}

\paragraph{Gallout a ra ezteurel ur c\textquotesingle hlozadur poellek
(logical conclusion) pe ur vartezeadenn greñv diazezet war fedoù
kinniget}\label{gallout_a_ra_ezteurel_ur_chlozadur_poellek_logical_conclusion_pe_ur_vartezeadenn_greuxf1v_diazezet_war_fedouxf9_kinniget}

\section{Stummoù damskoazellañ
all}\label{stummouxf9_damskoazellauxf1_all}

\subsection{\texorpdfstring{\emph{Ret}}{Ret}}\label{ret}

Ret a zegas mennozh an dlead moral, ar redi lezennel pe an ezhomm
boutin. Pa vez liammet gant ar gopulenn, e teu da vezañ ur stumm
damskoazellañ
(\href{https://arbres.iker.cnrs.fr/index.php?title=Ret}{ARBRES : Ret}).
