\begin{LTR}
\begin{otherlanguage}{breton}

\subsection[\emph{War} + anv-verb implijet evit merkañ padelezh ur
c'hemm]{\texorpdfstring{\protect\hypertarget{War_.2B_anv-verb_implijet_evit_merka.C3.B1_padelezh_ur_c.27hemm}{}{}\emph{War}
+ anv-verb implijet evit merkañ padelezh ur
c'hemm}{War + anv-verb implijet evit merkañ padelezh ur c'hemm}}\label{War_+_anv-verb_implijet_evit_merkauxf1_padelezh_ur_cux27hemm}

{{{[}}\href{/wiki/Amzer?action=edit&section=1}{{modifier}}{{]}}}

Ar stumm \textbf{\emph{war} + anv-verb} a implijer evit merkañ padelezh
ur c'hemm pe eztaoler an obererezh a
c'hallfe c'hoarvezout pe a
c'hoarvezo a-benn nebeut.

Les formes des maisons~s'affaiblissent,
s'exténuent.

Queffélec 1970, 6

War wanaat, war ziviañ ez ae~trolinennoù an tiez.

Beyer 2016, 8

\subsection[Ar stumm-danevelliñ implijet e brezhoneg evit treiñ
lavarennoù en amzer
dremenet]{\texorpdfstring{\protect\hypertarget{Ar_stumm-danevelli.C3.B1_implijet_e_brezhoneg_evit_trei.C3.B1_lavarenno.C3.B9_en_amzer_dremenet}{}{}Ar
stumm-danevelliñ implijet e brezhoneg evit treiñ lavarennoù en amzer
dremenet}{Ar stumm-danevelliñ implijet e brezhoneg evit treiñ lavarennoù en amzer dremenet}}\label{Ar_stumm-danevelliuxf1_implijet_e_brezhoneg_evit_treiuxf1_lavarennouxf9_en_amzer_dremenet}

{{{[}}\href{/wiki/Amzer?action=edit&section=2}{{modifier}}{{]}}}

Ar \href{/wiki/Stumm-danevelli\%C3\%B1}{stumm-danevelliñ} a zo ur stumm
ma implijer un anv-verb e-lec'h ur verb displeget.

Il rétorquait simplement~: -- C'est vrai...

Queffélec 1970, 45

Hag eñ hag eilgeriañ hep mui ken: -- Gwir eo...

Beyer 2016, 25

«~Mes chers frères~», leur dit-til, et il entama un discours véhément
contre les mauvais chrétiens.

Queffélec 1970, 46

``Va breudeur ker,'' emezañ dezho, ha kregiñ neuze gant ur brezegenn
daer a-enep ar gristenien fall.

Beyer 2016, 26

Le sacristain s'approcha de la petite fenêtre et
contempla la dune rase, les rochers, le ciel bleu.

Queffélec 1970, 47

Tostaat a reas ar sakrist ouzh ar prenestrig hag arvestiñ ouzh an tevenn
touz, ar reier, an oabl glas.

Beyer 2016, 29

\subsection{\texorpdfstring{\emph{War-nes} a ginnig un obererezh
darig}{War-nes a ginnig un obererezh darig}}\label{War-nes_a_ginnig_un_obererezh_darig}

{{{[}}\href{/wiki/Amzer?action=edit&section=3}{{modifier}}{{]}}}

An araogenn \emph{war-nes} a glot gant sur \emph{le point de (faire
qch)} e galleg. Kinnig a ra un obererezh a c'hoarvezo
diozhtu.

Il pensait maintenant que cette fille était belle, il se souvenait
qu'on la célébrait comme la plus belle, et il fut sur le
point de bondir par-dessus le mur de pierres sèches, par-dessus un autre
mur, un autre et un autre, et de courir après Scolastique.

Queffélec 1970, 54

Bremañ e kave gantañ e oa kaer ar plac'h-se, e teue da
soñj dezhañ e oa brudet evit bezañ an hini gaerañ, ha war-nes lammat
dreist ar voger vein sec'h e voe, dreist ur voger all,
hag un all, hag un all, ha redek war-lerc'h Scolastique.

Beyer 2016, 34

Implijet e c'hall bezañ ivez evit treiñ an
\href{/wiki/Amzer_dremenet_ledan}{amzer dremenet ledan}:

Il allait partir.

Queffélec 1970, 29

War-nes mont kuit e oa.

Beyer 2016, 12

\subsection{\texorpdfstring{Stumm \emph{pare da} evit kinnig un
obererezh
darig}{Stumm pare da evit kinnig un obererezh darig}}\label{Stumm_pare_da_evit_kinnig_un_obererezh_darig}

{{{[}}\href{/wiki/Amzer?action=edit&section=4}{{modifier}}{{]}}}

Tost eo ar stumm-mañ ouzh an hini meneget a-us. \emph{Bezañ war ar pare}
a dalvez bezañ prest d'ober un dra bennak, \emph{être
sur le point de faire quelque chose} e galleg.

Le dimanche, quand il retrouva devant lui la paroisse attentive, prête à
l'admirer, Thomas Gourvennec eut une défaillance.

Queffélec 1970, 57

D'ar Sul, pa 'n em gavas ar barreziad
evezhiek dirazañ, pare da vezañ bamet gantañ, e fallaas Tomaz
Gourvenneg.

Beyer 2016, 37

Au lieu de cela, toutes étaient vivantes, toutes étaient là, sur la
dune, prêtes à reprendre la mer dès le premier calme...

Queffélec 1970, 152

E-lec'h se e oant bev holl, holl e oant aze war an
tevenn, pare da vont war vor en-dro kerkent hag ar galmijenn gentañ...

Beyer 2016, 132

\end{otherlanguage}
\end{LTR}
