\chapter{Amzer}

\section{\texorpdfstring{\emph{War} + anv-verb implijet evit merkañ padelezh ur c\textquotesingle hemm}{War + anv-verb implijet evit merkañ padelezh ur c\textquotesingle hemm}}\label{war_anv_verb_implijet_evit_merkauxf1_padelezh_ur_chemm}

Ar stumm \textbf{\emph{war} + anv-verb} a implijer evit merkañ padelezh ur c\textquotesingle hemm pe eztaoler an obererezh a c\textquotesingle hallfe c\textquotesingle hoarvezout pe a c\textquotesingle hoarvezo a-benn nebeut.

\begin{longtable}[]{|p{0.45\textwidth}|p{0.45\textwidth}}

Les formes des maisons~s\textquotesingle affaiblissent, s\textquotesingle exténuent. & War wanaat, war ziviañ ez ae~trolinennoù an tiez. \\ Queffélec 1970, 6 & Beyer 2016, 8 \\ \end{longtable}

\section{Ar stumm-danevelliñ implijet e brezhoneg evit treiñ lavarennoù en amzer dremenet}\label{ar_stumm_danevelliuxf1_implijet_e_brezhoneg_evit_treiuxf1_lavarennouxf9_en_amzer_dremenet}

Ar stumm-danevelliñ a zo ur stumm ma implijer un anv-verb e-lec\textquotesingle h ur verb displeget.

\begin{longtable}[]{|p{0.45\textwidth}|p{0.45\textwidth}}

Il rétorquait simplement~: -- C\textquotesingle est vrai... & Hag eñ hag eilgeriañ hep mui ken: -- Gwir eo... \\ Queffélec 1970, 45 & Beyer 2016, 25 \\ \end{longtable}

\begin{longtable}[]{|p{0.45\textwidth}|p{0.45\textwidth}}

«~Mes chers frères~», leur dit-til, et il entama un discours véhément contre les mauvais chrétiens. & ``Va breudeur ker,'' emezañ dezho, ha kregiñ neuze gant ur brezegenn daer a-enep ar gristenien fall. \\ Queffélec 1970, 46 & Beyer 2016, 26 \\ \end{longtable}

\begin{longtable}[]{|p{0.45\textwidth}|p{0.45\textwidth}}

Le sacristain s\textquotesingle approcha de la petite fenêtre et contempla la dune rase, les rochers, le ciel bleu. & Tostaat a reas ar sakrist ouzh ar prenestrig hag arvestiñ ouzh an tevenn touz, ar reier, an oabl glas. \\ Queffélec 1970, 47 & Beyer 2016, 29 \\ \end{longtable}

\section{\texorpdfstring{\emph{War-nes} a ginnig un obererezh darig}{War-nes a ginnig un obererezh darig}}\label{war_nes_a_ginnig_un_obererezh_darig}

An araogenn \emph{war-nes} a glot gant sur \emph{le point de (faire qch)} e galleg. Kinnig a ra un obererezh a c\textquotesingle hoarvezo diozhtu.

\begin{longtable}[]{|p{0.45\textwidth}|p{0.45\textwidth}}

Il pensait maintenant que cette fille était belle, il se souvenait qu\textquotesingle on la célébrait comme la plus belle, et il fut sur le point de bondir par-dessus le mur de pierres sèches, par-dessus un autre mur, un autre et un autre, et de courir après Scolastique. & Bremañ e kave gantañ e oa kaer ar plac\textquotesingle h-se, e teue da soñj dezhañ e oa brudet evit bezañ an hini gaerañ, ha war-nes lammat dreist ar voger vein sec\textquotesingle h e voe, dreist ur voger all, hag un all, hag un all, ha redek war-lerc\textquotesingle h Scolastique. \\ Queffélec 1970, 54 & Beyer 2016, 34 \\ \end{longtable}

Implijet e c\textquotesingle hall bezañ ivez evit treiñ an amzer dremenet ledan:

\begin{longtable}[]{|p{0.45\textwidth}|p{0.45\textwidth}}

Il allait partir. & War-nes mont kuit e oa. \\ Queffélec 1970, 29 & Beyer 2016, 12 \\ \end{longtable}

\section{\texorpdfstring{Stumm \emph{pare da} evit kinnig un obererezh darig}{Stumm pare da evit kinnig un obererezh darig}}\label{stumm_pare_da_evit_kinnig_un_obererezh_darig}

Tost eo ar stumm-mañ ouzh an hini meneget a-us. \emph{Bezañ war ar pare} a dalvez bezañ prest d\textquotesingle ober un dra bennak, \emph{être sur le point de faire quelque chose} e galleg.

\begin{longtable}[]{|p{0.45\textwidth}|p{0.45\textwidth}}

Le dimanche, quand il retrouva devant lui la paroisse attentive, prête à l\textquotesingle admirer, Thomas Gourvennec eut une défaillance. & D\textquotesingle ar Sul, pa \textquotesingle n em gavas ar barreziad evezhiek dirazañ, pare da vezañ bamet gantañ, e fallaas Tomaz Gourvenneg. \\ Queffélec 1970, 57 & Beyer 2016, 37 \\ \end{longtable}

\begin{longtable}[]{|p{0.45\textwidth}|p{0.45\textwidth}}

Au lieu de cela, toutes étaient vivantes, toutes étaient là, sur la dune, prêtes à reprendre la mer dès le premier calme... & E-lec\textquotesingle h se e oant bev holl, holl e oant aze war an tevenn, pare da vont war vor en-dro kerkent hag ar galmijenn gentañ... \\ Queffélec 1970, 152 & Beyer 2016, 132 \\ \end{longtable} 