\chapter{Présentation du projet en français}

\section{STYLE}\label{style}

Le nom STYLE est un acronyme breton de \textbf{S}tudi an
\textbf{T}roidigezhioù er \textbf{Y}ezh \textbf{Le}nnegel: Étude des
traductions en langue littéraire. Ce site constitue mon carnet de
recherche sur l\textquotesingle étude stylistique du breton, commencé
dans le cadre de mon mémoire de master à l\textquotesingle Université
Rennes 2.

\section{Résumé du travail}\label{ruxe9sumuxe9_du_travail}

Cette recherche vise à construire une ressource de stylistique
directement utilisable pour la communauté parlante du breton, classée
comme « langue sérieusement en danger » selon l\textquotesingle UNESCO
(\href{Moseley_2010}{Moseley 2010}, 24-25). Ma recherche applique la
méthode de la stylistique comparée suivant le modèle de
\href{Rottet_&_Moris_2018}{Rottet et Moris (2018)}, lui-même dérivant du
travail de \href{Darbelnet_&_Vinay_1993}{Darbelnet \& Vinay (1993)}.
Elle consiste à relever les structures nécessitant commentaire à partir
d\textquotesingle un corpus de traduction français \textgreater{} breton
(\href{Queffélec_1970}{Queffélec 1970} traduit par
\href{Beyer_2016}{Beyer 2016}) et breton \textgreater{} français
(\href{Drezen_2012}{Drezen 2012} traduit par \href{Drezen_1943}{Drezen
1943}), et à les organiser dans une forme directement préhensible,
cherchable, par les utilisatrices et utilisateurs, principalement
traducteurs. Elle fournit un recensement traductologique des structures,
et renvoie aux grammaires existantes pour leur description et étude
(\href{Kervella_1995}{Kervella 1995},
\href{Jouitteau_2009-2025}{Jouitteau 2009-2025}). La rédaction numérique
choisie pour ce travail en permettra la distribution en se basant sur le
format MediaWiki, et l'améliorabilité au-delà : ce format permettra de
continuer le projet de façon incrémentale par l'enrichissement du corpus
de texte traduit au-delà des deux ouvrages choisis, et potentiellement
d'y intégrer un aspect collaboratif. Finalement, l'ouvrage numérique
sous une licence CC-BY constitue aussi fondamentalement un corpus aligné
récupérable pour les entrainements d'outils de l\textquotesingle IA,
corpus nécessaire aux langues numériquement sous-développées pour
compenser leur carence en corpus brut (\href{Jouitteau_2023d}{Jouitteau
2023d}).

Mots-clés : breton, langue minoritaire, stylistique comparée,
traduction.

\section{But}\label{but}

Construire une ressource de stylistique directement utilisable pour la
communauté parlante d\textquotesingle une langue en haut danger
d\textquotesingle extinction selon l\textquotesingle UNESCO
(\href{Moseley_2010}{Moseley 2010}, 24-25).

\section{Méthodologie}\label{muxe9thodologie}

À partir d\textquotesingle un corpus de traduction français
\textgreater{} breton (\href{Queffélec_1970}{Queffélec 1970} traduit par
\href{Beyer_2016}{Beyer 2016}) et breton \textgreater{} français
(\href{Drezen_2012}{Drezen 2012} traduit par \href{Drezen_1943}{Drezen
1943}), relever les structures nécessitant commentaire, suivant la
méthode de la stylistique comparée (\href{Rottet_&_Moris_2018}{Rottet \&
Moris 2018}, \href{Darbelnet_&_Vinay_1993}{Darbelnet \& Vinay 1993}).

Les organiser dans une forme directement préhensible, cherchable, par
les utilisatrices et utilisateurs (principalement traducteurs). Fournir
un recensement traductologique des structures et faire référence aux
grammaires existantes pour leur description et étude.

Conduire le travail sous forme numérique en utilisant MediaWiki, un
logiciel de gestion de contenu collaboratif fonctionnant sur le principe
du wiki et permettant de créer, lier et structurer des pages.

\section{La question de la terminologie
grammaticale}\label{la_question_de_la_terminologie_grammaticale}

Le but de créer un matériel directement préhensible pour la communauté
brittophone amène une double contrainte : rédiger l'intégrité des
commentaires en breton dans une forme qui soit également accessible aux
lecteurs non spécialisés en linguistique. Le principal défi est ici
d'adopter une terminologie grammaticale, stylistique et traductologique
et regrouper les termes dans un glossaire terminologique. Dans ce
glossaire, on a à la fois la traduction trilingue (en, fr, br), mais
aussi des définitions et exemples des termes utilisés. Pour le breton,
il y a des cas où une terminologie standardisée fait défaut. Dans ce
cas, il faut faire des choix terminologiques, soit en abandonnant des
termes utilisés par certains auteurs au profit de ceux utilisés par
d'autres, soit en inventant une traduction des termes anglophones et
francophones.

Un des exemples que j'explore dans mon travail contient une structure
débutant par un groupe nominal qui ressemble au sujet sous certains
aspects syntaxiques, mais n\textquotesingle est pas le sujet de la
phrase (\href{Jouitteau_2009-2025}{Jouitteau 2009-2025}). Jouitteau
(2005/2010) et Rezac (2009) l\textquotesingle appellent la construction
du faux sujet en français et wrong subject construction en anglais. Le
Clerc (1986) la nomme un complément anticipé, Urien (1989) --- une
relation médiate, Fave (1998) --- un complément redoublé. Fave fournit
l\textquotesingle équivalent breton du terme proposé (\emph{renadenn
adveneget}), alors que dans d\textquotesingle autres sources, la
terminologie est présentée exclusivement en français. Dans ce cas, la
difficulté est donc de déterminer d\textquotesingle abord lequel des
termes français correspond le mieux à l\textquotesingle exemple
recherché, puis d\textquotesingle en trouver (ou inventer)
l\textquotesingle équivalent breton.

Afin d\textquotesingle assurer la qualité des termes traduits ou
inventés, il serait intéressant de solliciter le conseil scientifique de
l\textquotesingle association \href{https://brezhoneg21.com/}{Kreizenn
ar Geriaouiñ}, créée à l\textquotesingle initiative de Diwan afin
d\textquotesingle élaborer et de fournir les outils terminologiques et
pédagogiques nécessaires à l\textquotesingle ouverture du premier
collège Diwan.
