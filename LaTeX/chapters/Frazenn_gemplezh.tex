\begin{LTR}
\begin{otherlanguage}{breton}

\subsection[Lavarennoù kenurzhiet troet gant un islavarenn
degouezh]{\texorpdfstring{\protect\hypertarget{Lavarenno.C3.B9_kenurzhiet_troet_gant_un_islavarenn_degouezh}{}{}Lavarennoù
kenurzhiet troet gant un islavarenn
degouezh}{Lavarennoù kenurzhiet troet gant un islavarenn degouezh}}\label{Lavarennouxf9_kenurzhiet_troet_gant_un_islavarenn_degouezh}

{{{[}}\href{/wiki/Frazenn_gemplezh?action=edit&section=1}{{modifier}}{{]}}}

An
\href{/wiki/Islavarenno\%C3\%B9_degouezh?action=edit&redlink=1}{islavarennoù
degouezh} a zo frazennoù n'o deus ket ur verb, pe
n'eo ket displeget ar verb-se.

Sur le continent, des maisons humaines, des fermes qui se disent
pauvres, mais où la lande étincelle dans les cheminées plus belle
qu'à la floraison de Pâques;

Queffélec 1970, 5

War an douar bras, tiez tud, atantoù o tiskouez paourentez pa sked avat
al lann en o oaledoù, kaeroc'h eget bleuniadur Pask;

Beyer 2016, 7

Les animaux, derrière la cloison, qui réchauffent les maîtres;

Queffélec 1970, 5

Al loened a-dreñv ar speurenn o tommañ ar vistri;

Beyer 2016, 7

Le fumier gras, dehors, qui suinte comme du beurre;

Queffélec 1970, 5

An teil druz, er-maez, o ouelañ evel amann;

Beyer 2016, 7

\subsection[Renadennoù abeg troet e bzg gant ar stagell \emph{dre
ma}]{\texorpdfstring{\protect\hypertarget{Renadenno.C3.B9_abeg_troet_e_bzg_gant_ar_stagell_dre_ma}{}{}Renadennoù
abeg troet e bzg gant ar stagell \emph{dre
ma}}{Renadennoù abeg troet e bzg gant ar stagell dre ma}}\label{Renadennouxf9_abeg_troet_e_bzg_gant_ar_stagell_dre_ma}

{{{[}}\href{/wiki/Frazenn_gemplezh?action=edit&section=2}{{modifier}}{{]}}}

Après les cantiques, comme il redoutait le blâme des pêcheurs, il gagna
la sacristie au lieu de suivre la foule.

Queffélec 1970, 47

War-lerc'h ar c'hantikoù, dre ma touje
karez ar besketaerien ez eas d'ar sakristiri
e-lec'h mont da heul an engroez.

Beyer 2016, 29

\end{otherlanguage}
\end{LTR}
