\chapter{Frazenn gemplezh}

\section{Lavarennoù kenurzhiet troet gant un islavarenn
degouezh}\label{lavarennouxf9_kenurzhiet_troet_gant_un_islavarenn_degouezh}

An \href{islavarennoù_degouezh}{islavarennoù degouezh} a zo frazennoù
n\textquotesingle o deus ket ur verb, pe n\textquotesingle eo ket
displeget ar verb-se.

\begin{longtable}[]{@{}ll@{}}
\toprule\noalign{}
\endhead
\bottomrule\noalign{}
\endlastfoot
Sur le continent, des maisons humaines, des fermes qui se disent
pauvres, mais où la lande étincelle dans les cheminées plus belle
qu\textquotesingle à la floraison de Pâques; & War an douar bras, tiez
tud, atantoù o tiskouez paourentez pa sked avat al lann en o oaledoù,
kaeroc\textquotesingle h eget bleuniadur Pask; \\
Queffélec 1970, 5 & Beyer 2016, 7 \\
\end{longtable}

\begin{longtable}[]{@{}ll@{}}
\toprule\noalign{}
\endhead
\bottomrule\noalign{}
\endlastfoot
Les animaux, derrière la cloison, qui réchauffent les maîtres; & Al
loened a-dreñv ar speurenn o tommañ ar vistri; \\
Queffélec 1970, 5 & Beyer 2016, 7 \\
\end{longtable}

\begin{longtable}[]{@{}ll@{}}
\toprule\noalign{}
\endhead
\bottomrule\noalign{}
\endlastfoot
Le fumier gras, dehors, qui suinte comme du beurre; & An teil druz,
er-maez, o ouelañ evel amann; \\
Queffélec 1970, 5 & Beyer 2016, 7 \\
\end{longtable}

\section{\texorpdfstring{Renadennoù abeg troet e bzg gant ar stagell
\emph{dre
ma}}{Renadennoù abeg troet e bzg gant ar stagell dre ma}}\label{renadennouxf9_abeg_troet_e_bzg_gant_ar_stagell_dre_ma}

\begin{longtable}[]{@{}ll@{}}
\toprule\noalign{}
\endhead
\bottomrule\noalign{}
\endlastfoot
Après les cantiques, comme il redoutait le blâme des pêcheurs, il gagna
la sacristie au lieu de suivre la foule. & War-lerc\textquotesingle h ar
c\textquotesingle hantikoù, dre ma touje karez ar besketaerien ez eas
d\textquotesingle ar sakristiri e-lec\textquotesingle h mont da heul an
engroez. \\
Queffélec 1970, 47 & Beyer 2016, 29 \\
\end{longtable}
