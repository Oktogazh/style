\chapter{Doareerien perc'hennañ}

\section{Terminologiezh}\label{terminologiezh}

Meur a dermen a zo evit komz eus gerioù perc\textquotesingle hennañ
(\emph{ma, da, e, he} hag all). E galleg e vez graet \emph{déterminants
possessifs} anezho, daoust m\textquotesingle o c\textquotesingle haver
dindan an anvioù \emph{pronoms possessifs} pe \emph{adjectifs
possessifs} ivez
(\href{https://arbres.iker.cnrs.fr/index.php?title=D\%C3\%A9terminants_possessifs\#Terminologie}{Jouitteau
2009-2025}).

E brezhoneg e reer \emph{anvioù-gwan perc\textquotesingle hennañ anezho}
kentoc\textquotesingle h eget \emph{raganvioù
perc\textquotesingle hennañ}. Kervella (\href{Kervella_1995}{1995,
259-260}) a zispleg n\textquotesingle eus ket a wir
raganvioù-perc\textquotesingle hennañ e brezhoneg. Hervezañ, e stummoù
evel \emph{ma hini}, \emph{da hini,}e hini\emph{,}he re\emph{h.a.
n\textquotesingle eus nemet un anv-gwan perc\textquotesingle hennañ
(}ma, da, e\emph{hag all) hag ur ger diskouezañ (}hini,
re\textquotesingle\textquotesingle) lakaet war e lerc\textquotesingle h.

Evel displeget gant Jouitteau
\href{https://arbres.iker.cnrs.fr/index.php?title=D\%C3\%A9terminants_possessifs}{(2009-2025)},
ne c\textquotesingle hallont ket bezañ \emph{anvioù-gwan
perc\textquotesingle hennañ} na \emph{raganvioù
perc\textquotesingle hennañ}, dre ma ne erlec\textquotesingle hiont
morse an anv-gwan pe an anv.

War a seblannt n\textquotesingle eus ket a droidigezh vrezhonek resis
ebet evit an termen \emph{déterminants possessifs} evit ar mare. E
\href{KAG_2016}{KAG 2016} e kinniger \emph{ger spisaat} evit
\emph{déterminant} ha \emph{ger perc\textquotesingle hannañ} evit
\emph{possessif}. Posupl e vefe implijout \emph{gerioù spisaat
perc\textquotesingle hennañ} evit komz eus \emph{déterminants
possessifs}. Mod all e kaver an termen \emph{doareer} e
\href{https://devri.bzh/recherche/?q=doareer&submit=}{Devri} kinniget
gant \href{Kervella_1962}{Kervella 1962}. Etre an div opsion-mañ e
hañval skañvoc\textquotesingle h \emph{doareer
perc\textquotesingle hennañ}, pehini a vo implijet el labour-mañ.

\section{Gerioù-mell gallek troet gant doareerien
perc\textquotesingle hennañ e
brezhoneg}\label{geriouxf9_mell_gallek_troet_gant_doareerien_perchennauxf1_e_brezhoneg}

Ar gerioù-perc\textquotesingle hennañ a implijer alies e brezhoneg pa
vez implijet kentoc\textquotesingle h ur ger-mell e galleg evit diskouez
ar berc\textquotesingle henniezh:

(An trede bloavezh ma oa ar person war an enez)

Er skouer da heul e voe miret stumm unan an anv gant ar ger-mell
(\emph{la tête}) e galleg evit komz eus ur strollad tud.

E troidigezh vrezhonek ar frazenn resis-mañ e voe lakaet an anv-kadarn
el liester, diaraoget gant un doareer perc\textquotesingle hennañ
(\emph{o fennoù}). Koulskoude e kaver skouerioù lec\textquotesingle h
m\textquotesingle eo dalc\textquotesingle het ar stumm unan e
degouezhioù a-seurt-se, evel er frazenn \emph{Dudiusoc'h eo
da\textquotesingle m skouarn klevet al labous o kanañ gae war beg e
skourr, eget youc\textquotesingle herezh paotred kêr, savet ar banne d'o
fenn} \href{https://niverel.brezhoneg.bzh/br/meurgorf/30821}{Meurgorf}.

Muioc\textquotesingle h a skouerioù a ranker kavout evit resisaat ar
patrom implijet aliesoc\textquotesingle h.

\section{Anvioù noaz gallek troet gant doareerien
perc\textquotesingle hennañ e
brezhoneg}\label{anviouxf9_noaz_gallek_troet_gant_doareerien_perchennauxf1_e_brezhoneg}

Pa implijer un
\href{https://arbres.iker.cnrs.fr/index.php?title=Noms_nus\#Terminologie}{anv
noaz} e galleg e c\textquotesingle haller ouzhpennañ un anv-gwañ
perc\textquotesingle hennañ en droidigezh vrezhonek:

\section{Doareerien perc\textquotesingle hennañ implijet e brezhoneg
pa gomzer diwar-benn saviadoù ar
c\textquotesingle horf}\label{doareerien_perchennauxf1_implijet_e_brezhoneg_pa_gomzer_diwar_benn_saviadouxf9_ar_chorf}
