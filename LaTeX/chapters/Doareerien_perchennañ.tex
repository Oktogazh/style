\chapter{Doareerien perc'hennañ}

\section{Terminologiezh}\label{terminologiezh}

Meur a dermen a zo evit komz eus gerioù perc\textquotesingle hennañ
(\emph{ma, da, e, he} hag all). E galleg e vez graet \emph{déterminants
possessifs} anezho, daoust m\textquotesingle o c\textquotesingle haver
dindan an anvioù \emph{pronoms possessifs} pe \emph{adjectifs
possessifs} ivez
(\href{https://arbres.iker.cnrs.fr/index.php?title=D\%C3\%A9terminants_possessifs\#Terminologie}{Jouitteau
2009-2025}).

E brezhoneg e reer \emph{anvioù-gwan perc\textquotesingle hennañ anezho}
kentoc\textquotesingle h eget \emph{raganvioù
perc\textquotesingle hennañ}. Kervella (\href{Kervella_1995}{1995,
259-260}) a zispleg n\textquotesingle eus ket a wir
raganvioù-perc\textquotesingle hennañ e brezhoneg. Hervezañ, e stummoù
evel \emph{ma hini}, \emph{da hini,}e hini\emph{,}he re\emph{h.a.
n\textquotesingle eus nemet un anv-gwan perc\textquotesingle hennañ
(}ma, da, e\emph{hag all) hag ur ger diskouezañ (}hini,
re\textquotesingle\textquotesingle) lakaet war e lerc\textquotesingle h.

Evel displeget gant Jouitteau
\href{https://arbres.iker.cnrs.fr/index.php?title=D\%C3\%A9terminants_possessifs}{(2009-2025)},
ne c\textquotesingle hallont ket bezañ \emph{anvioù-gwan
perc\textquotesingle hennañ} na \emph{raganvioù
perc\textquotesingle hennañ}, dre ma ne erlec\textquotesingle hiont
morse an anv-gwan pe an anv.

War a seblannt n\textquotesingle eus ket a droidigezh vrezhonek resis
ebet evit an termen \emph{déterminants possessifs} evit ar mare. E
\href{KAG_2016}{KAG 2016} e kinniger \emph{ger spisaat} evit
\emph{déterminant} ha \emph{ger perc\textquotesingle hannañ} evit
\emph{possessif}. Posupl e vefe implijout \emph{gerioù spisaat
perc\textquotesingle hennañ} evit komz eus \emph{déterminants
possessifs}. Mod all e kaver an termen \emph{doareer} e
\href{https://devri.bzh/recherche/?q=doareer&submit=}{Devri} kinniget
gant \href{Kervella_1962}{Kervella 1962}. Etre an div opsion-mañ e
hañval skañvoc\textquotesingle h \emph{doareer
perc\textquotesingle hennañ}, pehini a vo implijet el labour-mañ.

\section{Gerioù-mell gallek troet gant doareerien
perc\textquotesingle hennañ e
brezhoneg}\label{geriouxf9_mell_gallek_troet_gant_doareerien_perchennauxf1_e_brezhoneg}

Ar gerioù-perc\textquotesingle hennañ a implijer alies e brezhoneg pa
vez implijet kentoc\textquotesingle h ur ger-mell e galleg evit diskouez
ar berc\textquotesingle henniezh:

\begin{longtable}[]{p{0.5\textwidth}p{0.5\textwidth}}

Sur le continent, des maisons humaines, des fermes qui se disent
pauvres, mais où la lande étincelle dans les cheminées plus belle
qu\textquotesingle à la floraison de Pâques; & War an douar bras, tiez
tud, atantoù o tiskouez paourentez pa sked avat al lann en o oaledoù,
kaeroc\textquotesingle h eget bleuniadur Pask; \\
Queffélec 1970, 5 & Beyer 2016, 7 \\
\end{longtable}

\begin{longtable}[]{p{0.5\textwidth}p{0.5\textwidth}}

A la fin de~la~troisième année, c\textquotesingle est Anne Le Berre qui
sort lever des lignes à cent mètres du port, derrière un rocher
tranquille. & E dibenn~e~drede bloavezh, sed aze Ann ar Berr o vont da
sevel linennoù, kant metrad diouzh ar porzh, a-dreñv ur garreg
habask. \\
Queffélec 1970, 6 & Beyer 2016, 8 \\
\end{longtable}

(An trede bloavezh ma oa ar person war an enez)

\begin{longtable}[]{p{0.5\textwidth}p{0.5\textwidth}}

Il leva la main pour s\textquotesingle essuyer le front~; au lieu de
s\textquotesingle essuyer le front, il se signa. & Sevel a reas e zorn
da sec\textquotesingle hiñ e dal; e-lec\textquotesingle h
sec\textquotesingle hiñ e dal e reas sin ar groaz. \\
Queffélec 1970, 57 & Beyer 2016, 37 \\
\end{longtable}

Er skouer da heul e voe miret stumm unan an anv gant ar ger-mell
(\emph{la tête}) e galleg evit komz eus ur strollad tud.

\begin{longtable}[]{p{0.5\textwidth}p{0.5\textwidth}}

Ils s\textquotesingle assoient sur des mottes de terre un peu hautes et,
levant la tête, la tournent de droite à gauche et de gauche à droite
comme s\textquotesingle ils voulaient la dévisser. & Koazezañ a reont
war voudennoù uhelik ha, savet o fennoù, e troont anezho a gleiz da
zehoù hag a zehoù da gleiz e-giz pa glaskjent o diviñsañ. \\
Queffélec 1970, 6 & Beyer 2016, 8 \\
\end{longtable}

E troidigezh vrezhonek ar frazenn resis-mañ e voe lakaet an anv-kadarn
el liester, diaraoget gant un doareer perc\textquotesingle hennañ
(\emph{o fennoù}). Koulskoude e kaver skouerioù lec\textquotesingle h
m\textquotesingle eo dalc\textquotesingle het ar stumm unan e
degouezhioù a-seurt-se, evel er frazenn \emph{Dudiusoc'h eo
da\textquotesingle m skouarn klevet al labous o kanañ gae war beg e
skourr, eget youc\textquotesingle herezh paotred kêr, savet ar banne d'o
fenn} \href{https://niverel.brezhoneg.bzh/br/meurgorf/30821}{Meurgorf}.

Muioc\textquotesingle h a skouerioù a ranker kavout evit resisaat ar
patrom implijet aliesoc\textquotesingle h.

\section{Anvioù noaz gallek troet gant doareerien
perc\textquotesingle hennañ e
brezhoneg}\label{anviouxf9_noaz_gallek_troet_gant_doareerien_perchennauxf1_e_brezhoneg}

Pa implijer un
\href{https://arbres.iker.cnrs.fr/index.php?title=Noms_nus\#Terminologie}{anv
noaz} e galleg e c\textquotesingle haller ouzhpennañ un anv-gwañ
perc\textquotesingle hennañ en droidigezh vrezhonek:

\begin{longtable}[]{p{0.5\textwidth}p{0.5\textwidth}}

les petites fenêtres~qui emprisonnent~dans l\textquotesingle air le
froid, la pluie et le vent, et les obligent, pour réclamer pitance, à ne
passer qu\textquotesingle un bras maigre qui tâtonne; & ar prenestri
bihan~hag a zalc\textquotesingle h~bac\textquotesingle het en aer~ar
yenien, ar glav hag an avel, ken ne c\textquotesingle hellont nemet
tremen ur vrec\textquotesingle h dreut war dastorn evit goulenn o zamm
kreun; \\
Queffélec 1970, 5 & Beyer 2016, 7 \\
\end{longtable}

\section{Doareerien perc\textquotesingle hennañ implijet e brezhoneg
pa gomzer diwar-benn saviadoù ar
c\textquotesingle horf}\label{doareerien_perchennauxf1_implijet_e_brezhoneg_pa_gomzer_diwar_benn_saviadouxf9_ar_chorf}

\begin{longtable}[]{p{0.5\textwidth}p{0.5\textwidth}}

Elle se retournait, déjà les gens qui stationnaient dans le bas de
l\textquotesingle église passaient dans le cimetière. & Treiñ a rae he
c\textquotesingle hein ha kentizh e kroge an dud bet e traoñ an iliz da
dremen er vered. \\
Queffélec 1970, 44 & Beyer 2016, 24 \\
\end{longtable}

\begin{longtable}[]{p{0.5\textwidth}p{0.5\textwidth}}

Trois hommes, assis sur des escabeaux, lui firent place. & Tri gwaz en o
c\textquotesingle hoazez war skabelloù a reas plas dezhañ. \\
Queffélec 1970, 48 & Beyer 2016, 29 \\
\end{longtable}

\begin{longtable}[]{p{0.5\textwidth}p{0.5\textwidth}}

Le dimanche, de très bonne heure, les gens se réunissaient dans
l\textquotesingle église, les hommes à droite, les femmes à gauche, tous
debout ou à genoux {[}...{]}. & Da Sul vintin, abred-tre, en em dolpe an
dud en iliz, ar wazed a-zehoù, ar maouezed a-gleiz, an holl en o sav pe
war o daoulin {[}...{]}. \\
Queffélec 1970, 43 & Beyer 2016, 23 \\
\end{longtable}

\begin{longtable}[]{p{0.5\textwidth}p{0.5\textwidth}}

Thomas, à genoux, ne murmurait aucun mot de prière. & Ne oa ket Tomaz o
vouslavarout gerioù pediñ war e zaoulin. \\
Queffélec 1970, 52 & Beyer 2016, 32 \\
\end{longtable}
