\chapter{Doareerien perc'hennañ}
\section{Terminologiezh}\label{Terminologiezh}

{{{[}}\href{/wiki/Doareerien_perc\%27henna\%C3\%B1?action=edit&section=1}{{modifier}}{{]}}}

Meur a dermen a zo evit komz eus gerioù perc'hennañ
(\emph{ma, da, e, he} hag all). E galleg e vez graet \emph{déterminants
possessifs} anezho, daoust m'o kaver dindan an anv
\emph{pronoms possessifs} pe \emph{adjectifs possessifs} ivez
(\href{https://arbres.iker.cnrs.fr/index.php?title=D\%C3\%A9terminants_possessifs\#Terminologie}{Jouitteau
2009-2025}).

E brezhoneg e reer \emph{anvioù-gwan perc'hennañ anezho}
kentoc'h eget \emph{raganvioù
perc'hennañ}. Kervella (\href{/wiki/Kervella_1995}{1995,
259-260}) a zispleg n'eus ket a wir
raganvioù-perc'hennañ e brezhoneg. Hervezañ, e stummoù
evel \emph{ma hini}, \emph{da hini,} e hini\emph{,} he re \emph{h.a.
n'eus nemet un anv-gwan perc'hennañ
(}ma, da, e \emph{hag all) hag ur ger diskouezañ (}hini, re\emph{)
lakaet war e lerc'h.}

Evel displeget gant Jouitteau
\href{https://arbres.iker.cnrs.fr/index.php?title=D\%C3\%A9terminants_possessifs}{(2009-2025)},
ne c'hallont ket bezañ \emph{anvioù-gwan
perc'hennañ} na \emph{raganvioù
perc'hennañ}, dre ma ne erlec'hiont
morse an anv-gwan pe an anv.

War a seblannt n'eus ket a droidigezh vrezhonek resis
ebet evit an termen \emph{déterminants possessifs} evit ar mare. E
\href{/wiki/KAG_2016}{KAG 2016} e kinniger \emph{ger spisaat} evit
\emph{déterminant} ha \emph{ger perc'hannañ} evit
\emph{possessif}. Posupl e vefe implijout \emph{gerioù spisaat
perc'hennañ} evit komz eus \emph{déterminants
possessifs}. Mod all e kaver an termen \emph{doareer} e
\href{https://devri.bzh/recherche/?q=doareer&submit=}{Devri} kinniget
gant \href{/wiki/Kervella_1962}{Kervella 1962}. Etre an div opsion-mañ e
hañval skañvoc'h \emph{doareer
perc'hennañ}, pehini a vo implijet el labour-mañ.

\subsection[Gerioù-mell gallek troet gant doareerien
perc'hennañ e
brezhoneg]{\texorpdfstring{\protect\hypertarget{Gerio.C3.B9-mell_gallek_troet_gant_doareerien_perc.27henna.C3.B1_e_brezhoneg}{}{}Gerioù-mell
gallek troet gant doareerien perc'hennañ e
brezhoneg}{Gerioù-mell gallek troet gant doareerien perc'hennañ e brezhoneg}}\label{Geriouxf9-mell_gallek_troet_gant_doareerien_percux27hennauxf1_e_brezhoneg}

{{{[}}\href{/wiki/Doareerien_perc\%27henna\%C3\%B1?action=edit&section=2}{{modifier}}{{]}}}

Ar gerioù-perc'hennañ a implijer alies e brezhoneg pa
vez implijet kentoc'h ur ger-mell e galleg evit diskouez
ar berc'henniezh:

Sur le continent, des maisons humaines, des fermes qui se disent
pauvres, mais où la lande étincelle dans les cheminées plus belle
qu'à la floraison de Pâques;

Queffélec 1970, 5

War an douar bras, tiez tud, atantoù o tiskouez paourentez pa sked avat
al lann en o oaledoù, kaeroc'h eget bleuniadur Pask;

Beyer 2016, 7

A la fin de~la~troisième année, c'est Anne Le Berre qui
sort lever des lignes à cent mètres du port, derrière un rocher
tranquille.

Queffélec 1970, 6

E dibenn~e~drede bloavezh, sed aze Ann ar Berr o vont da sevel linennoù,
kant metrad diouzh ar porzh, a-dreñv ur garreg habask.

Beyer 2016, 8

(An trede bloavezh ma oa ar person war an enez)

Il leva la main pour s'essuyer le front~; au lieu de
s'essuyer le front, il se signa.

Queffélec 1970, 57

Sevel a reas e zorn da sec'hiñ e dal;
e-lec'h sec'hiñ e dal e reas sin ar
groaz.

Beyer 2016, 37

Er skouer da heul e voe miret stumm unan an anv gant ar ger-mell
(\emph{la tête}) e galleg evit komz eus ur strollad tud.

Ils s'assoient sur des mottes de terre un peu hautes et,
levant la tête, la tournent de droite à gauche et de gauche à droite
comme s'ils voulaient la dévisser.

Queffélec 1970, 6

Koazezañ a reont war voudennoù uhelik ha, savet o fennoù, e troont
anezho a gleiz da zehoù hag a zehoù da gleiz e-giz pa glaskjent o
diviñsañ.

Beyer 2016, 8

E troidigezh vrezhonek ar frazenn resis-mañ e voe lakaet an anv-kadarn
el liester, diaraoget gant un doareer perc'hennañ
(\emph{o fennoù}). Koulskoude e kaver skouerioù lec'h
m'eo dalc'het ar stumm unan e
degouezhioù a-seurt-se, evel er frazenn \emph{Dudiusoc'h eo
da'm skouarn klevet al labous o kanañ gae war beg e
skourr, eget youc'herezh paotred kêr, savet ar banne d'o
fenn} \href{https://niverel.brezhoneg.bzh/br/meurgorf/30821}{Meurgorf}.

Muioc'h a skouerioù a ranker kavout evit resisaat ar
patrom implijet aliesoc'h.

\subsection[Anvioù noaz gallek troet gant doareerien
perc'hennañ e
brezhoneg]{\texorpdfstring{\protect\hypertarget{Anvio.C3.B9_noaz_gallek_troet_gant_doareerien_perc.27henna.C3.B1_e_brezhoneg}{}{}Anvioù
noaz gallek troet gant doareerien perc'hennañ e
brezhoneg}{Anvioù noaz gallek troet gant doareerien perc'hennañ e brezhoneg}}\label{Anviouxf9_noaz_gallek_troet_gant_doareerien_percux27hennauxf1_e_brezhoneg}

{{{[}}\href{/wiki/Doareerien_perc\%27henna\%C3\%B1?action=edit&section=3}{{modifier}}{{]}}}

Pa implijer un
\href{https://arbres.iker.cnrs.fr/index.php?title=Noms_nus\#Terminologie}{anv
noaz} e galleg e c'haller ouzhpennañ un anv-gwañ
perc'hennañ en droidigezh vrezhonek:

les petites fenêtres~qui emprisonnent~dans l'air le
froid, la pluie et le vent, et les obligent, pour réclamer pitance, à ne
passer qu'un bras maigre qui tâtonne;

Queffélec 1970, 5

ar prenestri bihan~hag a zalc'h~bac'het
en aer~ar yenien, ar glav hag an avel, ken ne c'hellont
nemet tremen ur vrec'h dreut war dastorn evit goulenn o
zamm kreun;

Beyer 2016, 7

\subsection[Doareerien perc'hennañ implijet e brezhoneg
pa gomzer diwar-benn saviadoù ar
c'horf]{\texorpdfstring{\protect\hypertarget{Doareerien_perc.27henna.C3.B1_implijet_e_brezhoneg_pa_gomzer_diwar-benn_saviado.C3.B9_ar_c.27horf}{}{}Doareerien
perc'hennañ implijet e brezhoneg pa gomzer diwar-benn
saviadoù ar
c'horf}{Doareerien perc'hennañ implijet e brezhoneg pa gomzer diwar-benn saviadoù ar c'horf}}\label{Doareerien_percux27hennauxf1_implijet_e_brezhoneg_pa_gomzer_diwar-benn_saviadouxf9_ar_cux27horf}

{{{[}}\href{/wiki/Doareerien_perc\%27henna\%C3\%B1?action=edit&section=4}{{modifier}}{{]}}}

Elle se retournait, déjà les gens qui stationnaient dans le bas de
l'église passaient dans le cimetière.

Queffélec 1970, 44

Treiñ a rae he c'hein ha kentizh e kroge an dud bet e
traoñ an iliz da dremen er vered.

Beyer 2016, 24

Trois hommes, assis sur des escabeaux, lui firent place.

Queffélec 1970, 48

Tri gwaz en o c'hoazez war skabelloù a reas plas dezhañ.

Beyer 2016, 29

Le dimanche, de très bonne heure, les gens se réunissaient dans
l'église, les hommes à droite, les femmes à gauche, tous
debout ou à genoux {[}...{]}.

Queffélec 1970, 43

Da Sul vintin, abred-tre, en em dolpe an dud en iliz, ar wazed a-zehoù,
ar maouezed a-gleiz, an holl en o sav pe war o daoulin {[}...{]}.

Beyer 2016, 23

Thomas, à genoux, ne murmurait aucun mot de prière.

Queffélec 1970, 52

Ne oa ket Tomaz o vouslavarout gerioù pediñ war e zaoulin.

Beyer 2016, 32
