\chapter{Lostgerioù}
\section{Digoradur}\label{Digoradur}

{{{[}}\href{/wiki/Lostgerio\%C3\%B9?action=edit&section=1}{{modifier}}{{]}}}

Hervez \href{/wiki/Kervella_1995}{Kervella (1995, 425)}, ur yezh a zo
enni un niver gerioù-diazez, gwriziennoù ger pe c'hoazh
pennrannoù, hag un niver kalz brasoc'h a
c'herioù savet dizar ar-se. Renabliñ a ra doareoù
disheñvel da stummañ gerioù e brezhoneg, en o zouez:

\begin{itemize}
\tightlist
\item
  Dre zeveradur: gerioù nevez (verboù, anvioù, anvioù-gwan hag all) a
  c'haller sevel en ur stagañ dibennoù-ger ouzh ar
  benn-rann;
\item
  Dre astenn talvoudegezh ar gerioù a zo anezho dija;
\item
  Dre gevrennadur.
\end{itemize}

\subsection[Lostger
\emph{ad/iad}]{\texorpdfstring{\protect\hypertarget{Lostger_ad.2Fiad}{}{}Lostger
\emph{ad/iad}}{Lostger ad/iad}}\label{Lostger_adux2fiad}

{{{[}}\href{/wiki/Lostgerio\%C3\%B9?action=edit&section=2}{{modifier}}{{]}}}

\subsubsection[Pouezañ war hirder ur prantad
amzer]{\texorpdfstring{\protect\hypertarget{Poueza.C3.B1_war_hirder_ur_prantad_amzer}{}{}Pouezañ
war hirder ur prantad
amzer}{Pouezañ war hirder ur prantad amzer}}\label{Pouezauxf1_war_hirder_ur_prantad_amzer}

{{{[}}\href{/wiki/Lostgerio\%C3\%B9?action=edit&section=3}{{modifier}}{{]}}}

Al lost-ger-mañ a c'haller stagañ ouzh un niver evit
pouezañ war hirder ur prantad amzer:

On ne devait pas oublier, disait-il, que, pendant dix années,
l'ile avait porté son prêtre.

Queffélec 1970, 40

Arabat e oa ankouaat, emezañ, he doa an enez bet e veleg dek
vloavezh-pad.

Beyer 2016, 20

Tout le long du dimanche, des îliens défilèrent chez les Gourvennec.

Queffélec 1970, 58

Ar sulvezhiad a-bezh e voe un dibunadeg enezourien e ti Gourvenneg.

Beyer 2016, 38

Anvioù-kadarn a c'hell bezañ
\href{/wiki/Deveradurezh}{deveret} diwar niveroù en ur implijout al
lostger-mañ. \emph{Un dekvloavezhiad} er skouer lakaet amañ dindan a
glot gant \emph{dix années} e galleg ha neket \emph{dix ans}, peogwir e
pouezer war ar fed eo tremenet ar prantad-se pe war an darvoudoù
c'hoarvezet d'ar mare-se
kentoc'h eget merkañ un niver a vloavezhioù hepken:

Dix années.

Queffélec 1970, 5

Un dekvloavezhiad.

Beyer 2016, 7

Un tu all da bouezañ war hirded ur prantad amzer eo ouzhpennañ an adverb
pouezañ \emph{-pad}:

On ne devait pas oublier, disait-il, que, pendant dix années,
l'ile avait porté son prêtre.

Queffélec 1970, 40

Arabat e oa ankouaat, emezañ, he doa an enez bet e veleg dek
vloavezh-pad.

Beyer 2016, 20

\subsubsection[Merkañ an
dalc'had]{\texorpdfstring{\protect\hypertarget{Merka.C3.B1_an_dalc.27had}{}{}Merkañ
an
dalc'had}{Merkañ an dalc'had}}\label{Merkauxf1_an_dalcux27had}

{{{[}}\href{/wiki/Lostgerio\%C3\%B9?action=edit&section=4}{{modifier}}{{]}}}

Posupl eo ouzhpennañ al lostger \emph{-ad}/\emph{-iad}
d'un anv e brezhoneg evit merkañ e
zalc'had. Bez ez eus un diforc'h etre
\emph{ul loa arc'hant} hag \emph{ul loiad
arc'hant}; \emph{ul loa} a verk an danvez, hag \emph{ul
loiad} --- ar pezh a zo e-barzh.

E degouezhioù zo e c'haller merkañ an
diforc'h etre ul lestr goulo pe leun gant al
lostger-mañ, pe pouezañ war ar pezh a zo e-barzh:

Des épaves arrivèrent à la côte~: des rames, des morceaux de coque, un
canot en bon état, des tonneaux d'eau potable, un baril
de vin rouge et trois cadavres -- dont à la rigueur on se fût passé.

Queffélec 1970, 44

Dont a reas peñse d'an aod: roeñvoù, tammoù
kouc'h, ur c'hanod e ratre vat,
tonelladoù dour mat, ur varilhad gwin ruz ha tri
c'horf-marv --- dreistezhomm e oa ar re-mañ pa soñjed
ervat.

Beyer 2016, 24

En nebeud degouezhioù e c'haller merkañ an
dalc'had e gallek gant \emph{ée}, evel \emph{cuillerée},
\emph{bouchée} hag all. Koulskoude, n'eo ket reoliek ar
sistem-se e galleg: da skouer, ne vez ket lavaret \emph{une verrée} met
\emph{un verre de}. A-wechoù e vez degaset gant \emph{ée} ur cheñchamant
ster: ar ger \emph{nuitée} ne dalvez ket mui padelezh an noz, met
kentoc'h kementad a nozvezhioù tremenet en ul leti
(\href{https://www.cnrtl.fr/definition/Nuit\%C3\%A9e}{Geriadur CNRTL}).

Un doare all da verkañ an dalc'had e brezhoneg a zo
displeget
\href{/wiki/Araogenno\%C3\%B9\#Merkañ_dalc'had_gant_an_araogenn_en}{amañ}.

\subsubsection{Eztaoler ul ledanvad}\label{Eztaoler_ul_ledanvad}

{{{[}}\href{/wiki/Lostgerio\%C3\%B9?action=edit&section=5}{{modifier}}{{]}}}

En ur heuliañ ar memes patrom e c'hall \emph{-ad/-iad}
treiñ ul \href{/wiki/Ledanvad}{ledanvad} gallek.

Da skouer e komzer amañ diwar-benn an dud e-barzh an iliz, ha neket ar
savadur:

D'un seul et même courant, l'église se
vidait.

Queffélec 1970, 44

Gant ur froud nemetken e tileunie an ilizad.

Beyer 2016, 24

Hag en degouezh-mañ ez eus un hollekadur pa reer kaoz eus strollad an
dud a zo perzh d'ar barrez:

La paroisse l'écoutait sans surprise, avec une attention
fidèle.

Queffélec 1970, 46

E selaou a rae ar barreziad hep souezhiñ, gant un evezh feal.

Beyer 2016, 26

\subsection{\texorpdfstring{Al lostger
\emph{at}}{Al lostger at}}\label{Al_lostger_at}

{{{[}}\href{/wiki/Lostgerio\%C3\%B9?action=edit&section=6}{{modifier}}{{]}}}

Al lostger \emph{at} a c'hall servijout evit lakaat un
anv-gwan da vezañ ur ger estlammañ:

Montrer aux îliens de quelle faveur Dieu les entourait, belle chose,
mais difficile.

Queffélec 1970, 57

Diskouez d'an enezourien pegen mat e oant erru gant
Doue? Bravat tra, diaesat tra avat.

Beyer 2016, 37
