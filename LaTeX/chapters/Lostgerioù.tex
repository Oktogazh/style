\chapter{Lostgerioù}

\section{Digoradur}\label{digoradur}

Hervez \href{Kervella_1995}{Kervella (1995, 425)}, ur yezh a zo enni un
niver gerioù-diazez, gwriziennoù ger pe c\textquotesingle hoazh
pennrannoù, hag un niver kalz brasoc\textquotesingle h a
c\textquotesingle herioù savet dizar ar-se. Renabliñ a ra doareoù
disheñvel da stummañ gerioù e brezhoneg, en o zouez:

\begin{itemize}
\tightlist
\item
  Dre zeveradur: gerioù nevez (verboù, anvioù, anvioù-gwan hag all) a
  c\textquotesingle haller sevel en ur stagañ dibennoù-ger ouzh ar
  benn-rann;
\item
  Dre astenn talvoudegezh ar gerioù a zo anezho dija;
\item
  Dre gevrennadur.
\end{itemize}

\section{\texorpdfstring{Lostger
\emph{ad/iad}}{Lostger ad/iad}}\label{lostger_adiad}

\subsection{Pouezañ war hirder ur prantad
amzer}\label{pouezauxf1_war_hirder_ur_prantad_amzer}

Al lost-ger-mañ a c\textquotesingle haller stagañ ouzh un niver evit
pouezañ war hirder ur prantad amzer:

\begin{longtable}[]{|p{0.45\textwidth}|p{0.45\textwidth}}

On ne devait pas oublier, disait-il, que, pendant dix années,
l\textquotesingle ile avait porté son prêtre. & Arabat e oa ankouaat,
emezañ, he doa an enez bet e veleg dek vloavezh-pad. \\
Queffélec 1970, 40 & Beyer 2016, 20 \\
\end{longtable}

\begin{longtable}[]{|p{0.45\textwidth}|p{0.45\textwidth}}

Tout le long du dimanche, des îliens défilèrent chez les Gourvennec. &
Ar sulvezhiad a-bezh e voe un dibunadeg enezourien e ti Gourvenneg. \\
Queffélec 1970, 58 & Beyer 2016, 38 \\
\end{longtable}

Anvioù-kadarn a c\textquotesingle hell bezañ
\href{Deveradurezh}{deveret} diwar niveroù en ur implijout al
lostger-mañ. \emph{Un dekvloavezhiad} er skouer lakaet amañ dindan a
glot gant \emph{dix années} e galleg ha neket \emph{dix ans}, peogwir e
pouezer war ar fed eo tremenet ar prantad-se pe war an darvoudoù
c\textquotesingle hoarvezet d\textquotesingle ar mare-se
kentoc\textquotesingle h eget merkañ un niver a vloavezhioù hepken:

\begin{longtable}[]{|p{0.45\textwidth}|p{0.45\textwidth}}

Dix années. & Un dekvloavezhiad. \\
Queffélec 1970, 5 & Beyer 2016, 7 \\
\end{longtable}

Un tu all da bouezañ war hirded ur prantad amzer eo ouzhpennañ an adverb
pouezañ \emph{-pad}:

\begin{longtable}[]{|p{0.45\textwidth}|p{0.45\textwidth}}

On ne devait pas oublier, disait-il, que, pendant dix années,
l\textquotesingle ile avait porté son prêtre. & Arabat e oa ankouaat,
emezañ, he doa an enez bet e veleg dek vloavezh-pad. \\
Queffélec 1970, 40 & Beyer 2016, 20 \\
\end{longtable}

\subsection{Merkañ an
dalc\textquotesingle had}\label{merkauxf1_an_dalchad}

Posupl eo ouzhpennañ al lostger \emph{-ad}/\emph{-iad}
d\textquotesingle un anv e brezhoneg evit merkañ e
zalc\textquotesingle had. Bez ez eus un diforc\textquotesingle h etre
\emph{ul loa arc\textquotesingle hant} hag \emph{ul loiad
arc\textquotesingle hant}; \emph{ul loa} a verk an danvez, hag \emph{ul
loiad} --- ar pezh a zo e-barzh.

E degouezhioù zo e c\textquotesingle haller merkañ an
diforc\textquotesingle h etre ul lestr goulo pe leun gant al
lostger-mañ, pe pouezañ war ar pezh a zo e-barzh:

\begin{longtable}[]{|p{0.45\textwidth}|p{0.45\textwidth}}

Des épaves arrivèrent à la côte~: des rames, des morceaux de coque, un
canot en bon état, des tonneaux d\textquotesingle eau potable, un baril
de vin rouge et trois cadavres -- dont à la rigueur on se fût passé. &
Dont a reas peñse d\textquotesingle an aod: roeñvoù, tammoù
kouc\textquotesingle h, ur c\textquotesingle hanod e ratre vat,
tonelladoù dour mat, ur varilhad gwin ruz ha tri
c\textquotesingle horf-marv --- dreistezhomm e oa ar re-mañ pa soñjed
ervat. \\
Queffélec 1970, 44 & Beyer 2016, 24 \\
\end{longtable}

En nebeud degouezhioù e c\textquotesingle haller merkañ an
dalc\textquotesingle had e gallek gant \emph{ée}, evel \emph{cuillerée},
\emph{bouchée} hag all. Koulskoude, n\textquotesingle eo ket reoliek ar
sistem-se e galleg: da skouer, ne vez ket lavaret \emph{une verrée} met
\emph{un verre de}. A-wechoù e vez degaset gant \emph{ée} ur cheñchamant
ster: ar ger \emph{nuitée} ne dalvez ket mui padelezh an noz, met
kentoc\textquotesingle h kementad a nozvezhioù tremenet en ul leti
(\href{https://www.cnrtl.fr/definition/Nuit\%C3\%A9e}{Geriadur CNRTL}).

Un doare all da verkañ an dalc\textquotesingle had e brezhoneg a zo
displeget \href{Araogennoù\#Merkañ_dalc'had_gant_an_araogenn_en}{amañ}.

\subsection{Eztaoler ul ledanvad}\label{eztaoler_ul_ledanvad}

En ur heuliañ ar memes patrom e c\textquotesingle hall \emph{-ad/-iad}
treiñ ul \url{ledanvad} gallek.

Da skouer e komzer amañ diwar-benn an dud e-barzh an iliz, ha neket ar
savadur:

\begin{longtable}[]{|p{0.45\textwidth}|p{0.45\textwidth}}

D\textquotesingle un seul et même courant, l\textquotesingle église se
vidait. & Gant ur froud nemetken e tileunie an ilizad. \\
Queffélec 1970, 44 & Beyer 2016, 24 \\
\end{longtable}

Hag en degouezh-mañ ez eus un hollekadur pa reer kaoz eus strollad an
dud a zo perzh d\textquotesingle ar barrez:

\begin{longtable}[]{|p{0.45\textwidth}|p{0.45\textwidth}}

La paroisse l\textquotesingle écoutait sans surprise, avec une attention
fidèle. & E selaou a rae ar barreziad hep souezhiñ, gant un evezh
feal. \\
Queffélec 1970, 46 & Beyer 2016, 26 \\
\end{longtable}

\section{\texorpdfstring{Al lostger
\emph{at}}{Al lostger at}}\label{al_lostger_at}

Al lostger \emph{at} a c\textquotesingle hall servijout evit lakaat un
anv-gwan da vezañ ur ger estlammañ:

\begin{longtable}[]{|p{0.45\textwidth}|p{0.45\textwidth}}

Montrer aux îliens de quelle faveur Dieu les entourait, belle chose,
mais difficile. & Diskouez d\textquotesingle an enezourien pegen mat e
oant erru gant Doue? Bravat tra, diaesat tra avat. \\
Queffélec 1970, 57 & Beyer 2016, 37 \\
\end{longtable}
