\chapter{Treuzkas mennozh an dalc'h e brezhoneg}

\section{\texorpdfstring{Mennozh an dalc\textquotesingle h treuzkaset
gant
\emph{e-kerz}}{Mennozh an dalc\textquotesingle h treuzkaset gant e-kerz}}\label{mennozh_an_dalch_treuzkaset_gant_e_kerz}

Mennozh an dalc\textquotesingle h a c\textquotesingle haller diskouez
gant an \url{araogenn} \emph{kerz}, dreist-holl pa gomzer eus traoù
difetis.

\begin{longtable}[]{|p{0.45\textwidth}|p{0.45\textwidth}}

Il disposait d\textquotesingle un pouvoir de prêtre. & Galloud ur beleg
a oa en e gerz. \\
Queffélec 1970, 57 & Beyer 2016, 37 \\
\end{longtable}

\begin{longtable}[]{|p{0.45\textwidth}|p{0.45\textwidth}}

Penché sur l\textquotesingle autel, ne disposant plus, pour son oraison
intérieure, de phrases latines, il se trouvait seul. & Stouet war an
aoter ne oa ket mui frazennoù latin en e gerz evit e bedadenn ziabarzh
hag e-unan en em sante. \\
Queffélec 1970, 212 & Beyer 2016, 192 \\
\end{longtable}

\section{Un anv-gwan doareañ gallek troet e brezhoneg gant ur stumm
perc\textquotesingle hennañ}\label{un_anv_gwan_doareauxf1_gallek_troet_e_brezhoneg_gant_ur_stumm_perchennauxf1}

\begin{longtable}[]{|p{0.45\textwidth}|p{0.45\textwidth}}

Sur le continent, des maisons humaines, des fermes qui se disent
pauvres, mais où la lande étincelle dans les cheminées plus belle
qu\textquotesingle à la floraison de Pâques; & War an douar bras, tiez
tud, atantoù o tiskouez paourentez pa sked avat al lann en o oaledoù,
kaeroc\textquotesingle h eget bleuniadur Pask; \\
Queffélec 1970, 5 & Beyer 2016, 7 \\
\end{longtable}

\section{\texorpdfstring{Ar stumm \emph{gant +
bezañ}}{Ar stumm gant + bezañ}}\label{ar_stumm_gant_bezauxf1}

Implijout a c\textquotesingle haller an araogenn \emph{gant} gant ar
verb \emph{bezañ} evit diskleriañ an dalc\textquotesingle h war un dra,
fetis pe difetis.

\begin{longtable}[]{|p{0.45\textwidth}|p{0.45\textwidth}}

Sans atteindre à la gloire du Mont-Saint-Michel {[}\ldots{]},
l\textquotesingle île aurait la foi d\textquotesingle un monastère. &
Hep tizhout gloar Menez-Mikael {[}\ldots{]} e vefe gant an enez feiz ur
manati. \\
Queffélec 1970, 29 & Beyer 2016, 12 \\
\end{longtable}
