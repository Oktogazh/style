\chapter{Treuzkas mennozh an dalc'h e brezhoneg}
\subsection[Mennozh an dalc'h treuzkaset gant
\emph{e-kerz}]{\texorpdfstring{\protect\hypertarget{Mennozh_an_dalc.27h_treuzkaset_gant_e-kerz}{}{}Mennozh
an dalc'h treuzkaset gant
\emph{e-kerz}}{Mennozh an dalc'h treuzkaset gant e-kerz}}\label{Mennozh_an_dalcux27h_treuzkaset_gant_e-kerz}

{{{[}}\href{/wiki/Treuzkas_mennozh_an_dalc\%27h_e_brezhoneg?action=edit&section=1}{{modifier}}{{]}}}

Mennozh an dalc'h a c'haller diskouez
gant an \href{/wiki/Araogenn}{araogenn} \emph{kerz}, dreist-holl pa
gomzer eus traoù difetis.

Il disposait d'un pouvoir de prêtre.

Queffélec 1970, 57

Galloud ur beleg a oa en e gerz.

Beyer 2016, 37

Penché sur l'autel, ne disposant plus, pour son oraison
intérieure, de phrases latines, il se trouvait seul.

Queffélec 1970, 212

Stouet war an aoter ne oa ket mui frazennoù latin en e gerz evit e
bedadenn ziabarzh hag e-unan en em sante.

Beyer 2016, 192

\subsection[Un anv-gwan doareañ gallek troet e brezhoneg gant ur stumm
perc'hennañ]{\texorpdfstring{\protect\hypertarget{Un_anv-gwan_doarea.C3.B1_gallek_troet_e_brezhoneg_gant_ur_stumm_perc.27henna.C3.B1}{}{}Un
anv-gwan doareañ gallek troet e brezhoneg gant ur stumm
perc'hennañ}{Un anv-gwan doareañ gallek troet e brezhoneg gant ur stumm perc'hennañ}}\label{Un_anv-gwan_doareauxf1_gallek_troet_e_brezhoneg_gant_ur_stumm_percux27hennauxf1}

{{{[}}\href{/wiki/Treuzkas_mennozh_an_dalc\%27h_e_brezhoneg?action=edit&section=2}{{modifier}}{{]}}}

Sur le continent, des maisons humaines, des fermes qui se disent
pauvres, mais où la lande étincelle dans les cheminées plus belle
qu'à la floraison de Pâques;

Queffélec 1970, 5

War an douar bras, tiez tud, atantoù o tiskouez paourentez pa sked avat
al lann en o oaledoù, kaeroc'h eget bleuniadur Pask;

Beyer 2016, 7

\subsection[Ar stumm \emph{gant +
bezañ}]{\texorpdfstring{\protect\hypertarget{Ar_stumm_gant_.2B_beza.C3.B1}{}{}Ar
stumm \emph{gant +
bezañ}}{Ar stumm gant + bezañ}}\label{Ar_stumm_gant_+_bezauxf1}

{{{[}}\href{/wiki/Treuzkas_mennozh_an_dalc\%27h_e_brezhoneg?action=edit&section=3}{{modifier}}{{]}}}

Implijout a c'haller an araogenn \emph{gant} gant ar
verb \emph{bezañ} evit diskleriañ an dalc'h war un dra,
fetis pe difetis.

Sans atteindre à la gloire du Mont-Saint-Michel {[}\ldots{]},
l'île aurait la foi d'un monastère.

Queffélec 1970, 29

Hep tizhout gloar Menez-Mikael {[}\ldots{]} e vefe gant an enez feiz ur
manati.

Beyer 2016, 12
