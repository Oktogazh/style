\chapter{Trugarekadennoù}

A wir galon em eus c\textquotesingle hoant trugarekaat an dud o deus
sikouret ac\textquotesingle hanon da gas an enklask-mañ
d\textquotesingle ar poent m\textquotesingle emañ hiziv, hepto ne vije
ket bet posupl morse al labour-mañ:

Mélanie Jouitteau hag Erwan Hupel, ma c\textquotesingle henrenerien
kounskrid. E-kreiz ar raktres-mañ e oa Mélanie atav, o sturiañ hag o
kalonekaat ac\textquotesingle hanon gant pasianted pa raen ma fazennoù
kentañ e bed an enklask, atav o kinnig alioù talvoudus hag he
skiant-prenet divevenn. Erwan, unan eus ma c\textquotesingle hentañ
kelennerien war ar brezhoneg, a zo kiriek evit ma
c\textquotesingle hallfen seveniñ ur seurt labour e brezhoneg e
penn-kentañ. Reizh met kalonekaus e voe e holl evezhiadennoù, ha
pouezus-kenañ e voe evidon e vadelezh hag e basianted a-hed ar raktres.
Trugarez vras a fell din lavaret dezho o-daou evit o labour sturiañ, o
skoazell hag o fiziañs didorr.

Myriam Guillevic, hag a oa ivez e-touez ma c\textquotesingle hentañ
kelennerien brezhoneg. Brokus, e roas hec\textquotesingle h amzer hag e
savpoent arbennik, koulz evel enklaskerez ha kelennerez evit souten ar
c\textquotesingle hounskrid-mañ.

Kevin Rottet, unan eus kenaozerion
\href{Rottet_&_Moris_2018}{\emph{Comparative Stylistics of Welsh and
English}}, al levr en deus heñchet ac\textquotesingle hanon a-hed an
enklask-mañ, evit bezañ rannet e alioù hag e skiant-prenet prizius ganin
e deroù ma labour.

Mich Beyer ha Pierrette Kermoal, evit bezañ pourvezet stumm niverel
levrioù studiet amañ.

Yves Drezen, evit e fiziañs em implij oberenn e dad.

Bernez Rouz, evit e sikour hag e alioù prizius.

Nicolas Vigneron ha Loic Grobol, evit o sikour teknikel gant ar wiki.

Ma amezegez Beatriz Jouin, evit bezañ prestet din levrioù eus he
levraoueg.

Myrzinn Jaouen, evit bezañ aliet din mont e darempred gant Mélanie, un
emgav a gasas war-eeun da savidigezh ar raktres-mañ.

Alan Kersaudy, pa roas kalon ha skoazell din a-hed ma enklask, ma
talc\textquotesingle hfen da vont pa felle din dilezel, gant respontoù
d\textquotesingle am goulennoù diniver ha barregezhioù emaozañ ha
teknikel, ha peurgetket, evit ober war-dro ma c\textquotesingle hazh
e-pad ma bloavezh diwezhañ a studioù en Aberystwyth.

Da ziwezhañ, ha neket da zisterañ, Vadym Kolesnichenko, ma zad, eñ eo an
den am broudas da studiañ tramor hag heptañ ne vije bet na tu, na poell,
na talvoudegezh e kement tra am eus gallet seveniñ.
