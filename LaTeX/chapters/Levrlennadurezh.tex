\chapter{Levrlennadurezh}
Bally, Charles. 1951. \emph{Traité de Stylistique Française}. 3ème
édition. Vol. I. II vols. Editions Klincksieck, Paris.
\href{https://gallica.bnf.fr/ark:/12148/bpt6k273942/f22.item}{Testenn}
\href{/wiki/Bally_1951}{Bally 1951}

Beyer, Mich. 2016. \emph{Ur person evit enez Sun}, An
Alarc'h, Lannion a-ziwar Queffélec, Henri. 1970.
\emph{Un recteur de l'île de Sein}, Editions Stock,
Delamain et Boutelleau, Paris. \href{/wiki/Beyer_2016}{Beyer 2016}
a-ziwar \href{/wiki/Queff\%C3\%A9lec_1970}{Queffélec 1970}

Cressot, Marcel. 1959. \emph{Le style et ses techniques}. 4e ed. Presses
Universitaires de France, Paris. \href{/wiki/Cressot_1959}{Cressot 1959}

Darbelnet, Jean \& Vinay, Jean-Paul. 1993. \emph{Stylistique comparée du
français et de l'anglais}, Didier, Paris.
\href{/wiki/Darbelnet_\%26_Vinay_1993}{Darbelnet \& Vinay 1993}

Drezen, Youenn. 1943. \emph{Notre-Dame Bigoudenn}, Denoël, Paris a-ziwar
Drezen, Youenn. 2012. \emph{Itron Varia Garmez}, Aber, Landéda.
\href{/wiki/Drezen_1943}{Drezen 1943} a-ziwar
\href{/wiki/Drezen_2012}{Drezen 2012}

Even, Arzel. 1987. \emph{Istor ar Yezhoù Keltiek I}. Hor Yezh, Lesneven.
\href{/wiki/Even_1987}{Even 1987}

Fave, Visant. 1998. \emph{Notennou yezadur}, Emgleo Breiz, Brest.
\href{/wiki/Fave_1998}{Fave 1998}

Favereau, Francis. 1993. \emph{Yezhadur ar brezhoneg a-vremañ}, Skol
Vreizh, Morlaix. \href{/wiki/Favereau_1993}{Favereau 1993}

Gourmelon, Yvon. 2012. 'Div araogenn dirak ar memes
anv\textquotesingle, Notennoù yezhadur, Al Liamm (éd.), 63-66. {[}rééd.
2008. Al Liamm 368, 81-84{]}. \href{/wiki/Gourmelon_2012}{Gourmelon
2012}

Gros, Jules. 1970b. Le trésor du breton parlé II (Eléments de
Stylistique Trégorroise). Dictionnaire breton-français des expressions
figurées, Librairie Bretonne Giraudon.
\href{/wiki/Gros_1970b?action=edit&redlink=1}{Gros 1970b}

Guillemin-Flescher, Jacqueline. 1981. \emph{Syntaxe comparée du français
et de l'anglais}. Problèmes de traduction. Éditions
Ophrys. Paris. \href{/wiki/Guillemin-Flescher_1981}{Guillemin-Flescher
1981}

Hacquard, Valentine. 2009. 58. Modality. In: Heusinger, K., Maienborn,
C. and Portner, P. ed. Volume 2. Berlin, Boston: De Gruyter Mouton, pp.
1484-1515. Testenn \href{/wiki/Hacquard_2009}{Hacquard 2009}

Jouitteau, Mélanie. (éd.). 2007-2025. ARBRES, wikigrammaire des
dialectes du breton et centre de ressources pour son étude linguistique
formelle, IKER, CNRS, \url{http://arbres.iker.cnrs.fr}. Licence Creative
Commons BY-NC-SA \href{/wiki/Jouitteau_2009-2024}{Jouitteau 2009-2024}

Jouitteau, Mélanie. 2023d. 'Guide de survie des langues
minorisées à l'heure de l'intelligence
artificielle~: Appel aux communautés parlantes\textquotesingle,
Lapurdum, numéro spécial 6, Texte.
\href{/wiki/Jouitteau_2023d}{Jouitteau 2023d}

Jouitteau, Mélanie. 2005. \emph{La syntaxe comparée du Breton}, PhD ms,
Université de Nantes. \href{/wiki/Jouitteau_2005}{Jouitteau 2005}

Kreizenn Ar Geriaouiñ. 2016. 'Geriaoueg
yezhadur\textquotesingle, Brezhoneg 21 (éd.),
\href{https://www.brezhoneg21.com/resources/geriaouegou/YEZHADUR.pdf}{Testenn}.
\href{/wiki/KAG_2016}{KAG 2016}

Kergoat, Lukian, Yvon Gourmelon, Francis Favereau \& Martial Ménard.
1989. Yezhadur (skolaj ha lise), TES.
\href{/wiki/Kergoat_\%26_al._(1989)?action=edit&redlink=1}{Kergoat \&
al. (1989)}

Kervella, Frañsez. 1962. \emph{Evezhiadennoù war
c'heariadur Roparz Hemon}. Levrenn gentañ. Skol,
Plouezec. \href{/wiki/Kervella_1962}{Kervella 1962}

Kervella, Frañsez. 1995. \emph{Yezhadur bras ar brezhoneg}. Trede
Mouladur. Al Liamm, Brest. \href{/wiki/Kervella_1995}{Kervella 1995}

Le Berre, Yves. 2011b. 'La force et
l'élégance. Les litotes dans la Passion bretonne de
1530\textquotesingle, Nelly Blanchard, Ronan Calvez, Yves Le Berre,
Daniel Le Bris, Jean Le Dû, Mannaig Thomas (dir.), La Bretagne
Linguistique 16, CRBC, 123-150. texte. \href{/wiki/Le_Berre_2011}{Le
Berre 2011}

Le Clerc, Louis. 1986 {[}1906, 1911{]}, \emph{Grammaire Bretonne du
dialecte de Tréguier}, 3ième édition, Ar Skol Vrezoneg, Emgleo Breiz
(précédentes Saint-Brieuc: Prud'homme).
\href{/wiki/Le_Clerc_1986}{Le Clerc 1986}

Le Dimna, Nicole. 2005. \emph{Palimpsestes franco-bretons.
L'autotraduction de Youenn Drézen. Textes inédits de
Youenn Drézen}. L'Harmattan. Paris.
\href{/wiki/Le_Dimna_2005}{Le Dimna 2005}

Le Gléau, René. 1999. \emph{Études Syntaxiques Bretonnes} (Tome 1), 2e
édition entièrement revue et augmentée. Brest~: R. Le Gléau.
\href{/wiki/Le_Gl\%C3\%A9au_1999}{Le Gléau 1999}

Moseley, Christopher (dir.) \& Nicolas, Alexandre (cartographie) (préf.
Bokova, Irina). 2010. \emph{Atlas des langues en danger dans le monde},
UNESCO, coll. «~Mémoire des peuples~», 3e éd., 230 p.
\href{/wiki/Moseley_2010}{Moseley 2010}

Rezac, Milan. 2013. 'The Breton double subject
construction\textquotesingle, Ali Tifrit (éd.), Phonologie, Morphologie,
Syntaxe Mélanges offerts à Jean-Pierre Angoujard, PUR, 355-379. -
version 2009 avant édition. \href{/wiki/Rezac_2013}{Rezac 2013}

Rottet, Kevin \& Morris, Steve. 2018. \emph{Comparative Stylistics of
Welsh and English: Arddulleg y Gymraeg}. University of Wales Press,
Cardiff. \href{/wiki/Rottet_\%26_Moris_2018}{Rottet \& Moris 2018}

Urien, Jean-Yves. 1987-9. \emph{La trame d'une langue,
Le breton. Présentation d'une théorie de la syntaxe et
application}, Lesneven: Mouladurioù Hor Yezh (première édition 1987).
\href{/wiki/Urien_1987}{Urien 1987}

NewPP limit report Parsed by mw193 Cached time: 20251212222617 Cache
expiry: 1296000 Reduced expiry: false Complications: {[}{]} CPU time
usage: 0.013 seconds Real time usage: 0.014 seconds Preprocessor visited
node count: 1/1000000 Post‐expand include size: 0/2097152 bytes Template
argument size: 0/2097152 bytes Highest expansion depth: 1/100 Expensive
parser function count: 0/99 Unstrip recursion depth: 0/20 Unstrip
post‐expand size: 0/5000000 bytes Transclusion expansion time report
(\%,ms,calls,template) 100.00\% 0.000 1 -total Saved in parser cache
with key stylewiki:pcache:320:\textbar\#\textbar:idhash:canonical and
timestamp 20251212222617 and revision id 2762. Rendering was triggered
because: page-view
