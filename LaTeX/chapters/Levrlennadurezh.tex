\chapter{Levrlennadurezh}

Bally, Charles. 1951. \emph{Traité de Stylistique Française}. 3ème édition. Vol. I. II vols. Editions Klincksieck, Paris. Testenn Bally 1951

Beyer, Mich. 2016. \emph{Ur person evit enez Sun}, An Alarc\textquotesingle h, Lannion a-ziwar Queffélec, Henri. 1970. \emph{Un recteur de l\textquotesingle île de Sein}, Editions Stock, Delamain et Boutelleau, Paris. Beyer 2016 a-ziwar Queffélec 1970

Cressot, Marcel. 1959. \emph{Le style et ses techniques}. 4e ed. Presses Universitaires de France, Paris. Cressot 1959

Darbelnet, Jean \& Vinay, Jean-Paul. 1993. \emph{Stylistique comparée du français et de l\textquotesingle anglais}, Didier, Paris. Darbelnet \& Vinay 1993

Drezen, Youenn. 1943. \emph{Notre-Dame Bigoudenn}, Denoël, Paris a-ziwar Drezen, Youenn. 2012. \emph{Itron Varia Garmez}, Aber, Landéda. Drezen 1943 a-ziwar Drezen 2012

Even, Arzel. 1987. \emph{Istor ar Yezhoù Keltiek I}. Hor Yezh, Lesneven. Even 1987

Fave, Visant. 1998. \emph{Notennou yezadur}, Emgleo Breiz, Brest. Fave 1998

Favereau, Francis. 1993. \emph{Yezhadur ar brezhoneg a-vremañ}, Skol Vreizh, Morlaix. Favereau 1993

Gourmelon, Yvon. 2012. \textquotesingle Div araogenn dirak ar memes anv\textquotesingle, Notennoù yezhadur, Al Liamm (éd.), 63-66. {[}rééd. 2008. Al Liamm 368, 81-84{]}. Gourmelon 2012

Gros, Jules. 1970b. Le trésor du breton parlé II (Eléments de Stylistique Trégorroise). Dictionnaire breton-français des expressions figurées, Librairie Bretonne Giraudon. Gros 1970b

Guillemin-Flescher, Jacqueline. 1981. \emph{Syntaxe comparée du français et de l\textquotesingle anglais}. Problèmes de traduction. Éditions Ophrys. Paris. Guillemin-Flescher 1981

Hacquard, Valentine. 2009. 58. Modality. In: Heusinger, K., Maienborn, C. and Portner, P. ed. Volume 2. Berlin, Boston: De Gruyter Mouton, pp. 1484-1515. Testenn Hacquard 2009

Jouitteau, Mélanie. (éd.). 2007-2025. ARBRES, wikigrammaire des dialectes du breton et centre de ressources pour son étude linguistique formelle, IKER, CNRS, http://arbres.iker.cnrs.fr. Licence Creative Commons BY-NC-SA Jouitteau 2009-2024

Jouitteau, Mélanie. 2023d. \textquotesingle Guide de survie des langues minorisées à l\textquotesingle heure de l\textquotesingle intelligence artificielle : Appel aux communautés parlantes\textquotesingle, Lapurdum, numéro spécial 6, Texte. Jouitteau 2023d

Jouitteau, Mélanie. 2005. \emph{La syntaxe comparée du Breton}, PhD ms, Université de Nantes. Jouitteau 2005

Kreizenn Ar Geriaouiñ. 2016. \textquotesingle Geriaoueg yezhadur\textquotesingle, Brezhoneg 21 (éd.), Testenn. KAG 2016

Kergoat, Lukian, Yvon Gourmelon, Francis Favereau \& Martial Ménard. 1989. Yezhadur (skolaj ha lise), TES. Kergoat \& al. (1989)

Kervella, Frañsez. 1962. \emph{Evezhiadennoù war c\textquotesingle heariadur Roparz Hemon}. Levrenn gentañ. Skol, Plouezec. Kervella 1962

Kervella, Frañsez. 1995. \emph{Yezhadur bras ar brezhoneg}. Trede Mouladur. Al Liamm, Brest. Kervella 1995

Le Berre, Yves. 2011b. \textquotesingle La force et l\textquotesingle élégance. Les litotes dans la Passion bretonne de 1530\textquotesingle, Nelly Blanchard, Ronan Calvez, Yves Le Berre, Daniel Le Bris, Jean Le Dû, Mannaig Thomas (dir.), La Bretagne Linguistique 16, CRBC, 123-150. texte. Le Berre 2011

Le Clerc, Louis. 1986 {[}1906, 1911{]}, \emph{Grammaire Bretonne du dialecte de Tréguier}, 3ième édition, Ar Skol Vrezoneg, Emgleo Breiz (précédentes Saint-Brieuc: Prud\textquotesingle homme). Le Clerc 1986

Le Dimna, Nicole. 2005. \emph{Palimpsestes franco-bretons. L\textquotesingle autotraduction de Youenn Drézen. Textes inédits de Youenn Drézen}. L\textquotesingle Harmattan. Paris. Le Dimna 2005

Le Gléau, René. 1999. \emph{Études Syntaxiques Bretonnes} (Tome 1), 2e édition entièrement revue et augmentée. Brest : R. Le Gléau. Le Gléau 1999

Moseley, Christopher (dir.) \& Nicolas, Alexandre (cartographie) (préf. Bokova, Irina). 2010. \emph{Atlas des langues en danger dans le monde}, UNESCO, coll. « Mémoire des peuples », 3e éd., 230 p. Moseley 2010

Rezac, Milan. 2013. \textquotesingle The Breton double subject construction\textquotesingle, Ali Tifrit (éd.), Phonologie, Morphologie, Syntaxe Mélanges offerts à Jean-Pierre Angoujard, PUR, 355-379. - version 2009 avant édition. Rezac 2013

Rottet, Kevin \& Morris, Steve. 2018. \emph{Comparative Stylistics of Welsh and English: Arddulleg y Gymraeg}. University of Wales Press, Cardiff. Rottet \& Moris 2018

Urien, Jean-Yves. 1987-9. \emph{La trame d\textquotesingle une langue, Le breton. Présentation d\textquotesingle une théorie de la syntaxe et application}, Lesneven: Mouladurioù Hor Yezh (première édition 1987). Urien 1987 