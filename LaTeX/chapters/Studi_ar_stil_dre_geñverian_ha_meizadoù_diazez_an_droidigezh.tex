\chapter{Studi ar stil dre geñverian ha meizadoù diazez an droidigezh}

\section{\texorpdfstring{Kudennoù terminologiezh ar ger
\emph{stil}}{Kudennoù terminologiezh ar ger stil}}\label{kudennouxf9_terminologiezh_ar_ger_stil}

Touellus e c\textquotesingle hall bezañ an termen \emph{stil} e
brezhoneg, dre ma c\textquotesingle hall ober dave da zaou vennozh
disheñvel anavezet e galleg evel \emph{style} ha \emph{stylistique}.
Dre-se eo pouezus resisaat e vo implijet an termen \emph{stil} el
labour-mañ evit komz eus doare skrivañ un aozer resis, ha \emph{studi ar
stil} a roio dave d\textquotesingle an dachenn enklask.

Termenadurioù disheñvel a gaver eus studi ar stil ha bras-kenañ e
c\textquotesingle hall bezañ an diforc\textquotesingle h etrezo a-wechoù
(\href{Bally_1951}{Bally 1951}, \href{Cressot_1959}{Cressot 1959}). Ne
vo ket studiet pizh an termenadurioù liesseurt-se. Lakaet e vo ar pouez
kentoc\textquotesingle h war an hini a zo e diazez ar studiadenn-mañ:

«Studi ar stil a denn ouzh perzhioù ar yezh a reer \emph{opsionoù}
anezho: ar perzhioù a c\textquotesingle hall ar skrivagner pe an troer
dibab diouto» (\href{Rottet_&_Moris_2018}{Rottet \& Moris 2018}, X). Un
diforc\textquotesingle h a zo graet etre ar redioù, perzhioù ar yezh ne
aotreont dibab ebet, hag an opsionoù, perzhioù a
c\textquotesingle haller dibab diouto pe o leuskel a-gostez, diouzh
doare ar skrivagner, derez ar yezh implijet en destenn hag all.

Studi ar stil dre geñveriañ a zo un teknik hag a aotre keñveriañ a-dost
binvioù stil e div yezh evit sellet penaos e oar ober pep hini he mat
eus he skeudennoù stil (pe binvioù yezh) en un doare a dreuzkasfe he
spered idiomatek hag eztaolus. Diazezet eo an teknik-mañ war studi
testennoù gant o zroidigezh. Rummata a reer skouerioù a batromoù
yezhoniel en ur yezh e-kichen troiennoù a weler evel kevatal en o
zroidigezh (\href{Rottet_&_Moris_2018}{Rottet \& Moris 2018}, 1).

\section{Meizadoù diazez an
droidigezh}\label{meizadouxf9_diazez_an_droidigezh}

Mennegiñ a ra \href{Rottet_&_Moris_2018}{Rottet ha Moris (2018, 2)}
meizadoù an droidigezh diazez da heul:

\begin{itemize}
\tightlist
\item
  yezh orin ha yezh tal;
\item
  redioù hag opsionoù;
\item
  unanennoù troidigezh;
\item
  dreistdroidigezh;
\item
  derez ar yezh ha variadurioù diouzh an degouezh.
\end{itemize}

\section{Yezh orin ha yezh tal}\label{yezh_orin_ha_yezh_tal}

M\textquotesingle eo skrivet un destenn e galleg da gentañ ha troet e
brezhoneg da c\textquotesingle houde eo ar galleg ar yezh orin, hag ar
brezhoneg ar yezh tal. Er studiadenn-mañ e vez kinniget ivez skouerioù
tennet eus testennoù m'eo ar brezhoneg ar yezh orin hag ar galleg ar
yezh tal.

Pouezus eo studial oberennoù e daou ster an droidigezh (da skouer,
brezhoneg \textgreater{} galleg ha galleg \textgreater{} brezhoneg) dre
ma c\textquotesingle hall an dielfennadur keñveriet diskouez
diforc\textquotesingle hioù teknik. Un doare da gadarnaat ur patrom
yezhoniel bennak etre div yezh eo gwiriekaat hag-eñ e kaver ar patrom-se
e daou ster an droidigezh (\href{Rottet_&_Moris_2018}{Rottet \& Moris
2018}, 1).

A-hed al labour-mañ e vo roet an holl skoeurioù e kolonennoù gant ar
yezh orin en tu kleiz hag ar yezh tal war an tu dehou.

\section{Redioù hag opsionoù}\label{rediouxf9_hag_opsionouxf9}

Degas a ra pep yezh reolennoù da gemer e kont: da skouer, reizh ar
gerioù, displegadur ar verboù, kemmadurioù. Ar \textbf{redioù}-se a
gaver e degouezhioù ma ne aotre ar yezh da implijout nemet ur stumm
resis. An \textbf{opsionoù}, avat, a zo elfennoù a
c\textquotesingle hall ar skrivagner (pe un troer) dibab diouto, tra ma
touj ouzh ar bevennoù degaset gant yezhadur ar yezh. Ar frankiz-se eo ar
pezh a aotre da grouiñ un doare hiniennel d\textquotesingle en em
ezteurel.

Kemeromp da skouer \emph{imparfait du subjonctif} e galleg. Gwechall e
oa rekis implij ar stumm-se, hag en deiz ha hiriv eo diret
(\href{Darbelnet_&_Vinay_1993}{Darbelnet \& Vinay 1993}, 31). Degas a ra
\emph{imparfait du subjonctif} ar memes talvoudegezh ha \emph{subjonctif
présent}, nemet implijet e vez kentoc\textquotesingle h er yezh skrivet
pe er yezh uhel. Gallout a ra servijout da ezteurel un obererezh diasur
pe diechu. Koulskoude e c\textquotesingle hall kavout diamzeret ar
stumm-se lod al lennerien.

N\textquotesingle eo ket an holl ziforc\textquotesingle hioù etre div
yezh a zo danvez studi ar stil dre geñveriañ. Evel ma lavaras Vinay ha
Darbelnet, «ar redioù a zo tachenn ar yezhadur hag an opsionoù a zo
tachenn studi ar stil» (\href{Rottet_&_Moris_2018}{Rottet \& Moris
2018}, 4). Kefridi an troer eo diforc\textquotesingle hañ etre ar redioù
degaset gant reolennoù ar yezh ha dibaboù libr an aozer. Er yezh orin eo
dreist-holl an dibaboù personel-se a ranker anavezout. Er yezh tal e
rank an troer derc'hel kont eus ar strishadurioù a harz ouzh e frankiz
d\textquotesingle en em ezteurel en ur c\textquotesingle houzout dibab
etre an opsionoù a aotre treuzkas arlivioù an destenn.

Un nebeud diforc\textquotesingle hioù yezhadurel etre ar galleg hag ar
brezhoneg a c\textquotesingle haller gwelet er skouer verr roet amañ
dindan; an darn vrasañ anezho a denn da berzhioù rekis gant yezhadurioù
an div yezh.

Da skouer, \emph{ses pauvres ouailles} a zo troet er skouer-mañ evel
\emph{e zeñved kaezh}. Gant an doareer perc\textquotesingle hennañ
\emph{e} e vez degaset ur c\textquotesingle hemmadur dre vlotaat, ur fed
eus yezhadur ar brezhoneg ne c\textquotesingle hall ket cheñch an troer.
Ne c\textquotesingle haller ket dibab stumm ha
lec\textquotesingle hiadur an anv-gwan \emph{kaezh} kennebeut: lakaet e
vez an darn vrasañ eus an anvioù-gwan brezhonek
war-lerc\textquotesingle h an anv, estreget un nebeud nemedennoù.

An opsinoù eo danvez pennañ studi ar stil. Sellomp ouzh ar skouer
da-heul:

Evit ezteurel ar verb \emph{pleurer} e c'hallje an troer dibab
\emph{leñvañ} pe \emph{gouelañ}, verboù boutin a-walc'h e brezhoneg.
Hervez \href{https://devri.bzh/}{Devri} ez eus un
diforc\textquotesingle h ster etre an daou verb-se~: leñvañ a dalvez
\emph{pleurer en gémissant} ha gouelañ - \emph{verser des larmes sans
crier}. Evit ezteurel ur from kreñvoc\textquotesingle h e
c\textquotesingle halljed implijout \emph{gouelañ dourek}, an droienn a
glot gant \emph{pleurer à chaudes larmes} e galleg. Un droienn
\emph{gouelioù ki} a gaver ivez evit treiñ \emph{larmes de crocodile} en
un degouezh pilpous pe faos. Koulskoude, en degouezh-mañ e tibabas an
droerez \emph{skuilhañ daeloù}: ur stumm a aotre eztaoler verb
\emph{pleurer} neutrel hag amsklaer an destenn orin.

\section{Unanennoù troidigezh}\label{unanennouxf9_troidigezh}

An termen \emph{unanennoù troidigezh} a denn da elfennoù bihanañ al
lavar hag a c\textquotesingle haller treiñ hep o stagañ ouzh un unanenn
droidigezh all. Un unanenn droidigezh a c\textquotesingle hell bezañ ken
bihan hag ur ger, met ivez e c\textquotesingle hall bezañ ur frazenn, un
dro-lavar pe ur strollad gerioù a ya asambles hag e ranker sellet outo
evel un hollad. Pouezus eo treuzkas ster un unanenn droidigezh en he
fezh kentoc\textquotesingle h eget treiñ pep hini eus
hec\textquotesingle h elfennoù. E-touez doareoù unanennoù troidigezh e
kaver unanennoù simpl, troiennoù ha formulennoù.

\subsection{Unanennoù simpl}\label{unanennouxf9_simpl}

En degouezh simplañ, pep ger er yezh orin a glot gant ur ger er yezh
tal:

Er skouer a-us ez eus ar memes niver a c\textquotesingle herioù en
destenn orin hag en droidigezh; aes eo keñveriañ ar gerioù gallek ha
brezhonek ma kemerer e kont an diforc'hioù diazez en urzh ar gerioù.

Koulskoude, a-wechoù n\textquotesingle eo ket heñvel an niver a
c\textquotesingle herioù en testennoù yezh orin ha yezh tal, ha dre-se
ne glot ket ar gerioù en un doare ken treuzwelus. E degouezhioù zo e
c\textquotesingle haller ezteurel un nebeud unanennoù troidigezh gant
daou c\textquotesingle her pe meur a hini er yezh tal.

\section{Dreistdroidigezh}\label{dreistdroidigezh}

Dre vras e c\textquotesingle hoarvez an dreistdroidigezh pa ne zeu ket
a-benn an troer d\textquotesingle anavezout un heuliad liesger evel un
unanenn droidigezh nemetken hag e tro pep elfenn enni war-eeun pa vefe
unanennoù dizalc\textquotesingle h.

Un nebeud skouerioù a zreistdroidigezh a zo renablet amañ-dindan:

\begin{itemize}
\tightlist
\item
  lakaat e anv --- \emph{mettre son nom} e-lec\textquotesingle h
  \emph{s\textquotesingle inscrire}
\item
  \emph{faire attention} --- \emph{ober aked} kentoc\textquotesingle h
  eget \emph{diwall/bezañ aketus}
\item
  lakaat an daol --- \emph{mettre la table} e-lec\textquotesingle h
  \emph{mettre le couvert}
\item
  mont war-raok --- \emph{aller en avant} ha neket \emph{progresser}
\item
  \emph{se rendre compte} --- en em reiñ kont e-lec\textquotesingle h
  merzout
\item
  mont diwar wel --- \emph{aller hors de vue} kentoc\textquotesingle h
  eget \emph{disparaître}
\end{itemize}
