\chapter{Ar fokus}
\subsection[Ar fokus a c'haller merkañ gant urzh ar
gerioù e
brezhoneg]{\texorpdfstring{\protect\hypertarget{Ar_fokus_a_c.27haller_merka.C3.B1_gant_urzh_ar_gerio.C3.B9_e_brezhoneg}{}{}Ar
fokus a c'haller merkañ gant urzh ar gerioù e
brezhoneg}{Ar fokus a c'haller merkañ gant urzh ar gerioù e brezhoneg}}\label{Ar_fokus_a_cux27haller_merkauxf1_gant_urzh_ar_geriouxf9_e_brezhoneg}

{{{[}}\href{/wiki/Ar_fokus?action=edit&section=1}{{modifier}}{{]}}}

Ar fokus a c'haller merkañ gant urzh ar gerioù. Emañ
plas ar fokus e deroù ar frazenn a-raok ur rannig-verb e brezhoneg
peurliesañ
(\href{https://arbres.iker.cnrs.fr/index.php?title=Focus}{Arbres~:
Focus}).

Il doute du témoignage de ses yeux qui virent le printemps sur
l'île et il voudrait croire qu'il a vu
des mirages.

Queffélec 1970, 5

Arvariñ a ra war testeni e zaoulagad hag a welas an nevez amzer war an
enez, ha karout a rafe krediñ bezañ gwelet touelloù.

Beyer 2016, 7

Mais rien ne doit plus, aujourd'hui, le retenir dans
l'île...

Queffélec 1970, 25

Met hiziv ne zlee tra ebet e zerc'hel war an enez...

Beyer 2016, 7

Er frazenn c'hallek emañ ar
\href{/wiki/Fokus_kontrasta\%C3\%B1}{fokus kontrastañ} war
\emph{aujourd'hui}. Merket eo gant ul
\href{/wiki/Lavarenn_gennet?action=edit&redlink=1}{lavarenn gennet}.

E brezhoneg eo bet lakaet \emph{hiziv} e deroù ar frazenn evit merkañ ar
fokus.

Personne, ce jour-là, n'a rencontré Yvinec et, si
l'on demande aux pêcheurs d'où ils
connaissent l'histoire de l'épave, ils
crachent et se dérobent derrière des mots.

Queffélec 1970, 6

En devezh-se ne oa bet den oc'h en em gavout gant Ivineg
met pa vez goulennet ouzh ar besketaerien dre benaos ez anavezont afer
ar peñse e tufont hag e kuzhont a-dreñv ar gerioù.

Beyer 2016, 9

Les îliens, pendant les nuits de tempête, se figurent que les âmes des
noyés pleurent dans le Raz, qu'elles voltigent au-dessus
de la mer, qu'elles montent sur le rivage et heurtent
aux fenêtres.

Queffélec 1970, 7

E-pad an nozvezhioù tourmant e soñj gant an enezourien e vez eneoù ar re
veuzet er raz o ouelañ, e nijellont a-us ar mor, e savont war an aod hag
e skoont ouzh ar prenestri.

Beyer 2016, 9

Il écoutait, les yeux baissés, son supérieur lui décrire son futur
domaine spirituel et il regardait sur le plancher lui apparaître sa
nouvelle paroisse aussi clairement qu'une place publique
du haut d'une tour d'église.

Queffélec 1970, 28

Soublet e zaoulagad e selaoue e vestr o teskrivañ dezhañ e zomani
speredel da zont, ha treset war ar plañchod e wele e barrez nevez o tont
war wel dezhañ, ken sklaer hag ul leur gêr gwelet a-ziwar un tour iliz.

Beyer 2016, 11

Il lui arriva, au crépuscule, d'entrer dans
l'église et de s'y enfermer à clef pour
s'accorder une séance solitaire
d'imitation.

Queffélec 1970, 39

Gant ar serr-noz e c'hoarvezas gantañ mont tre en iliz
ha prennañ warnañ evit aotren dezhañ e-unan un abadenn dreveziñ en
digenvez.

Beyer 2016, 19

\subsection[Fokus degaset gant stummoù
skaret]{\texorpdfstring{\protect\hypertarget{Fokus_degaset_gant_stummo.C3.B9_skaret}{}{}Fokus
degaset gant stummoù
skaret}{Fokus degaset gant stummoù skaret}}\label{Fokus_degaset_gant_stummouxf9_skaret}

{{{[}}\href{/wiki/Ar_fokus?action=edit&section=2}{{modifier}}{{]}}}

Er skouer-mañ e voe lakaet ar fokus war ar rener en ur implijout ar
stumm skaret hag an araogenn \emph{sed}.

A la fin de~la~troisième année, c'est Anne Le Berre qui
sort lever des lignes à cent mètres du port, derrière un rocher
tranquille.

Queffélec 1970, 6

E dibenn~e~drede bloavezh, sed aze Ann ar Berr o vont da sevel linennoù,
kant metrad diouzh ar porzh, a-dreñv ur garreg habask.

Beyer 2016, 8

Amañ e voe implijet \emph{an hini eo} evit pouezañ war ar fokus.

Et c'est une femme, que le recteur imaginait trop sainte
pour lui arracher une promesse, qu'on charge dorénavant
d'allumer les feux.

Queffélec 1970, 27

Hag ur vaouez an hini eo, lakaet gant ar person da re zevot evit
difreuzañ ur bromesa diganti, hag a vo karget hiviziken da enaouiñ an
tanioù.

Beyer 2016, 10

Merket eo ar fokus kontrastañ gant ur
\href{/wiki/Stumm_skaret?action=edit&redlink=1}{stumm skaret} amañ, e
brezhoneg hag e galleg:

Ce n'étaient pas des prêtres qu'il
fallait leur envoyer, de bons prêtres de Quimper qui leur parleraient
breton, mais des missionnaires espagnols et des hommes
d'armes.

Queffélec 1970, 46

N'eo ket beleien e oa dav kas dezho, beleien vat a
Gemper hag a 'z afe e brezhoneg outo met misionerien
spagnolat ha tud armet.

Beyer 2016, 26

\subsection[Plas ar fokus cheñchet gant
adverboù]{\texorpdfstring{\protect\hypertarget{Plas_ar_fokus_che.C3.B1chet_gant_adverbo.C3.B9}{}{}Plas
ar fokus cheñchet gant
adverboù}{Plas ar fokus cheñchet gant adverboù}}\label{Plas_ar_fokus_cheuxf1chet_gant_adverbouxf9}

{{{[}}\href{/wiki/Ar_fokus?action=edit&section=3}{{modifier}}{{]}}}

Adverboù kreizennus a c'hall cheñch plas fokus ar
frazenn:

Qui sait si son refuge,~loin d'être la prière, ne serait
pas la folie.

Queffélec 1970, 5

Piv 'oar ha ne gavfe ket e repu er follentez
kentoc'h eget er bedenn~?

Beyer 2016, 7

Ouzhpennet e voe an adverboù \emph{kentoc'h} (eget) evit
lakaat ar fokus war ar memes elfenn hag er frazenn orin.

\subsection[Fokus degaset gant raganvioù
heklev]{\texorpdfstring{\protect\hypertarget{Fokus_degaset_gant_raganvio.C3.B9_heklev}{}{}Fokus
degaset gant raganvioù
heklev}{Fokus degaset gant raganvioù heklev}}\label{Fokus_degaset_gant_raganviouxf9_heklev}

{{{[}}\href{/wiki/Ar_fokus?action=edit&section=4}{{modifier}}{{]}}}

Raganvioù heklev a c'hall merkañ ar fokus ivez:

Quand ses pauvres ouailles, déjà, s'efforçaient
d'écouter Dieu, de quoi ne devait-il pas être capable~?

Queffélec 1970, 29

Pa gie e zeñved kaezh da selaou mouezh Doue paneveken, petra a
c'helle bezañ dreist d'e varregezh-eñ~?

Beyer 2016, 12

\subsection{Fokus degaset gant ur raganv
diskouez}\label{Fokus_degaset_gant_ur_raganv_diskouez}

{{{[}}\href{/wiki/Ar_fokus?action=edit&section=5}{{modifier}}{{]}}}

Quand Thomas avait vu pour la première fois ce prêtre {[}\ldots{]}, il
s'était échappé de l'église pour avertir
ses parents que l'ile tenait un prêtre différent des
autres et qui ne s'en irait pas~: -- Celui-là sera
enterré dans le cimetière.

Queffélec 1970, 34

Kentañ gwech en doa Tomaz gwelet ar beleg {[}\ldots{]} e oa
tec'het eus an iliz da gemenn d'e dud he
doa tapet an enez kaout ur beleg disheñvel diouzh ar re all ha na
'z afe ket kuit~: -- Henne vo intiarret er vered.

Beyer 2016, 14

\subsection[Fokus merket gant raganvioù
gour]{\texorpdfstring{\protect\hypertarget{Fokus_merket_gant_raganvio.C3.B9_gour}{}{}Fokus
merket gant raganvioù
gour}{Fokus merket gant raganvioù gour}}\label{Fokus_merket_gant_raganviouxf9_gour}

{{{[}}\href{/wiki/Ar_fokus?action=edit&section=6}{{modifier}}{{]}}}

Er frazenn c'hallek amañ dindan e voe merket ar fokus
kontrastañ gant ur raganv gour dizalc'h \emph{lui}
(pauvre terrien):

Ils détiennent une science effrayante et, lui, pauvre terrien, il ne
lutte pas contre eux.

Queffélec 1970, 7

Ganto ez eus ur skiant efreizhus hag ar paourkaezh douarad anezhañ ne
zalc'h ket penn outo.

Beyer 2016, 9

Implijet en deus an troer an araogenn \emph{anezhañ} evit treuzkas ar
mennozh-se, koulskoude e hañval kollet an elfenn keñveriañ. Un doare
d'e virout a c'hellfe bezañ ouzhpennañ
ar raganv kreñv \emph{-eñ}: Ganto ez eus ur skiant efreizhus hag ar
paourkaezh douarad anezhañ ne zalc'h ket {[}-eñ{]} penn
outo.

Ainsi donc, lui, pécheur public, coupable d'une très
lourde faute contre la pureté, et d'une faute aggravée
par le scandale, il allait écrire à son évêque~: «~Monseigneur, ces gens
vous rendent grâces parce que vous m'avez envoyé pour
être leur curé...~»

Queffélec 1970, 36

Evel-se neuze ar pec'her brudet a oa anezhañ, kablus eus
ur faot pounner-spontus ouzh ar c'hlanded, --- fazi
gwashaet dre an droukskouer --- a oa vont da skrivañ d'e
eskob: «~Aotrou 'n Eskob, an dud amañ ho trugareka evit
bezañ kaset ac'hanon da berson dezho....~»

Beyer 2016, 16

Mais l'exemple de son frère l'empêchait
de crier son mécontentement et il ne laissait pas non plus
d'éprouver une jouissance à se croire ici le seul pur

Queffélec 1970, 35

Met skouer e vreur a vire outañ da grial e zisplijadur hag ivez, krediñ
e oa-eñ ar galon c'hlan nemetañ amañ a zegase un doare
levenez dezhañ.

Beyer 2016, 15

\subsection[Fokus kontrastañ diskouezet gant urzh ar
gerioù]{\texorpdfstring{\protect\hypertarget{Fokus_kontrasta.C3.B1_diskouezet_gant_urzh_ar_gerio.C3.B9}{}{}Fokus
kontrastañ diskouezet gant urzh ar
gerioù}{Fokus kontrastañ diskouezet gant urzh ar gerioù}}\label{Fokus_kontrastauxf1_diskouezet_gant_urzh_ar_geriouxf9}

{{{[}}\href{/wiki/Ar_fokus?action=edit&section=7}{{modifier}}{{]}}}

Amañ e weler ur fokus kontrastañ, rak heñvel eo stumm an div lavarenn er
frazenn-mañ, met disheñvel eo ar rener hag ar renadenn dra eeun.

Le prêtre possède la confession, le paroissien possède sa ruse:
«~Promets-moi que tu n'allumeras plus de feux. Je
promets.~»

Queffélec 1970, 7

D'ar beleg eo ar c'hofesa,
d'ar parrezian e widre: ``Gra promesa din ne grogi ket
tan ken. -- Prometet eo.''

Beyer 2016, 10
