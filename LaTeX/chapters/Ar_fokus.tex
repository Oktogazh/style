\chapter{Ar fokus}

\section{Ar fokus a c\textquotesingle haller merkañ gant urzh ar
gerioù e
brezhoneg}\label{ar_fokus_a_challer_merkauxf1_gant_urzh_ar_geriouxf9_e_brezhoneg}

Ar fokus a c\textquotesingle haller merkañ gant urzh ar gerioù. Emañ
plas ar fokus e deroù ar frazenn a-raok ur rannig-verb e brezhoneg
peurliesañ
(\href{https://arbres.iker.cnrs.fr/index.php?title=Focus}{Arbres :
Focus}).

\begin{longtable}[]{|p{0.45\textwidth}|p{0.45\textwidth}}

Il doute du témoignage de ses yeux qui virent le printemps sur
l\textquotesingle île et il voudrait croire qu\textquotesingle il a vu
des mirages. & Arvariñ a ra war testeni e zaoulagad hag a welas an nevez
amzer war an enez, ha karout a rafe krediñ bezañ gwelet touelloù. \\
Queffélec 1970, 5 & Beyer 2016, 7 \\
\end{longtable}

\begin{longtable}[]{|p{0.45\textwidth}|p{0.45\textwidth}}

Mais rien ne doit plus, aujourd\textquotesingle hui, le retenir dans
l\textquotesingle île... & Met hiziv ne zlee tra ebet e
zerc\textquotesingle hel war an enez... \\
Queffélec 1970, 25 & Beyer 2016, 7 \\
\end{longtable}

Er frazenn c\textquotesingle hallek emañ ar
\href{fokus_kontrastañ}{fokus kontrastañ} war
\emph{aujourd\textquotesingle hui}. Merket eo gant ul
\href{lavarenn_gennet}{lavarenn gennet}.

E brezhoneg eo bet lakaet \emph{hiziv} e deroù ar frazenn evit merkañ ar
fokus.

\begin{longtable}[]{|p{0.45\textwidth}|p{0.45\textwidth}}

Personne, ce jour-là, n\textquotesingle a rencontré Yvinec et, si
l\textquotesingle on demande aux pêcheurs d\textquotesingle où ils
connaissent l\textquotesingle histoire de l\textquotesingle épave, ils
crachent et se dérobent derrière des mots. & En devezh-se ne oa bet den
oc\textquotesingle h en em gavout gant Ivineg met pa vez goulennet ouzh
ar besketaerien dre benaos ez anavezont afer ar peñse e tufont hag e
kuzhont a-dreñv ar gerioù. \\
Queffélec 1970, 6 & Beyer 2016, 9 \\
\end{longtable}

\begin{longtable}[]{|p{0.45\textwidth}|p{0.45\textwidth}}

Les îliens, pendant les nuits de tempête, se figurent que les âmes des
noyés pleurent dans le Raz, qu\textquotesingle elles voltigent au-dessus
de la mer, qu\textquotesingle elles montent sur le rivage et heurtent
aux fenêtres. & E-pad an nozvezhioù tourmant e soñj gant an enezourien e
vez eneoù ar re veuzet er raz o ouelañ, e nijellont a-us ar mor, e
savont war an aod hag e skoont ouzh ar prenestri. \\
Queffélec 1970, 7 & Beyer 2016, 9 \\
\end{longtable}

\begin{longtable}[]{|p{0.45\textwidth}|p{0.45\textwidth}}

Il écoutait, les yeux baissés, son supérieur lui décrire son futur
domaine spirituel et il regardait sur le plancher lui apparaître sa
nouvelle paroisse aussi clairement qu\textquotesingle une place publique
du haut d\textquotesingle une tour d\textquotesingle église. & Soublet e
zaoulagad e selaoue e vestr o teskrivañ dezhañ e zomani speredel da
zont, ha treset war ar plañchod e wele e barrez nevez o tont war wel
dezhañ, ken sklaer hag ul leur gêr gwelet a-ziwar un tour iliz. \\
Queffélec 1970, 28 & Beyer 2016, 11 \\
\end{longtable}

\begin{longtable}[]{|p{0.45\textwidth}|p{0.45\textwidth}}

Il lui arriva, au crépuscule, d\textquotesingle entrer dans
l\textquotesingle église et de s\textquotesingle y enfermer à clef pour
s\textquotesingle accorder une séance solitaire
d\textquotesingle imitation. & Gant ar serr-noz e
c\textquotesingle hoarvezas gantañ mont tre en iliz ha prennañ warnañ
evit aotren dezhañ e-unan un abadenn dreveziñ en digenvez. \\
Queffélec 1970, 39 & Beyer 2016, 19 \\
\end{longtable}

\section{Fokus degaset gant stummoù
skaret}\label{fokus_degaset_gant_stummouxf9_skaret}

Er skouer-mañ e voe lakaet ar fokus war ar rener en ur implijout ar
stumm skaret hag an araogenn \emph{sed}.

\begin{longtable}[]{|p{0.45\textwidth}|p{0.45\textwidth}}

A la fin de~la~troisième année, c\textquotesingle est Anne Le Berre qui
sort lever des lignes à cent mètres du port, derrière un rocher
tranquille. & E dibenn~e~drede bloavezh, sed aze Ann ar Berr o vont da
sevel linennoù, kant metrad diouzh ar porzh, a-dreñv ur garreg
habask. \\
Queffélec 1970, 6 & Beyer 2016, 8 \\
\end{longtable}

Amañ e voe implijet \emph{an hini eo} evit pouezañ war ar fokus.

\begin{longtable}[]{|p{0.45\textwidth}|p{0.45\textwidth}}

Et c\textquotesingle est une femme, que le recteur imaginait trop sainte
pour lui arracher une promesse, qu\textquotesingle on charge dorénavant
d\textquotesingle allumer les feux. & Hag ur vaouez an hini eo, lakaet
gant ar person da re zevot evit difreuzañ ur bromesa diganti, hag a vo
karget hiviziken da enaouiñ an tanioù. \\
Queffélec 1970, 27 & Beyer 2016, 10 \\
\end{longtable}

Merket eo ar fokus kontrastañ gant ur \href{stumm_skaret}{stumm skaret}
amañ, e brezhoneg hag e galleg:

\begin{longtable}[]{|p{0.45\textwidth}|p{0.45\textwidth}}

Ce n\textquotesingle étaient pas des prêtres qu\textquotesingle il
fallait leur envoyer, de bons prêtres de Quimper qui leur parleraient
breton, mais des missionnaires espagnols et des hommes
d\textquotesingle armes. & N\textquotesingle eo ket beleien e oa dav kas
dezho, beleien vat a Gemper hag a \textquotesingle z afe e brezhoneg
outo met misionerien spagnolat ha tud armet. \\
Queffélec 1970, 46 & Beyer 2016, 26 \\
\end{longtable}

\section{Plas ar fokus cheñchet gant
adverboù}\label{plas_ar_fokus_cheuxf1chet_gant_adverbouxf9}

Adverboù kreizennus a c\textquotesingle hall cheñch plas fokus ar
frazenn:

\begin{longtable}[]{|p{0.45\textwidth}|p{0.45\textwidth}}

Qui sait si son refuge,~loin d\textquotesingle être la prière, ne serait
pas la folie. & Piv \textquotesingle oar ha ne gavfe ket e repu er
follentez kentoc\textquotesingle h eget er bedenn~? \\
Queffélec 1970, 5 & Beyer 2016, 7 \\
\end{longtable}

Ouzhpennet e voe an adverboù \emph{kentoc\textquotesingle h} (eget) evit
lakaat ar fokus war ar memes elfenn hag er frazenn orin.

\section{Fokus degaset gant raganvioù
heklev}\label{fokus_degaset_gant_raganviouxf9_heklev}

Raganvioù heklev a c\textquotesingle hall merkañ ar fokus ivez:

\begin{longtable}[]{|p{0.45\textwidth}|p{0.45\textwidth}}

Quand ses pauvres ouailles, déjà, s\textquotesingle efforçaient
d\textquotesingle écouter Dieu, de quoi ne devait-il pas être capable~?
& Pa gie e zeñved kaezh da selaou mouezh Doue paneveken, petra a
c\textquotesingle helle bezañ dreist d\textquotesingle e
varregezh-eñ~? \\
Queffélec 1970, 29 & Beyer 2016, 12 \\
\end{longtable}

\section{Fokus degaset gant ur raganv
diskouez}\label{fokus_degaset_gant_ur_raganv_diskouez}

\begin{longtable}[]{|p{0.45\textwidth}|p{0.45\textwidth}}

Quand Thomas avait vu pour la première fois ce prêtre {[}\ldots{]}, il
s\textquotesingle était échappé de l\textquotesingle église pour avertir
ses parents que l\textquotesingle ile tenait un prêtre différent des
autres et qui ne s\textquotesingle en irait pas~: -- Celui-là sera
enterré dans le cimetière. & Kentañ gwech en doa Tomaz gwelet ar beleg
{[}\ldots{]} e oa tec\textquotesingle het eus an iliz da gemenn
d\textquotesingle e dud he doa tapet an enez kaout ur beleg disheñvel
diouzh ar re all ha na \textquotesingle z afe ket kuit~: -- Henne vo
intiarret er vered. \\
Queffélec 1970, 34 & Beyer 2016, 14 \\
\end{longtable}

\section{Fokus merket gant raganvioù
gour}\label{fokus_merket_gant_raganviouxf9_gour}

Er frazenn c\textquotesingle hallek amañ dindan e voe merket ar fokus
kontrastañ gant ur raganv gour dizalc\textquotesingle h \emph{lui}
(pauvre terrien):

\begin{longtable}[]{|p{0.45\textwidth}|p{0.45\textwidth}}

Ils détiennent une science effrayante et, lui, pauvre terrien, il ne
lutte pas contre eux. & Ganto ez eus ur skiant efreizhus hag ar
paourkaezh douarad anezhañ ne zalc\textquotesingle h ket penn outo. \\
Queffélec 1970, 7 & Beyer 2016, 9 \\
\end{longtable}

Implijet en deus an troer an araogenn \emph{anezhañ} evit treuzkas ar
mennozh-se, koulskoude e hañval kollet an elfenn keñveriañ. Un doare
d\textquotesingle e virout a c\textquotesingle hellfe bezañ ouzhpennañ
ar raganv kreñv \emph{-eñ}: Ganto ez eus ur skiant efreizhus hag ar
paourkaezh douarad anezhañ ne zalc\textquotesingle h ket {[}-eñ{]} penn
outo.

\begin{longtable}[]{|p{0.45\textwidth}|p{0.45\textwidth}}

Ainsi donc, lui, pécheur public, coupable d\textquotesingle une très
lourde faute contre la pureté, et d\textquotesingle une faute aggravée
par le scandale, il allait écrire à son évêque~: «~Monseigneur, ces gens
vous rendent grâces parce que vous m\textquotesingle avez envoyé pour
être leur curé...~» & Evel-se neuze ar pec\textquotesingle her brudet a
oa anezhañ, kablus eus ur faot pounner-spontus ouzh ar
c\textquotesingle hlanded, --- fazi gwashaet dre an droukskouer --- a oa
vont da skrivañ d\textquotesingle e eskob: «~Aotrou \textquotesingle n
Eskob, an dud amañ ho trugareka evit bezañ kaset
ac\textquotesingle hanon da berson dezho....~» \\
Queffélec 1970, 36 & Beyer 2016, 16 \\
\end{longtable}

\begin{longtable}[]{|p{0.45\textwidth}|p{0.45\textwidth}}

Mais l\textquotesingle exemple de son frère l\textquotesingle empêchait
de crier son mécontentement et il ne laissait pas non plus
d\textquotesingle éprouver une jouissance à se croire ici le seul pur &
Met skouer e vreur a vire outañ da grial e zisplijadur hag ivez, krediñ
e oa-eñ ar galon c\textquotesingle hlan nemetañ amañ a zegase un doare
levenez dezhañ. \\
Queffélec 1970, 35 & Beyer 2016, 15 \\
\end{longtable}

\section{Fokus kontrastañ diskouezet gant urzh ar
gerioù}\label{fokus_kontrastauxf1_diskouezet_gant_urzh_ar_geriouxf9}

Amañ e weler ur fokus kontrastañ, rak heñvel eo stumm an div lavarenn er
frazenn-mañ, met disheñvel eo ar rener hag ar renadenn dra eeun.

\begin{longtable}[]{|p{0.45\textwidth}|p{0.45\textwidth}}

Le prêtre possède la confession, le paroissien possède sa ruse:
«~Promets-moi que tu n\textquotesingle allumeras plus de feux. Je
promets.~» & D\textquotesingle ar beleg eo ar c\textquotesingle hofesa,
d\textquotesingle ar parrezian e widre: ``Gra promesa din ne grogi ket
tan ken. -- Prometet eo.'' \\
Queffélec 1970, 7 & Beyer 2016, 10 \\
\end{longtable}
