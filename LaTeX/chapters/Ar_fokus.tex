\chapter{Ar fokus}

\section{Ar fokus a c\textquotesingle haller merkañ gant urzh ar
gerioù e
brezhoneg}\label{ar_fokus_a_challer_merkauxf1_gant_urzh_ar_geriouxf9_e_brezhoneg}

Ar fokus a c\textquotesingle haller merkañ gant urzh ar gerioù. Emañ
plas ar fokus e deroù ar frazenn a-raok ur rannig-verb e brezhoneg
peurliesañ
(\href{https://arbres.iker.cnrs.fr/index.php?title=Focus}{Arbres :
Focus}).

Er frazenn c\textquotesingle hallek emañ ar
\href{fokus_kontrastañ}{fokus kontrastañ} war
\emph{aujourd\textquotesingle hui}. Merket eo gant ul
\href{lavarenn_gennet}{lavarenn gennet}.

E brezhoneg eo bet lakaet \emph{hiziv} e deroù ar frazenn evit merkañ ar
fokus.

\section{Fokus degaset gant stummoù
skaret}\label{fokus_degaset_gant_stummouxf9_skaret}

Er skouer-mañ e voe lakaet ar fokus war ar rener en ur implijout ar
stumm skaret hag an araogenn \emph{sed}.

Amañ e voe implijet \emph{an hini eo} evit pouezañ war ar fokus.

Merket eo ar fokus kontrastañ gant ur \href{stumm_skaret}{stumm skaret}
amañ, e brezhoneg hag e galleg:

\section{Plas ar fokus cheñchet gant
adverboù}\label{plas_ar_fokus_cheuxf1chet_gant_adverbouxf9}

Adverboù kreizennus a c\textquotesingle hall cheñch plas fokus ar
frazenn:

Ouzhpennet e voe an adverboù \emph{kentoc\textquotesingle h} (eget) evit
lakaat ar fokus war ar memes elfenn hag er frazenn orin.

\section{Fokus degaset gant raganvioù
heklev}\label{fokus_degaset_gant_raganviouxf9_heklev}

Raganvioù heklev a c\textquotesingle hall merkañ ar fokus ivez:

\section{Fokus degaset gant ur raganv
diskouez}\label{fokus_degaset_gant_ur_raganv_diskouez}

\section{Fokus merket gant raganvioù
gour}\label{fokus_merket_gant_raganviouxf9_gour}

Er frazenn c\textquotesingle hallek amañ dindan e voe merket ar fokus
kontrastañ gant ur raganv gour dizalc\textquotesingle h \emph{lui}
(pauvre terrien):

Implijet en deus an troer an araogenn \emph{anezhañ} evit treuzkas ar
mennozh-se, koulskoude e hañval kollet an elfenn keñveriañ. Un doare
d\textquotesingle e virout a c\textquotesingle hellfe bezañ ouzhpennañ
ar raganv kreñv \emph{-eñ}: Ganto ez eus ur skiant efreizhus hag ar
paourkaezh douarad anezhañ ne zalc\textquotesingle h ket {[}-eñ{]} penn
outo.

\section{Fokus kontrastañ diskouezet gant urzh ar
gerioù}\label{fokus_kontrastauxf1_diskouezet_gant_urzh_ar_geriouxf9}

Amañ e weler ur fokus kontrastañ, rak heñvel eo stumm an div lavarenn er
frazenn-mañ, met disheñvel eo ar rener hag ar renadenn dra eeun.
