\chapter{Project presentation in English}
\subsection[What is
STYLE?]{\texorpdfstring{\protect\hypertarget{What_is_STYLE.3F}{}{}What
is STYLE?}{What is STYLE?}}\label{What_is_STYLEux3f}

{{{[}}\href{/wiki/Project_presentation_in_English?action=edit&section=1}{{modifier}}{{]}}}

STYLE is a Breton acronym for Studi an Troidigezhioù er Yezh Lennegel,
meaning Study of Translations in Literary Language. This wiki serves as
my research notebook for the stylistic analysis of Breton, which I began
as part of master's dissertation at Rennes 2 University.

\subsection{Abstract}\label{Abstract}

{{{[}}\href{/wiki/Project_presentation_in_English?action=edit&section=2}{{modifier}}{{]}}}

The project consists in creating a stylistic resource for the
Breton-speaking community, mostly translators. I am comparing the
structure of French and Breton by analysing a translation corpus, then
highlighting the linguistic patterns I find. To analyse and describe the
structures I retrieve from the corpus, I'm leaning on
existing resources and grammars, primarily
\href{/wiki/Jouitteau_2009-2024}{Jouitteau 2009-2024} and
\href{/wiki/Kervella_1947}{Kervella 1947}. This method, known as
comparative stylistics, originated with
\href{/wiki/Darbelnet_\%26_Vinay_1993}{Darbelnet \& Vinay 1993} as a
means to teach translation between English and French. Later,
\href{/wiki/Rottet_\%26_Moris_2018}{Rottet \& Moris 2018} extended this
approach to comparing English and Welsh.

For my initial translation corpus, I've selected two
books---one translated from French to Breton, and the other from Breton
to French. By comparing translations in both directions, I can ensure
the analysis remains impartial, showing that specific linguistic
patterns exist regardless of a particular author's
style.

To make my work easily accessible and searchable, I've
opted for a digital format using Media Wiki. This setup will allow for
easy distribution and expansion of the translation corpus beyond the
initial two books. Plus, it opens the door to potential collaboration on
the project.

Lastly, the digital format under a CC-BY licence will also allow
extracting a parallel corpus for training AI tools, which is necessary
for digitally underdeveloped languages to compensate for their lack of
raw corpus (Jouitteau 2023d).

\subsection{Research purpose}\label{Research_purpose}

{{{[}}\href{/wiki/Project_presentation_in_English?action=edit&section=3}{{modifier}}{{]}}}

Develop a stylistic resource to be used directly by the speaking
community of Breton, a language that is in serious danger of extinction
according to UNESCO (Moseley 2010, 24-25).

\subsection{Methodology}\label{Methodology}

{{{[}}\href{/wiki/Project_presentation_in_English?action=edit&section=4}{{modifier}}{{]}}}

Based on a translation corpus from French to Breton (Queffélec 1944,
translated by Beyer 2016) and from Breton to French (Drezen 2012,
translated by Drezen 1943), the aim is to identify linguistic structures
that require commentary, following the method of comparative stylistics
(Rottet \& Morris 2018; Darbelnet \& Vinay 1993).

These structures will be organized in a format that is directly
accessible and searchable by users (primarily translators). The project
will provide a translational inventory of these structures and refer to
existing grammars for their description and analysis.

The work will be carried out in digital form using MediaWiki, a
collaborative content management system based on the wiki model, which
allows users to create, link, and structure pages.

\subsection{Grammatical terminology}\label{Grammatical_terminology}

{{{[}}\href{/wiki/Project_presentation_in_English?action=edit&section=5}{{modifier}}{{]}}}

Creating material that is directly accessible to the Breton-speaking
community presents a dual challenge: all commentary must be written in
Breton, and it must also be understandable to readers who are not
specialists in linguistics. The central difficulty lies in establishing
consistent grammatical, stylistic, and translational terminology, and
compiling these terms into a glossary.

This glossary will include trilingual translations of each term
(English, French, and Breton), along with clear definitions and
illustrative examples that demonstrate how each term is used in context.

In the case of Breton, standardized terminology is not always available.
When this occurs, careful choices must be made: either by selecting
existing terms used by some authors over others, or by creating new
translations of terminology already available in English and French.

One example explored in this project involves a structure that begins
with a noun phrase which shares certain syntactic features with the
subject, but is not actually the subject of the sentence (Jouitteau
2009--2024). This construction is referred to differently in the
literature. Jouitteau (2005/2010) and Rezac (2009) call it \emph{la
construction du faux sujet} in French, and \emph{wrong subject
construction} in English. Le Clerc (1986) refers to it as
\emph{complément anticipé}, Urien (1989) as \emph{relation médiate}, and
Fave (1998) as \emph{complément redoublé}. Fave also proposes a Breton
equivalent, \emph{renadenn adveneget}. However, many sources present
terminology exclusively in French, which adds to the complexity of
aligning concepts across languages.

In such cases, the challenge is twofold: first, to determine which
French term most accurately describes the structure in question; and
second, to find, or, if necessary, to create a suitable equivalent in
Breton.

In order to ensure the quality of the translated or invented terms, I
will seek scientific advice from the
\href{https://brezhoneg21.com/geriadurGB.php\#geriadur}{Kreizenn ar
Geriaouiñ} association, created on Diwan's initiative in
order to develop and provide the terminological and pedagogical tools
necessary for the opening of the first Diwan school.
