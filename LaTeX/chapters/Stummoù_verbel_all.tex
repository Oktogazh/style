\begin{LTR}
\begin{otherlanguage}{breton}

\subsection[Stumm gant Ha(g) + sujed +
da]{\texorpdfstring{\protect\hypertarget{Stumm_gant_Ha.28g.29_.2B_sujed_.2B_da}{}{}Stumm
gant Ha(g) + sujed +
da}{Stumm gant Ha(g) + sujed + da}}\label{Stumm_gant_Haux28gux29_+_sujed_+_da}

{{{[}}\href{/wiki/Stummo\%C3\%B9_verbel_all?action=edit&section=1}{{modifier}}{{]}}}

Tout le monde se signait, les femmes, les enfants, les hommes, ceux de
la nef qui le voyaient, et ceux des bas-côtés qui ne le voyaient pas et
qui suivaient la nef~; il disposait d'un pouvoir de
prêtre

Queffélec 1970, 57

Hag an holl d'ober sin ar groaz, ar maouezed, ar vugale,
ar wazed, ar re a oa en nev hag a wele anezhañ, hag ar re a oa en adnev
hag a 'z ae da heul an neviad.

Beyer 2016, 37

\subsection{Stumm-ober gallek troet gant un anv-kadarn e
brezhoneg}\label{Stumm-ober_gallek_troet_gant_un_anv-kadarn_e_brezhoneg}

{{{[}}\href{/wiki/Stummo\%C3\%B9_verbel_all?action=edit&section=2}{{modifier}}{{]}}}

Il n'est pas possible~que des aubes glorieuses, se
déployant dans le fond du ciel, aient éclairé ce morceau de récif.

Queffélec 1970, 5

Ne c'hell ket bezañ bet~ruzelloù-mintin o sklêrijennañ
an tamm penn-karreg-se en o dispak glorius e don an oabl.

Beyer 2016, 7

Pensant que cela durerait, elle n'avait plus récité ses
prières, ou ne les avait récitées que du bout des lèvres.

Queffélec 1970, 40

Gant ar soñj e padfe an traoù n'he doa ket mui dibunet
he fedennoù pe he doa dibunet anezho diwar blein an teod.

Beyer 2016, 20

Une tempête d'équinoxe, bouleversant la mer, emprisonna les îliens sur
leur récif --- et ce furent veillées et conciliabules, courants d'air
glapissants dans les ruelles tortes, repas chiches coupés de phrases
lentes...

Queffélec 1970, 59

Gant un tourmant kedez ha diroll ar mor e voe bac'het an enezourien war
o fenn-karreg --- ha neuze e voe beilhadegoù ha kuzhutadegoù,
avelioù-red o speuñial er straedigoù kamm-digamm, predoù moan poentaouet
gant kaozioù gorrek...

Beyer 2016, 39

\subsection[\emph{O vezañ
ma}]{\texorpdfstring{\protect\hypertarget{O_veza.C3.B1_ma}{}{}\emph{O
vezañ ma}}{O vezañ ma}}\label{O_vezauxf1_ma}

{{{[}}\href{/wiki/Stummo\%C3\%B9_verbel_all?action=edit&section=3}{{modifier}}{{]}}}

Implij ar rannig \emph{o} gant ar \href{/wiki/Stumm_ober}{stumm ober} a
verb bezañ a zo strizh. Ne c'haller ket displegañ an
holl verboù gant ar stumm-se.

Da skouer, lavaret e vez \emph{Gouzout a ra} ha neket \emph{Emañ o
c'houzout}.

Koulskoude e vez implijet ar stumm \emph{O vezañ ma} evit kinnig
\href{/wiki/Renadenno\%C3\%B9_abeg}{renadennoù abeg}, da skouer evit
treiñ stummoù evel \emph{étant donné que/puisque} diwar ar galleg.

\href{/wiki/O_veza\%C3\%B1_ma_ne_chome_ket_Sakramant_an_aoter_e_diskouez}{O
vezañ ma ne chome ket Sakramant an aoter e diskouez}

\end{otherlanguage}
\end{LTR}
