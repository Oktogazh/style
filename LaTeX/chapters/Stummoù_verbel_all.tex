\chapter{Stummoù verbel all}

\section{Stumm gant Ha(g) + sujed +
da}\label{stumm_gant_hag_sujed_da}

\begin{longtable}[]{@{}ll@{}}
\toprule\noalign{}
\endhead
\bottomrule\noalign{}
\endlastfoot
Tout le monde se signait, les femmes, les enfants, les hommes, ceux de
la nef qui le voyaient, et ceux des bas-côtés qui ne le voyaient pas et
qui suivaient la nef~; il disposait d\textquotesingle un pouvoir de
prêtre & Hag an holl d\textquotesingle ober sin ar groaz, ar maouezed,
ar vugale, ar wazed, ar re a oa en nev hag a wele anezhañ, hag ar re a
oa en adnev hag a \textquotesingle z ae da heul an neviad. \\
Queffélec 1970, 57 & Beyer 2016, 37 \\
\end{longtable}

\section{Stumm-ober gallek troet gant un anv-kadarn e
brezhoneg}\label{stumm_ober_gallek_troet_gant_un_anv_kadarn_e_brezhoneg}

\begin{longtable}[]{@{}ll@{}}
\toprule\noalign{}
\endhead
\bottomrule\noalign{}
\endlastfoot
Il n\textquotesingle est pas possible~que des aubes glorieuses, se
déployant dans le fond du ciel, aient éclairé ce morceau de récif. & Ne
c\textquotesingle hell ket bezañ bet~ruzelloù-mintin o sklêrijennañ an
tamm penn-karreg-se en o dispak glorius e don an oabl. \\
Queffélec 1970, 5 & Beyer 2016, 7 \\
\end{longtable}

\begin{longtable}[]{@{}ll@{}}
\toprule\noalign{}
\endhead
\bottomrule\noalign{}
\endlastfoot
Pensant que cela durerait, elle n\textquotesingle avait plus récité ses
prières, ou ne les avait récitées que du bout des lèvres. & Gant ar soñj
e padfe an traoù n\textquotesingle he doa ket mui dibunet he fedennoù pe
he doa dibunet anezho diwar blein an teod. \\
Queffélec 1970, 40 & Beyer 2016, 20 \\
\end{longtable}

\begin{longtable}[]{@{}ll@{}}
\toprule\noalign{}
\endhead
\bottomrule\noalign{}
\endlastfoot
Une tempête d'équinoxe, bouleversant la mer, emprisonna les îliens sur
leur récif --- et ce furent veillées et conciliabules, courants d'air
glapissants dans les ruelles tortes, repas chiches coupés de phrases
lentes... & Gant un tourmant kedez ha diroll ar mor e voe bac'het an
enezourien war o fenn-karreg --- ha neuze e voe beilhadegoù ha
kuzhutadegoù, avelioù-red o speuñial er straedigoù kamm-digamm, predoù
moan poentaouet gant kaozioù gorrek... \\
Queffélec 1970, 59 & Beyer 2016, 39 \\
\end{longtable}

\section{\texorpdfstring{\emph{O vezañ
ma}}{O vezañ ma}}\label{o_vezauxf1_ma}

Implij ar rannig \emph{o} gant ar \href{stumm_ober}{stumm ober} a verb
bezañ a zo strizh. Ne c\textquotesingle haller ket displegañ an holl
verboù gant ar stumm-se.

Da skouer, lavaret e vez \emph{Gouzout a ra} ha neket \emph{Emañ o
c\textquotesingle houzout}.

Koulskoude e vez implijet ar stumm \emph{O vezañ ma} evit kinnig
\href{renadennoù_abeg}{renadennoù abeg}, da skouer evit treiñ stummoù
evel \emph{étant donné que/puisque} diwar ar galleg.

\href{O_vezañ_ma_ne_chome_ket_Sakramant_an_aoter_e_diskouez}{O vezañ ma
ne chome ket Sakramant an aoter e diskouez}
