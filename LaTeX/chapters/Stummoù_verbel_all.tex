\chapter{Stummoù verbel all}

\section{Stumm gant Ha(g) + sujed +
da}\label{stumm_gant_hag_sujed_da}

\section{Stumm-ober gallek troet gant un anv-kadarn e
brezhoneg}\label{stumm_ober_gallek_troet_gant_un_anv_kadarn_e_brezhoneg}

\section{\texorpdfstring{\emph{O vezañ
ma}}{O vezañ ma}}\label{o_vezauxf1_ma}

Implij ar rannig \emph{o} gant ar \href{stumm_ober}{stumm ober} a verb
bezañ a zo strizh. Ne c\textquotesingle haller ket displegañ an holl
verboù gant ar stumm-se.

Da skouer, lavaret e vez \emph{Gouzout a ra} ha neket \emph{Emañ o
c\textquotesingle houzout}.

Koulskoude e vez implijet ar stumm \emph{O vezañ ma} evit kinnig
\href{renadennoù_abeg}{renadennoù abeg}, da skouer evit treiñ stummoù
evel \emph{étant donné que/puisque} diwar ar galleg.

\href{O_vezañ_ma_ne_chome_ket_Sakramant_an_aoter_e_diskouez}{O vezañ ma
ne chome ket Sakramant an aoter e diskouez}
