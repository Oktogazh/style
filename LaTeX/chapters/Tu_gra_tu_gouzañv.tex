\chapter{Tu-gra, tu-gouzañv}

\section{Tu-gouzañv}\label{tu_gouzauxf1v}

\section{Tu-gouzañv brezhonek implijet evit treuzkas ar
fokus}\label{tu_gouzauxf1v_brezhonek_implijet_evit_treuzkas_ar_fokus}

Implijet e voe an tu-gouzañv en droidigezh vrezhonek-mañ evit treuzkas
ar \href{fokus_kontrastañ}{fokus kontrastañ} dre ziskouez un enebiezh
etre an hini a enaou an tanioù hag ar re a beurechuo al labour.

\section{Tu-gouzañv implijet e brezhoneg evit pouezañ war ar stumm
boutin a verb
bezañ}\label{tu_gouzauxf1v_implijet_e_brezhoneg_evit_pouezauxf1_war_ar_stumm_boutin_a_verb_bezauxf1}

En ur lakaat an tu-gouzañv amañ e c\textquotesingle haller implijout
stumm boutin ar verb bezañ evit lakaat pouezh war un obererezh boutin
\emph{veze}, ar pezh ne vije ket bet posupl gant an tu-gra.

Implijet e voe an tu-gouzañv amañ evit diskouez dre ar stumm boutin a
verb bezañ e c\textquotesingle hoarveze ur seurt obererezh alies. Posupl
e vije bet lavaret \emph{Pevar be bemp kantik a ganed}, met
splannoc\textquotesingle h eo efed an obererezh boutin gant \emph{veze}.

\section{Tu-gouzañv implijet en droidigezh e-lec\textquotesingle h
tu-gra evit tremen hebiou displegañ ar verb kaout d\textquotesingle an
trede gour
unan}\label{tu_gouzauxf1v_implijet_en_droidigezh_e_lech_tu_gra_evit_tremen_hebiou_displegauxf1_ar_verb_kaout_dan_trede_gour_unan}

Er frazenn-mañ e tibabas ann droeerez implijout an tu-gouzañv
e-lec\textquotesingle h an tu-gra evit tremen hebiou displegañ ar verb
\emph{kaout} d\textquotesingle an trede gour unan: \emph{Ar mor, an
divroañ, an digenvez spered \textbf{en deus/o deus} diskaret e
nerzh-kalon}.
