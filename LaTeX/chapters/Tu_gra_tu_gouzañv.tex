\chapter{Tu-gra, tu-gouzañv}
\section[Tu-gouzañv]{\texorpdfstring{\protect\hypertarget{Tu-gouza.C3.B1v}{}{}Tu-gouzañv}{Tu-gouzañv}}\label{Tu-gouzauxf1v}

{{{[}}\href{/wiki/Tu-gra,_tu-gouza\%C3\%B1v?action=edit&section=1}{{modifier}}{{]}}}

\subsection[Tu-gouzañv brezhonek implijet evit treuzkas ar
fokus]{\texorpdfstring{\protect\hypertarget{Tu-gouza.C3.B1v_brezhonek_implijet_evit_treuzkas_ar_fokus}{}{}Tu-gouzañv
brezhonek implijet evit treuzkas ar
fokus}{Tu-gouzañv brezhonek implijet evit treuzkas ar fokus}}\label{Tu-gouzauxf1v_brezhonek_implijet_evit_treuzkas_ar_fokus}

{{{[}}\href{/wiki/Tu-gra,_tu-gouza\%C3\%B1v?action=edit&section=2}{{modifier}}{{]}}}

C'est lui qui allume les feux et les autres achèvent sa
besogne sans le considérer comme coupable.

Queffélec 1970, 27

Gantañ e vo enaouet an tanioù ha gant ar re all e vo peurechuet e labour
hep na vije lakaet ganto da giriek.

Beyer 2016, 10

Implijet e voe an tu-gouzañv en droidigezh vrezhonek-mañ evit treuzkas
ar \href{/wiki/Fokus_kontrasta\%C3\%B1}{fokus kontrastañ} dre ziskouez
un enebiezh etre an hini a enaou an tanioù hag ar re a beurechuo al
labour.

L'ermite de l'île, si Dieu
l'aimait --- et Dieu l'avait aimé ---,
avait dû lutter contre les offres des belles îliennes que tourmentait sa
piété.

Queffélec 1970, 54-55

Moarvat en doa bet penitiour an enez, mard e oa bet karet gant Doue ---
ha Doue en doa karet anezhañ ---, da stourm ouzh kinnigoù an enezourezed
kaer trubuilhet gant e zeoliezh.

Beyer 2016, 35

\subsection[Tu-gouzañv implijet e brezhoneg evit pouezañ war ar stumm
boutin a verb
bezañ]{\texorpdfstring{\protect\hypertarget{Tu-gouza.C3.B1v_implijet_e_brezhoneg_evit_poueza.C3.B1_war_ar_stumm_boutin_a_verb_beza.C3.B1}{}{}Tu-gouzañv
implijet e brezhoneg evit pouezañ war ar stumm boutin a verb
bezañ}{Tu-gouzañv implijet e brezhoneg evit pouezañ war ar stumm boutin a verb bezañ}}\label{Tu-gouzauxf1v_implijet_e_brezhoneg_evit_pouezauxf1_war_ar_stumm_boutin_a_verb_bezauxf1}

{{{[}}\href{/wiki/Tu-gra,_tu-gouza\%C3\%B1v?action=edit&section=3}{{modifier}}{{]}}}

Les gestes de son maître hantaient sa mémoire, et, quand il ne résistait
pas de toutes ses forces, il s'abandonnait à les imiter
{[}...{]}.

Queffélec 1970, 39

Sorc'hennet e veze e vemor gant jestroù e vestr, ha pa
ne lakae ket e holl nerzh da zerc'hel penn en em leze da
vont betek o zreveziñ {[}...{]}.

Beyer 2016, 19

En ur lakaat an tu-gouzañv amañ e c'haller implijout
stumm boutin ar verb bezañ evit lakaat pouezh war un obererezh boutin
\emph{veze}, ar pezh ne vije ket bet posupl gant an tu-gra.

On chantait quatre ou cinq cantiques, puis il y avait encore un temps de
silence et d'attente.

Queffélec 1970, 44

Pevar pe bemp kantik a veze kanet ha da heul e veze adarre ur berr a
c'hortozadeg didrouz.

Beyer 2016, 24

Implijet e voe an tu-gouzañv amañ evit diskouez dre ar stumm boutin a
verb bezañ e c'hoarveze ur seurt obererezh alies. Posupl
e vije bet lavaret \emph{Pevar be bemp kantik a ganed}, met
splannoc'h eo efed an obererezh boutin gant \emph{veze}.

\subsection[Tu-gouzañv implijet en droidigezh e-lec'h
tu-gra evit tremen hebiou displegañ ar verb kaout d'an
trede gour
unan]{\texorpdfstring{\protect\hypertarget{Tu-gouza.C3.B1v_implijet_en_droidigezh_e-lec.27h_tu-gra_evit_tremen_hebiou_displega.C3.B1_ar_verb_kaout_d.27an_trede_gour_unan}{}{}Tu-gouzañv
implijet en droidigezh e-lec'h tu-gra evit tremen hebiou
displegañ ar verb kaout d'an trede gour
unan}{Tu-gouzañv implijet en droidigezh e-lec'h tu-gra evit tremen hebiou displegañ ar verb kaout d'an trede gour unan}}\label{Tu-gouzauxf1v_implijet_en_droidigezh_e-lecux27h_tu-gra_evit_tremen_hebiou_displegauxf1_ar_verb_kaout_dux27an_trede_gour_unan}

{{{[}}\href{/wiki/Tu-gra,_tu-gouza\%C3\%B1v?action=edit&section=4}{{modifier}}{{]}}}

La mer, le dépaysement, la solitude morale ont détruit son courage et,
si l'on résistait à son entreprise, en poursuivrait-il
le dessein?

Queffélec 1970, 5

Diskaret eo bet e nerzh-kalon gant ar mor, an divroañ, an digenvez
spered, ha daoust ha derc'hel a rafe d'e
grog ma c'hoarvezfe da unan enebiñ ouzh e embregadenn?

Beyer 2016, 7

Er frazenn-mañ e tibabas ann droeerez implijout an tu-gouzañv
e-lec'h an tu-gra evit tremen hebiou displegañ ar verb
\emph{kaout} d'an trede gour unan: \emph{Ar mor, an
divroañ, an digenvez spered \textbf{en deus/o deus} diskaret e
nerzh-kalon}.
