\chapter{Tu-gra, tu-gouzañv}

\section{Tu-gouzañv}\label{tu_gouzauxf1v}

\section{Tu-gouzañv brezhonek implijet evit treuzkas ar
fokus}\label{tu_gouzauxf1v_brezhonek_implijet_evit_treuzkas_ar_fokus}

\begin{longtable}[]{|p{0.45\textwidth}|p{0.45\textwidth}}

C\textquotesingle est lui qui allume les feux et les autres achèvent sa
besogne sans le considérer comme coupable. & Gantañ e vo enaouet an
tanioù ha gant ar re all e vo peurechuet e labour hep na vije lakaet
ganto da giriek. \\
Queffélec 1970, 27 & Beyer 2016, 10 \\
\end{longtable}

Implijet e voe an tu-gouzañv en droidigezh vrezhonek-mañ evit treuzkas
ar \href{fokus_kontrastañ}{fokus kontrastañ} dre ziskouez un enebiezh
etre an hini a enaou an tanioù hag ar re a beurechuo al labour.

\begin{longtable}[]{|p{0.45\textwidth}|p{0.45\textwidth}}

L\textquotesingle ermite de l\textquotesingle île, si Dieu
l\textquotesingle aimait --- et Dieu l\textquotesingle avait aimé ---,
avait dû lutter contre les offres des belles îliennes que tourmentait sa
piété. & Moarvat en doa bet penitiour an enez, mard e oa bet karet gant
Doue --- ha Doue en doa karet anezhañ ---, da stourm ouzh kinnigoù an
enezourezed kaer trubuilhet gant e zeoliezh. \\
Queffélec 1970, 54-55 & Beyer 2016, 35 \\
\end{longtable}

\section{Tu-gouzañv implijet e brezhoneg evit pouezañ war ar stumm
boutin a verb
bezañ}\label{tu_gouzauxf1v_implijet_e_brezhoneg_evit_pouezauxf1_war_ar_stumm_boutin_a_verb_bezauxf1}

\begin{longtable}[]{|p{0.45\textwidth}|p{0.45\textwidth}}

Les gestes de son maître hantaient sa mémoire, et, quand il ne résistait
pas de toutes ses forces, il s\textquotesingle abandonnait à les imiter
{[}...{]}. & Sorc\textquotesingle hennet e veze e vemor gant jestroù e
vestr, ha pa ne lakae ket e holl nerzh da zerc\textquotesingle hel penn
en em leze da vont betek o zreveziñ {[}...{]}. \\
Queffélec 1970, 39 & Beyer 2016, 19 \\
\end{longtable}

En ur lakaat an tu-gouzañv amañ e c\textquotesingle haller implijout
stumm boutin ar verb bezañ evit lakaat pouezh war un obererezh boutin
\emph{veze}, ar pezh ne vije ket bet posupl gant an tu-gra.

\begin{longtable}[]{|p{0.45\textwidth}|p{0.45\textwidth}}

On chantait quatre ou cinq cantiques, puis il y avait encore un temps de
silence et d\textquotesingle attente. & Pevar pe bemp kantik a veze
kanet ha da heul e veze adarre ur berr a c\textquotesingle hortozadeg
didrouz. \\
Queffélec 1970, 44 & Beyer 2016, 24 \\
\end{longtable}

Implijet e voe an tu-gouzañv amañ evit diskouez dre ar stumm boutin a
verb bezañ e c\textquotesingle hoarveze ur seurt obererezh alies. Posupl
e vije bet lavaret \emph{Pevar be bemp kantik a ganed}, met
splannoc\textquotesingle h eo efed an obererezh boutin gant \emph{veze}.

\section{Tu-gouzañv implijet en droidigezh e-lec\textquotesingle h
tu-gra evit tremen hebiou displegañ ar verb kaout d\textquotesingle an
trede gour
unan}\label{tu_gouzauxf1v_implijet_en_droidigezh_e_lech_tu_gra_evit_tremen_hebiou_displegauxf1_ar_verb_kaout_dan_trede_gour_unan}

\begin{longtable}[]{|p{0.45\textwidth}|p{0.45\textwidth}}

La mer, le dépaysement, la solitude morale ont détruit son courage et,
si l\textquotesingle on résistait à son entreprise, en poursuivrait-il
le dessein? & Diskaret eo bet e nerzh-kalon gant ar mor, an divroañ, an
digenvez spered, ha daoust ha derc\textquotesingle hel a rafe
d\textquotesingle e grog ma c\textquotesingle hoarvezfe da unan enebiñ
ouzh e embregadenn? \\
Queffélec 1970, 5 & Beyer 2016, 7 \\
\end{longtable}

Er frazenn-mañ e tibabas ann droeerez implijout an tu-gouzañv
e-lec\textquotesingle h an tu-gra evit tremen hebiou displegañ ar verb
\emph{kaout} d\textquotesingle an trede gour unan: \emph{Ar mor, an
divroañ, an digenvez spered \textbf{en deus/o deus} diskaret e
nerzh-kalon}.
