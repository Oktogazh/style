\chapter{Stumm kadarn, nac'h, goulennata, estlammañ}

\section{Stumm goulennata}\label{stumm_goulennata}

\subsection{\texorpdfstring{Stumm gant \emph{ra}}{Stumm gant ra}}\label{stumm_gant_ra}

Ar stumm goulennata krouet gant an amginadur (inversion) e galleg a c\textquotesingle haller treiñ gant \emph{ra} e bzg:

\begin{longtable}[]{|p{0.45\textwidth}|p{0.45\textwidth}}

Puis-je, Monseigneur, vous raconter l\textquotesingle allégresse de ces gens à la vue du costume que je porte? & Ra vo permetet din, Aotrou \textquotesingle n Eskob, kontañ deoc\textquotesingle h levenez an dud-mañ o welet ma gwiskamant? \\ Queffélec 1970, 36 & Beyer 2016, 16 \\ \end{longtable}

\subsection{Stumm kadarn pe estlammañ gallek cheñchet d\textquotesingle ar stumm goulennata e brezhoneg}\label{stumm_kadarn_pe_estlammauxf1_gallek_cheuxf1chet_dar_stumm_goulennata_e_brezhoneg}

Ur goulenn retorik pe dieeun a c\textquotesingle haller treiñ e brezhoneg gant ur stagell isurzhiañ \emph{ha}. Neuze e c\textquotesingle hall cheñch ar stumm kadarn pe estlammañ er frazenn c\textquotesingle hallek d\textquotesingle ar stumm goulennata e brezhoneg.

\begin{longtable}[]{|p{0.45\textwidth}|p{0.45\textwidth}}

Qui sait si son refuge,~loin d\textquotesingle être la prière, ne serait pas la folie. & Piv \textquotesingle oar ha ne gavfe ket e repu er follentez kentoc\textquotesingle h eget er bedenn~? \\ Queffélec 1970, 5 & Beyer 2016, 7 \\ \end{longtable}

\begin{longtable}[]{|p{0.45\textwidth}|p{0.45\textwidth}}

N\textquotesingle est-ce pas plus terrible encore à la fin de l\textquotesingle antan, la mort de Louis Yvinec! & Ha daoust ha ne oa ket bet tra spontusoc\textquotesingle h c\textquotesingle hoazh~marv Loeiz Ivineg e dibenn ar bloaz tremenet? \\ Queffélec 1970, 6 & Beyer 2016, 8 \\ \end{longtable}

\section{Stumm nac\textquotesingle h}\label{stumm_nach}

\subsection{Stumm nac\textquotesingle h lakaet war wel gant disteraennoù}\label{stumm_nach_lakaet_war_wel_gant_disteraennouxf9}

Evit kreñvaat ar stumm nac\textquotesingle h e c\textquotesingle haller implijout un disteraenn.

\begin{longtable}[]{|p{0.45\textwidth}|p{0.45\textwidth}}

Personne ne parlait. & Den ne ranne grik. \\ Queffélec 1970, 48 & Beyer 2016, 29 \\ \end{longtable}

\begin{longtable}[]{|p{0.45\textwidth}|p{0.45\textwidth}}

La disparition d\textquotesingle Anne se conforme à une sagesse de l\textquotesingle horreur: personne n\textquotesingle a rien vu ni rien entendu et personne ne sait rien. & Reizhet e oa steuziadur Ann ouzh poell an euzh en ur mod: den n\textquotesingle en doa gwelet na klevet tra, den ne ouie tra. \\ Queffélec 1970, 26 & Beyer 2016, 8 \\ \end{longtable}

Category:Pennadoù 