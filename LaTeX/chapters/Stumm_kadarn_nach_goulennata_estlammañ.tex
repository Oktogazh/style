\chapter{Stumm kadarn, nac'h, goulennata, estlammañ}
\subsection{Stumm goulennata}\label{Stumm_goulennata}

{{{[}}\href{/wiki/Stumm_kadarn,_nac\%27h,_goulennata,_estlamma\%C3\%B1?action=edit&section=1}{{modifier}}{{]}}}

\subsubsection{\texorpdfstring{Stumm gant
\emph{ra}}{Stumm gant ra}}\label{Stumm_gant_ra}

{{{[}}\href{/wiki/Stumm_kadarn,_nac\%27h,_goulennata,_estlamma\%C3\%B1?action=edit&section=2}{{modifier}}{{]}}}

Ar stumm goulennata krouet gant an amginadur (inversion) e galleg a
c'haller treiñ gant \emph{ra} e bzg:

Puis-je, Monseigneur, vous raconter l'allégresse de ces
gens à la vue du costume que je porte?

Queffélec 1970, 36

Ra vo permetet din, Aotrou 'n Eskob, kontañ
deoc'h levenez an dud-mañ o welet ma gwiskamant?

Beyer 2016, 16

\subsubsection[Stumm kadarn pe estlammañ gallek cheñchet
d'ar stumm goulennata e
brezhoneg]{\texorpdfstring{\protect\hypertarget{Stumm_kadarn_pe_estlamma.C3.B1_gallek_che.C3.B1chet_d.27ar_stumm_goulennata_e_brezhoneg}{}{}Stumm
kadarn pe estlammañ gallek cheñchet d'ar stumm
goulennata e
brezhoneg}{Stumm kadarn pe estlammañ gallek cheñchet d'ar stumm goulennata e brezhoneg}}\label{Stumm_kadarn_pe_estlammauxf1_gallek_cheuxf1chet_dux27ar_stumm_goulennata_e_brezhoneg}

{{{[}}\href{/wiki/Stumm_kadarn,_nac\%27h,_goulennata,_estlamma\%C3\%B1?action=edit&section=3}{{modifier}}{{]}}}

Ur goulenn retorik pe dieeun a c'haller treiñ e
brezhoneg gant ur stagell isurzhiañ \emph{ha}. Neuze e
c'hall cheñch ar stumm kadarn pe estlammañ er frazenn
c'hallek d'ar stumm goulennata e
brezhoneg.

Qui sait si son refuge,~loin d'être la prière, ne serait
pas la folie.

Queffélec 1970, 5

Piv 'oar ha ne gavfe ket e repu er follentez
kentoc'h eget er bedenn~?

Beyer 2016, 7

N'est-ce pas plus terrible encore à la fin de
l'antan, la mort de Louis Yvinec!

Queffélec 1970, 6

Ha daoust ha ne oa ket bet tra spontusoc'h
c'hoazh~marv Loeiz Ivineg e dibenn ar bloaz tremenet?

Beyer 2016, 8

\subsection[Stumm
nac'h]{\texorpdfstring{\protect\hypertarget{Stumm_nac.27h}{}{}Stumm
nac'h}{Stumm nac'h}}\label{Stumm_nacux27h}

{{{[}}\href{/wiki/Stumm_kadarn,_nac\%27h,_goulennata,_estlamma\%C3\%B1?action=edit&section=4}{{modifier}}{{]}}}

\subsubsection[Stumm nac'h lakaet war wel gant
disteraennoù]{\texorpdfstring{\protect\hypertarget{Stumm_nac.27h_lakaet_war_wel_gant_disteraenno.C3.B9}{}{}Stumm
nac'h lakaet war wel gant
disteraennoù}{Stumm nac'h lakaet war wel gant disteraennoù}}\label{Stumm_nacux27h_lakaet_war_wel_gant_disteraennouxf9}

{{{[}}\href{/wiki/Stumm_kadarn,_nac\%27h,_goulennata,_estlamma\%C3\%B1?action=edit&section=5}{{modifier}}{{]}}}

Evit kreñvaat ar stumm nac'h e c'haller
implijout un \href{/wiki/Disteraenn}{disteraenn}.

Personne ne parlait.

Queffélec 1970, 48

Den ne ranne grik.

Beyer 2016, 29

La disparition d'Anne se conforme à une sagesse de
l'horreur: personne n'a rien vu ni rien
entendu et personne ne sait rien.

Queffélec 1970, 26

Reizhet e oa steuziadur Ann ouzh poell an euzh en ur mod: den
n'en doa gwelet na klevet tra, den ne ouie tra.

Beyer 2016, 8
