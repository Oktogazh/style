\chapter{Stumm kadarn, nac'h, goulennata, estlammañ}

\section{Stumm goulennata}\label{stumm_goulennata}

\subsection{\texorpdfstring{Stumm gant
\emph{ra}}{Stumm gant ra}}\label{stumm_gant_ra}

Ar stumm goulennata krouet gant an amginadur (inversion) e galleg a
c\textquotesingle haller treiñ gant \emph{ra} e bzg:

\subsection{Stumm kadarn pe estlammañ gallek cheñchet
d\textquotesingle ar stumm goulennata e
brezhoneg}\label{stumm_kadarn_pe_estlammauxf1_gallek_cheuxf1chet_dar_stumm_goulennata_e_brezhoneg}

Ur goulenn retorik pe dieeun a c\textquotesingle haller treiñ e
brezhoneg gant ur stagell isurzhiañ \emph{ha}. Neuze e
c\textquotesingle hall cheñch ar stumm kadarn pe estlammañ er frazenn
c\textquotesingle hallek d\textquotesingle ar stumm goulennata e
brezhoneg.

\section{Stumm nac\textquotesingle h}\label{stumm_nach}

\subsection{Stumm nac\textquotesingle h lakaet war wel gant
disteraennoù}\label{stumm_nach_lakaet_war_wel_gant_disteraennouxf9}

Evit kreñvaat ar stumm nac\textquotesingle h e c\textquotesingle haller
implijout un \url{disteraenn}.

\url{Category:Pennadoù}
