\chapter{Doare}

\section{Doare-divizout}\label{doare_divizout}

\subsection{\texorpdfstring{Doare-divizout implijet e brezhoneg evit
treiñ ur stumm gant \emph{si} diwar ar
galleg}{Doare-divizout implijet e brezhoneg evit treiñ ur stumm gant si diwar ar galleg}}\label{doare_divizout_implijet_e_brezhoneg_evit_treiuxf1_ur_stumm_gant_si_diwar_ar_galleg}

E galleg e implijer an amzer dremenet ledan pe an amzer a-vremañ goude
\emph{si}. E brezhoneg e c\textquotesingle haller implijout an
doare-divizout:

\section{Subjonctif}\label{subjonctif}

N\textquotesingle eus ket a subjonktif e brezhoneg. Doareoù disheñvel a
implijer e brezhoneg da eztaoler stadoù diwir evel ar
c\textquotesingle hoant, an het, ar bosublded, an ezhomm, ar varn, ar
sav-poent pe un obererezh ha n\textquotesingle eo ket
c\textquotesingle hoarvezet c\textquotesingle hoazh
\href{https://arbres.iker.cnrs.fr/index.php?title=Subjonctif}{Arbres:Subjonctif}.

\subsection{\texorpdfstring{Stumm gant rener + \emph{o} +
anv-verb}{Stumm gant rener + o + anv-verb}}\label{stumm_gant_rener_o_anv_verb}

\subsection{\texorpdfstring{Stumm gant \emph{da} + anv +
anv-verb}{Stumm gant da + anv + anv-verb}}\label{stumm_gant_da_anv_anv_verb}

\subsection{Talvoudegezh ar subjonktif degaset gant an doare-divizout
e
brezhoneg}\label{talvoudegezh_ar_subjonktif_degaset_gant_an_doare_divizout_e_brezhoneg}

\section{Lavarenn estlammañ}\label{lavarenn_estlammauxf1}

\subsection{\texorpdfstring{\emph{Na(g})}{Na(g)}}\label{nag}
