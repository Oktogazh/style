\chapter{Doare}

\section{Doare-divizout}\label{doare_divizout}

\subsection{\texorpdfstring{Doare-divizout implijet e brezhoneg evit
treiñ ur stumm gant \emph{si} diwar ar
galleg}{Doare-divizout implijet e brezhoneg evit treiñ ur stumm gant si diwar ar galleg}}\label{doare_divizout_implijet_e_brezhoneg_evit_treiuxf1_ur_stumm_gant_si_diwar_ar_galleg}

E galleg e implijer an amzer dremenet ledan pe an amzer a-vremañ goude
\emph{si}. E brezhoneg e c\textquotesingle haller implijout an
doare-divizout:

\begin{longtable}[]{@{}ll@{}}
\toprule\noalign{}
\endhead
\bottomrule\noalign{}
\endlastfoot
Ils s\textquotesingle assoient sur des mottes de terre un peu hautes et,
levant la tête, la tournent de droite à gauche et de gauche à droite
comme s\textquotesingle ils voulaient la dévisser. & Koazezañ a reont
war voudennoù uhelik ha, savet o fennoù, e troont anezho a gleiz da
zehoù hag a zehoù da gleiz e-giz pa glaskjent o diviñsañ. \\
Queffélec 1970, 6 & Beyer 2016, 8 \\
\end{longtable}

\section{Subjonctif}\label{subjonctif}

N\textquotesingle eus ket a subjonktif e brezhoneg. Doareoù disheñvel a
implijer e brezhoneg da eztaoler stadoù diwir evel ar
c\textquotesingle hoant, an het, ar bosublded, an ezhomm, ar varn, ar
sav-poent pe un obererezh ha n\textquotesingle eo ket
c\textquotesingle hoarvezet c\textquotesingle hoazh
\href{https://arbres.iker.cnrs.fr/index.php?title=Subjonctif}{Arbres:Subjonctif}.

\subsection{\texorpdfstring{Stumm gant rener + \emph{o} +
anv-verb}{Stumm gant rener + o + anv-verb}}\label{stumm_gant_rener_o_anv_verb}

\begin{longtable}[]{@{}ll@{}}
\toprule\noalign{}
\endhead
\bottomrule\noalign{}
\endlastfoot
Il n\textquotesingle est pas possible~que des aubes glorieuses, se
déployant dans le fond du ciel, aient éclairé ce morceau de récif. & Ne
c\textquotesingle hell ket bezañ bet~ruzelloù-mintin o sklêrijennañ an
tamm penn-karreg-se en o dispak glorius e don an oabl. \\
Queffélec 1970, 5 & Beyer 2016, 7 \\
\end{longtable}

\subsection{\texorpdfstring{Stumm gant \emph{da} + anv +
anv-verb}{Stumm gant da + anv + anv-verb}}\label{stumm_gant_da_anv_anv_verb}

\begin{longtable}[]{@{}ll@{}}
\toprule\noalign{}
\endhead
\bottomrule\noalign{}
\endlastfoot
Dieu était dans l\textquotesingle ile avant que n\textquotesingle y
débarquât le prêtre, mais comme le feu dans les rameaux avant que le
sauvage ne les frotte. & En enez edo Doue a-raok d\textquotesingle ar
beleg dilestrañ, met evel m\textquotesingle emañ an tan er skourroù
a-raok d\textquotesingle ar goueziad o frotañ. \\
Queffélec 1970, 33 & Beyer 2016, 13 \\
\end{longtable}

\subsection{Talvoudegezh ar subjonktif degaset gant an doare-divizout
e
brezhoneg}\label{talvoudegezh_ar_subjonktif_degaset_gant_an_doare_divizout_e_brezhoneg}

\begin{longtable}[]{@{}ll@{}}
\toprule\noalign{}
\endhead
\bottomrule\noalign{}
\endlastfoot
Les intentions défilaient, annoncées d\textquotesingle une voix
sifflante, pour la «~penzé~», pour que Dieu envoyât des épaves sur les
grèves, pour que Dieu fit tomber de l\textquotesingle eau, pour que Dieu
envoyât un prêtre dans son ile, pour que Dieu mit beaucoup de poissons
dans les filets ou dans les nasses. & Kemennet gant ur vouezh skiltr e
veze dibunet ar mennadoù pediñ gant ar peñse, ma vefe gant Doue kaset
peñse war an aodoù, ma vefe gant Doue lakaet glav
d\textquotesingle ober, ma vefe gant Doue kaset ur beleg war e enez, ma
vefe gant Doue lakaet pesked e-leizh er rouedoù pe er
c\textquotesingle hevell. \\
Queffélec 1970, 43 & Beyer 2016, 23 \\
\end{longtable}

\section{Lavarenn estlammañ}\label{lavarenn_estlammauxf1}

\subsection{\texorpdfstring{\emph{Na(g})}{Na(g)}}\label{nag}

\begin{longtable}[]{@{}ll@{}}
\toprule\noalign{}
\endhead
\bottomrule\noalign{}
\endlastfoot
Quel repos! & Nag~un distan! \\
Queffélec 1970, 6 & Beyer 2016, 8 \\
\end{longtable}

\begin{longtable}[]{@{}ll@{}}
\toprule\noalign{}
\endhead
\bottomrule\noalign{}
\endlastfoot
Quelle erreur d\textquotesingle accepter cette paroisse! & Nag ur fazi
kemer ar barrez-se! \\
Queffélec 1970, 28 & Beyer 2016, 10 \\
\end{longtable}

\begin{longtable}[]{@{}ll@{}}
\toprule\noalign{}
\endhead
\bottomrule\noalign{}
\endlastfoot
Quelle aubaine~! disait-on à François Guillerm. & ``Nag ur bevez!'' a
veze lavaret da Frañsez Gwilherm. \\
Queffélec 1970, 45 & Beyer 2016, 25 \\
\end{longtable}

\begin{longtable}[]{@{}ll@{}}
\toprule\noalign{}
\endhead
\bottomrule\noalign{}
\endlastfoot
Comme il en savait des choses~! & Nag a draoù a ouie! \\
Queffélec 1970, 48 & Beyer 2016, 29 \\
\end{longtable}
