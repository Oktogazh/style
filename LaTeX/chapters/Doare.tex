\begin{LTR}
\begin{otherlanguage}{breton}

\subsection{Doare-divizout}\label{Doare-divizout}

{{{[}}\href{/wiki/Doare?action=edit&section=1}{{modifier}}{{]}}}

\subsubsection[Doare-divizout implijet e brezhoneg evit treiñ ur stumm
gant \emph{si} diwar ar
galleg]{\texorpdfstring{\protect\hypertarget{Doare-divizout_implijet_e_brezhoneg_evit_trei.C3.B1_ur_stumm_gant_si_diwar_ar_galleg}{}{}Doare-divizout
implijet e brezhoneg evit treiñ ur stumm gant \emph{si} diwar ar
galleg}{Doare-divizout implijet e brezhoneg evit treiñ ur stumm gant si diwar ar galleg}}\label{Doare-divizout_implijet_e_brezhoneg_evit_treiuxf1_ur_stumm_gant_si_diwar_ar_galleg}

{{{[}}\href{/wiki/Doare?action=edit&section=2}{{modifier}}{{]}}}

E galleg e implijer an amzer dremenet ledan pe an amzer a-vremañ goude
\emph{si}. E brezhoneg e c'haller implijout an
doare-divizout:

Ils s'assoient sur des mottes de terre un peu hautes et,
levant la tête, la tournent de droite à gauche et de gauche à droite
comme s'ils voulaient la dévisser.

Queffélec 1970, 6

Koazezañ a reont war voudennoù uhelik ha, savet o fennoù, e troont
anezho a gleiz da zehoù hag a zehoù da gleiz e-giz pa glaskjent o
diviñsañ.

Beyer 2016, 8

\subsection{Subjonctif}\label{Subjonctif}

{{{[}}\href{/wiki/Doare?action=edit&section=3}{{modifier}}{{]}}}

N'eus ket a subjonktif e brezhoneg. Doareoù disheñvel a
implijer e brezhoneg da eztaoler stadoù diwir evel ar
c'hoant, an het, ar bosublded, an ezhomm, ar varn, ar
sav-poent pe un obererezh ha n'eo ket
c'hoarvezet c'hoazh
\href{https://arbres.iker.cnrs.fr/index.php?title=Subjonctif}{Arbres:Subjonctif}.

\subsubsection[Stumm gant rener + \emph{o} +
anv-verb]{\texorpdfstring{\protect\hypertarget{Stumm_gant_rener_.2B_o_.2B_anv-verb}{}{}Stumm
gant rener + \emph{o} +
anv-verb}{Stumm gant rener + o + anv-verb}}\label{Stumm_gant_rener_+_o_+_anv-verb}

{{{[}}\href{/wiki/Doare?action=edit&section=4}{{modifier}}{{]}}}

Il n'est pas possible~que des aubes glorieuses, se
déployant dans le fond du ciel, aient éclairé ce morceau de récif.

Queffélec 1970, 5

Ne c'hell ket bezañ bet~ruzelloù-mintin o sklêrijennañ
an tamm penn-karreg-se en o dispak glorius e don an oabl.

Beyer 2016, 7

\subsubsection[Stumm gant \emph{da} + anv +
anv-verb]{\texorpdfstring{\protect\hypertarget{Stumm_gant_da_.2B_anv_.2B_anv-verb}{}{}Stumm
gant \emph{da} + anv +
anv-verb}{Stumm gant da + anv + anv-verb}}\label{Stumm_gant_da_+_anv_+_anv-verb}

{{{[}}\href{/wiki/Doare?action=edit&section=5}{{modifier}}{{]}}}

Dieu était dans l'ile avant que n'y
débarquât le prêtre, mais comme le feu dans les rameaux avant que le
sauvage ne les frotte.

Queffélec 1970, 33

En enez edo Doue a-raok d'ar beleg dilestrañ, met evel
m'emañ an tan er skourroù a-raok d'ar
goueziad o frotañ.

Beyer 2016, 13

\subsubsection{Talvoudegezh ar subjonktif degaset gant an doare-divizout
e
brezhoneg}\label{Talvoudegezh_ar_subjonktif_degaset_gant_an_doare-divizout_e_brezhoneg}

{{{[}}\href{/wiki/Doare?action=edit&section=6}{{modifier}}{{]}}}

Les intentions défilaient, annoncées d'une voix
sifflante, pour la «~penzé~», pour que Dieu envoyât des épaves sur les
grèves, pour que Dieu fit tomber de l'eau, pour que Dieu
envoyât un prêtre dans son ile, pour que Dieu mit beaucoup de poissons
dans les filets ou dans les nasses.

Queffélec 1970, 43

Kemennet gant ur vouezh skiltr e veze dibunet ar mennadoù pediñ gant ar
peñse, ma vefe gant Doue kaset peñse war an aodoù, ma vefe gant Doue
lakaet glav d'ober, ma vefe gant Doue kaset ur beleg war
e enez, ma vefe gant Doue lakaet pesked e-leizh er rouedoù pe er
c'hevell.

Beyer 2016, 23

\subsection[Lavarenn
estlammañ]{\texorpdfstring{\protect\hypertarget{Lavarenn_estlamma.C3.B1}{}{}Lavarenn
estlammañ}{Lavarenn estlammañ}}\label{Lavarenn_estlammauxf1}

{{{[}}\href{/wiki/Doare?action=edit&section=7}{{modifier}}{{]}}}

\subsubsection[\emph{Na(g})]{\texorpdfstring{\protect\hypertarget{Na.28g.29}{}{}\emph{Na(g})}{Na(g)}}\label{Naux28gux29}

{{{[}}\href{/wiki/Doare?action=edit&section=8}{{modifier}}{{]}}}

Quel repos!

Queffélec 1970, 6

Nag~un distan!

Beyer 2016, 8

Quelle erreur d'accepter cette paroisse!

Queffélec 1970, 28

Nag ur fazi kemer ar barrez-se!

Beyer 2016, 10

Quelle aubaine~! disait-on à François Guillerm.

Queffélec 1970, 45

``Nag ur bevez!'' a veze lavaret da Frañsez Gwilherm.

Beyer 2016, 25

Comme il en savait des choses~!

Queffélec 1970, 48

Nag a draoù a ouie!

Beyer 2016, 29

\end{otherlanguage}
\end{LTR}
