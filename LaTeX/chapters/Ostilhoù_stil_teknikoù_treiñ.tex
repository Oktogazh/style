\chapter{Ostilhoù stil, teknikoù treiñ}

\section{Teknikoù treiñ hervez Vinay ha
Darbelnet}\label{teknikouxf9_treiuxf1_hervez_vinay_ha_darbelnet}

Goude bezañ displeget meizadoù diazez an droidigezh er pennad kent, e
troomp bremañ war-du teknikoù treiñ disheñvel renablet gant Vinay ha
Darbelnet:

\begin{longtable}[]{@{}lll@{}}
\caption{Strategiezhioù treiñ pennañ}\tabularnewline
\toprule\noalign{}
Rummad & Testenn orin & Toidigezh \\
\midrule\noalign{}
\endfirsthead
\toprule\noalign{}
Rummad & Testenn orin & Toidigezh \\
\midrule\noalign{}
\endhead
\bottomrule\noalign{}
\endlastfoot
\url{Amprest} & Pennsoner ar bagad eo. & C\textquotesingle est le
pennsoner du bagad. \\
\url{Kalk} & Senniñ a rae ar bombard. & Il sonnait la bombarde. \\
\href{Troidigezh_ger-ha-ger}{Troidigezh ger-ha-ger} & Elle joue de la
guitare. & Senniñ a ra ar gitar. \\
\url{Trañspozadur} & La mer avait fini de descendre. & Echu e oa e
ziskenn gant ar mor. \\
\url{Moduladur} & Ils rient des vaches qui courent. & Ar saout o redek a
ra dezho c\textquotesingle hoarzhin. \\
\url{Kevatalded} & Pleurer à chaudes larmes. & Gouelañ dourek. \\
\url{Azasadur} & Pennsoner ar bagad eo. & C\textquotesingle est le chef
d\textquotesingle orchestre. \\
\end{longtable}

\section{An amprest}\label{an_amprest}

E-touez an holl deknikoù treiñ eo an amprest an hini simplañ. Anvet eo
ivez \emph{treuzkas}. Derc\textquotesingle hel un termen pe un droienn
er yezh orin hep e dreiñ a ya d\textquotesingle ober anezhañ
\href{Rottet_&_Moris_2018}{Rottet \& Moris 2018}, 11). Implijout a
c\textquotesingle haller an amprest evit krouiñ un efed stilistikel. Da
skouer, e c\textquotesingle haller komz eus \emph{tatami} pe
\emph{samourai} e Japan, \emph{pizza} pe \emph{raviolli} en Itali,
\emph{corrida} pe \emph{fiesta} e Bro-Spagn, \emph{cowboy} pe
\emph{sheriff} er Stadoù-Unanet, hag all, evit treuzkas ur spered
lec\textquotesingle hel pe arlivioù sevenadurel resis.

Niverus eo gerioù a orin brezhonek deut en implij boutin e galleg, e
Breizh hag e lec\textquotesingle h all. En o zouez e kaver termenoù evel
\emph{bagad, menhir, gwenn-ha-du, fest-noz, korrigan, binioù} hag all.

Ar pezh a zedenn an troer eo dreist-holl an amprestoù nevez pe zoken ar
re bersonel. Da skouer, e c\textquotesingle hellfed implijout termenoù
evel \emph{taksi} evit chom tost ouzh ar bed modern:

An amprestoù kozh a emdro o ster gant an amzer hag e
c\textquotesingle hallont treiñ da vignoned faos. Un nebeud gerioù
brezhonek a c\textquotesingle hallfe evel-se touellañ gallegerien
a-vihanik, e.g. \emph{mignon} a zeu eus ar galleg, ha koulskoude en deus
kollet ar ger e ster \emph{moutik}.

\section{Ar c\textquotesingle halk}\label{ar_chalk}

Ur seurt amprest ispisial diwar ur yezh all eo ar
c\textquotesingle halk. E-lec\textquotesingle h enporzhiañ ur ger estren
evel m\textquotesingle emañ er yezh orin e vez troet an termen-se,
elfenn da elfenn, er yezh tal. Da skouer eo ar ger \emph{dibenn-sizhun}
ur c'halk eus \emph{week-end} saoznek
(\href{Darbelnet_&_Vinay_1993}{Darbelnet \& Vinay 1993}, 6), erruet e
brezhoneg dre \emph{fin de semaine} gallek.

Hervez Vinay \& Darbelnet ez eus daou seurt kalkoù. \emph{Ar
c\textquotesingle halk ezteurel} a ginnig un doare nevez da frazennañ an
traoù, en ur zoujañ avat ouzh urzh ar gerioù boutin er yezh tal. Ur
skouer eo \emph{chemin de fer}, krouet diwar \emph{railroad} e saozneg.
Ar ger \emph{hent-houarn} brezhonek a zo ur c'halk eeun eus an termen
gallek. \emph{Ar c\textquotesingle halk framm}, diouzh un tu all, a
ginnig ur patrom nevez a-fed urzh ar gerioù er yezh tal, evel
\emph{skiant-faltazi} erruet e brezhoneg dre ar saozneg.

E degouezhioù zo e implijer ar c\textquotesingle halk dre fazi. Hervez
\href{Gourmelon_2012}{Gourmelon (2012, 63)}, eo ur fazi boutin implijout
div araogenn dirak ar memes anv, e-giz e gallek: \emph{avec ou sans son
chien, pendant et après le marché, sur et sous la table}. Hervezañ, ne
vez ket lavaret e brezhoneg \textbf{gant pe hep e gi}, \emph{e-pad ha
goude ar marc\textquotesingle had, war ha dindan an daol}, abalamour ma
ne c\textquotesingle heller ket implijout \emph{gant}, \emph{war} hag
\emph{e-pad} evel adverboù. Kinnig a ra frammañ frazennoù gant an
araogennoù-mañ kentoc\textquotesingle h evel-se:

\begin{itemize}
\tightlist
\item
  Gwelet \textquotesingle poa anezhañ gant e gi pe heptañ?
\item
  Klasket \textquotesingle m eus war an daol ha dindan (dindani).
\item
  Komzet \textquotesingle poa outañ e-pad ar marc\textquotesingle had pe
  goude?
\end{itemize}

Ur fazi boutin all eo treiñ \emph{un peu} diwar ar galleg evel \emph{un
tammig} e brezhoneg e degouezhioù ma ranker lavaret ur bannig. Da
skouer, \emph{un tamm bara}, met \emph{ur banne dour}.

Evit echuiñ, en droidigezh lennegel e c\textquotesingle haller implijout
a-dra-sur kalkennoù a zo anezho dija, met un droer evezhiek a rankfe
diwall pa ginnig ur c\textquotesingle halk nevez er yezh e-barzh un
droidigezh (\href{Rottet_&_Moris_2018}{Rottet \& Moris 2018}, 15).

\section{An droidigezh ger-ha-ger}\label{an_droidigezh_ger_ha_ger}

En droidigezh ger-ha-ger e troer un destenn orin en ur yezh tal en un
doare reizh hag idiomatek, hep derc\textquotesingle hel e kont tra ebet
all en tu all d\textquotesingle ar frammoù yezhoniel. Ar skouerioù
kinniget aman-dindan a zo boutin a-walc\textquotesingle h:

\begin{itemize}
\tightlist
\item
  J\textquotesingle ai bu un café. \textgreater{} Evet em eus ur banne
  kafe.
\item
  Pelec\textquotesingle h emaoc\textquotesingle h? \textgreater{} Où
  êtes-vous?
\item
  Il a rencontré une personne parlant breton. \textgreater{} Kejet en
  deus gant un den hag a gomze brezhoneg.
\end{itemize}

Er skouerioù-mañ eo doujet holl reolennoù yezhadurel ar brezhoneg, ha
n\textquotesingle eus ket kalz doareoù all da dreiñ ar frazennoù
kinniget. Er c\textquotesingle hontrol d\textquotesingle ar
c\textquotesingle halkoù, pa seller ouzh un droidigezh ger-ha-ger a
seurt-se, n\textquotesingle eo ket anat eo un droidigezh an destenn
ginniget; ar frazennoù a seblant naturel er yezh tal.

Koulskoude, mard eo direizh an droidigezh ger-ha-ger, e ranker implijout
un doare all da dreiñ. Un droidigezh ger-ha-ger a vez sellet outi evel
dibosupl pa dreuzkas ur ster disheñvel pe eo dister, pa vez direizh
a-fet yezhadur, pa ne glot ket gant reolennoù ar yezh tarzh, pe pa
implij ul live yezh disheñvel da-geñver ar yezh orin.

Da skouer e c\textquotesingle haller treiñ an droienn
\emph{c\textquotesingle hoari mell-droad} e galleg diwar ar brezhoneg
evel \emph{jouer au foot}, met \emph{c\textquotesingle hoari biz meud}
ne c\textquotesingle heller ket treiñ evel \emph{jouer du pouce}. Un
droidigezh c\textquotesingle hallek natureloc\textquotesingle h evel
\emph{faire du stop} a dreuzkasfe ur ster kevatal, anavezet gant an
troer diouzh ur sav-poent diavaez d\textquotesingle ar yezh orin ha
d\textquotesingle ar yezh tal.

Dre vras ne implijer ket kalkoù evit treiñ elfennoù ar yezh frazeologel,
evel troiù-lavar pe krennlavaroù. Dre se, an droienn \emph{en tenue de
Noé} er skouer amañ-dindan a voe troet evel \emph{noazh-pilh} evit
treuzkas ar mennozh-se:

Koulskoude, e degouezhioù zo, e c\textquotesingle haller kalkiñ
troioù-lavar zo war-eeun eus ar galleg d\textquotesingle ar brezhoneg.
Da skouer e lavarer \emph{kouezhañ en avaloù} e brezhoneg, pehini a zo
ur c\textquotesingle halk diwar \emph{tomber dans les pommes} gallek.

\section{Trañspozadur}\label{trauxf1spozadur}

An trañspozadur a zo un teknik treiñ ma vez erlec\textquotesingle hiet
unan eus kevrennoù ar prezeg pe eus rummadoù gerioù gant egile etre ar
yezh orin hag ar yezh tal hep cheñch ster ar gemennadenn
(\href{Rottet_&_Moris_2018}{Rottet \& Moris 2018}, 15).

Da skouer, er frazenn da heul e voe erlec\textquotesingle hiet verb
\emph{descendre} an destenn orin gant an anv-kadarn \emph{diskenn} er
yezh tal:

\section{Moduladur}\label{moduladur}

Ar moduladur a zo un teknik treiñ hag a ziskouez ur sav-poent disheñvel
war un darvoud evit ezteurel ar memes mennozh en ur
zerc\textquotesingle hel e ster (\href{Rottet_&_Moris_2018}{Rottet \&
Moris 2018}, 19).

Er skouerioù kinniget amañ e weler e voe cheñchet ar rener etre ar yezh
orin hag ar yezh tal, met chomet eo digemm ster ar gemennadenn:

\section{Kevatalded}\label{kevatalded}

Hervez \href{Rottet_&_Moris_2018}{Rottet ha Moris (2018, 25)} e servij
ar gevatalded dreist-holl da dreuzkas troiennoù eus ar yezh frazeologel,
evel ar c\textquotesingle hrennlavaroù, ar c\textquotesingle hlichedoù
hag an troioù-lavar. En degouezhioù-se eo kefridi an troer kavout un
droienn a glot a-fet ster gant ar yezh tal. Sellomp ouzh skouerioù
kevatalded tennet eus ar c\textquotesingle horpus troidigezhioù.

Amañ e voe troet \emph{aux trois quarts sourd} evel
\emph{bouzar-kloc'h}:

Er frazenn da heul e voe treuzkaset mennozh \emph{pleurer à chaudes
larmes} gant an droienn\emph{gouelañ dourek}:

\emph{Ouvert à tous les vents} a voe troet amañ evel \emph{digor
d\textquotesingle ar pevar avel}:

Evit treiñ an droienn c\textquotesingle hallek \emph{de but en blanc} e
lakas an droerez an droienn gevatal \emph{krak-ha-berr}:

\emph{Regarder à la dérobée} a c\textquotesingle haller treiñ evel
\emph{sellet dre laer}:

\section{Azasadur}\label{azasadur}

An azasadur a zo ur strategiezh treiñ a implijer evit treuzkas elfennoù
n\textquotesingle eus ket anezho e sevenadur ar yezh tal
\href{Rottet_&_Moris_2018}{(Rottet ha Moris 2018, 26)}.

Ar skouer amañ-dindan a ziskouez penaos e c\textquotesingle haller
treuzkas unan eus ar modoù boutin da saludiñ an dud e Breizh en ur
implijout an azasadur:

Implijout an azasadur a c\textquotesingle haller ivez evit treuzkas
elfennoù lec\textquotesingle hel pe rannyezhel:

\section{Teknikoù treiñ all}\label{teknikouxf9_treiuxf1_all}

Ouzhpenn ar strategiezhioù treiñ diazez kinniget gant Darbelnet ha Vinay
e plijfe din eztaoler un nebeud teknikoù all a gavan talvoudus.

\section{Astenn}\label{astenn}

An teknik-mañ a servij da resisaat ster ur ger en destenn troet en ur
ouzhpennañ gerioù war e lerc\textquotesingle h. Talvoudus e
c\textquotesingle hall bezañ pa vez liesster ur ger er yezh tal ha neket
er yezh orin. Da skouer, evit treiñ ar ger \emph{réservoire} e saozneg
diwar ar galleg, eo gwelloc\textquotesingle h lavaret \emph{water tank}
hag argas an amsterder a zegasfe ar ger \emph{tank} e- unan, pe hini a
lakafe da soñjal er c\textquotesingle harr milourel.

\section{Treiñ dre nac\textquotesingle hañ an
enep-ster}\label{treiuxf1_dre_nachauxf1_an_enep_ster}

Treiñ dre nac\textquotesingle hañ an enep-ster a zo un doare da dreiñ un
elfenn gadarn diwar ar yezh orin gant un elfenn nac\textquotesingle h er
yezh tal, pe er c\textquotesingle hontrol, hep cheñch ster ar frazenn
orin.

\section{An didroc\textquotesingle hañ}\label{an_didrochauxf1}

An didroc'hañ a zo un teknik treiñ lec\textquotesingle h ma ranner ar
frazenn orin e div frazenn pe muioc\textquotesingle h frazennoù
bihanoc\textquotesingle h.

Ar frazenn-mañ zo bet didroc\textquotesingle het e div evit resisaat al
liamm etre ar rener \emph{le prêtre} hag ar renadenn dra ameeun \emph{la
piété de ses ouailles}:

Ma vije dalc\textquotesingle het stumm ar frazenn orin en droidigezh
vrezhonek ne vefe ket sklaer petra zo o sevel
\emph{dizalc\textquotesingle h, diamen ha glan} - \emph{deoliezh e
zeñved} pe \emph{an heol}.

E galleg eo anat e komzer diwar-benn \emph{deoliezh} a-drugarez da reizh
gwregel ar ger \emph{piété} : ne vijet ket bet posupl lavaret \emph{le
soleil lointaine}.

E brezhoneg n\textquotesingle eo ket anat reizh ar ger \emph{deoliezh} e
kenarroud-mañ, rak n\textquotesingle eus kemmadur ebet goude \emph{o
sevel}.

An doare disheñvel da ziskouez al liamm-se a vefe ouzhpennañ \emph{hag}:

Edo ar beleg o soñjal e deoliezh e zeñved o tivogediñ er vrumenn ha
dindan an heol {[}hag{]} o sevel, dizalc\textquotesingle h, diamen ha
glan.

Marteze he deus dibabet an droerez troc\textquotesingle hañ ar frazenn e
div evit chom hep adlavaret \emph{ha} teir gwech : {[}\ldots{]} ha
dindan an heol hag o sevel, dizalc\textquotesingle h, diamen ha glan.

E brezhoneg e c\textquotesingle hall ar ger \emph{seizet} bezañ un anv
pe un anv-gwan. Evit diskouez eo \emph{seizet ha
bouzar-kloc\textquotesingle h} an tad-kozh ha n\textquotesingle eo ket
ar gwele e voe krouet div frazenn en droidigezh vrezhonek:

\section{Ar stagañ}\label{ar_stagauxf1}

Ar stagañ eo kontrol an didroc\textquotesingle hañ.

Amañ eo bet kendeuzet un nebeud frazennoù gallek en ur frazenn nemeti en
droidigezh vrezhonek.

Strizh ar c\textquotesingle henarroud evit kregiñ ur frazenn e brezhoneg
gant ur rannig-verb.

Posupl e vije lakaat ur pik ha kregiñ al lavarenn nevez gant an anv, an
anv-gwan hag ar verb:

{[}\ldots{]} bugale sentus ha doujus e oant, eürus e oant {[}\ldots{]},
ober a rafent o seizh gwellañ {[}\ldots{]}.

\section{Ostilhoù stil}\label{ostilhouxf9_stil}

Ar bazenn gentañ da astenn ar pennad-mañ a vefe kinnig ostilhoù stil
disheñvel e brezhoneg, evel ledanvad hag all. Pouezus-bras eo studiañ
pennad \href{Le_Berre_2011}{Le Berre 2011} war al litote.

\section{Ledanvad}\label{ledanvad}

Er skouer kinniget amañ dindan e komzer diwar-benn an dud e-barzh an
iliz, ha neket ar savadur:

Hag en degouezh-mañ ez eus un hollekadur pa reer kaoz eus strollad an
dud a zo perzh d\textquotesingle ar barrez:
