\chapter{Ostilhoù stil, teknikoù treiñ}
\section[Teknikoù treiñ hervez Vinay ha
Darbelnet]{\texorpdfstring{\protect\hypertarget{Tekniko.C3.B9_trei.C3.B1_hervez_Vinay_ha_Darbelnet}{}{}Teknikoù
treiñ hervez Vinay ha
Darbelnet}{Teknikoù treiñ hervez Vinay ha Darbelnet}}\label{Teknikouxf9_treiuxf1_hervez_Vinay_ha_Darbelnet}

{{{[}}\href{/wiki/Ostilho\%C3\%B9_stil,_tekniko\%C3\%B9_trei\%C3\%B1?action=edit&section=1}{{modifier}}{{]}}}

Goude bezañ displeget meizadoù diazez an droidigezh er pennad kent, e
troomp bremañ war-du teknikoù treiñ disheñvel renablet gant Vinay ha
Darbelnet:

\begin{longtable}[]{@{}lll@{}}
\caption{Strategiezhioù treiñ pennañ}\tabularnewline
\toprule\noalign{}
\endfirsthead
\endhead
\bottomrule\noalign{}
\endlastfoot
Rummad & Testenn orin & Toidigezh \\
\href{/wiki/Amprest}{Amprest} & Pennsoner ar bagad eo. &
C'est le pennsoner du bagad. \\
\href{/wiki/Kalk}{Kalk} & Senniñ a rae ar bombard. & Il sonnait la
bombarde. \\
\href{/wiki/Troidigezh_ger-ha-ger}{Troidigezh ger-ha-ger} & Elle joue de
la guitare. & Senniñ a ra ar gitar. \\
\href{/wiki/Tra\%C3\%B1spozadur}{Trañspozadur} & La mer avait fini de
descendre. & Echu e oa e ziskenn gant ar mor. \\
\href{/wiki/Moduladur}{Moduladur} & Ils rient des vaches qui courent. &
Ar saout o redek a ra dezho c'hoarzhin. \\
\href{/wiki/Kevatalded}{Kevatalded} & Pleurer à chaudes larmes. &
Gouelañ dourek. \\
\href{/wiki/Azasadur}{Azasadur} & Pennsoner ar bagad eo. &
C'est le chef d'orchestre. \\
\end{longtable}

\subsection{An amprest}\label{An_amprest}

{{{[}}\href{/wiki/Ostilho\%C3\%B9_stil,_tekniko\%C3\%B9_trei\%C3\%B1?action=edit&section=2}{{modifier}}{{]}}}

E-touez an holl deknikoù treiñ eo an amprest an hini simplañ. Anvet eo
ivez \emph{treuzkas}. Derc'hel un termen pe un droienn
er yezh orin hep e dreiñ a ya d'ober anezhañ
\href{/wiki/Rottet_\%26_Moris_2018}{Rottet \& Moris 2018}, 11).
Implijout a c'haller an amprest evit krouiñ un efed
stilistikel. Da skouer, e c'haller komz eus
\emph{tatami} pe \emph{samourai} e Japan, \emph{pizza} pe
\emph{raviolli} en Itali, \emph{corrida} pe \emph{fiesta} e Bro-Spagn,
\emph{cowboy} pe \emph{sheriff} er Stadoù-Unanet, hag all, evit treuzkas
ur spered lec'hel pe arlivioù sevenadurel resis.

Niverus eo gerioù a orin brezhonek deut en implij boutin e galleg, e
Breizh hag e lec'h all. En o zouez e kaver termenoù evel
\emph{bagad, menhir, gwenn-ha-du, fest-noz, korrigan, binioù} hag all.

Ar pezh a zedenn an troer eo dreist-holl an amprestoù nevez pe zoken ar
re bersonel. Da skouer, e c'hellfed implijout termenoù
evel \emph{taksi} evit chom tost ouzh ar bed modern:

Kemerit un taksi.

Drezen 2012, 183

Prenez un taxi.

Drezen 2002, 114

An amprestoù kozh a emdro o ster gant an amzer hag e
c'hallont treiñ da vignoned faos. Un nebeud gerioù
brezhonek a c'hallfe evel-se touellañ gallegerien
a-vihanik, e.g. \emph{mignon} a zeu eus ar galleg, ha koulskoude en deus
kollet ar ger e ster \emph{moutik}.

\subsection[Ar
c'halk]{\texorpdfstring{\protect\hypertarget{Ar_c.27halk}{}{}Ar
c'halk}{Ar c'halk}}\label{Ar_cux27halk}

{{{[}}\href{/wiki/Ostilho\%C3\%B9_stil,_tekniko\%C3\%B9_trei\%C3\%B1?action=edit&section=3}{{modifier}}{{]}}}

Ur seurt amprest ispisial diwar ur yezh all eo ar
c'halk. E-lec'h enporzhiañ ur ger estren
evel m'emañ er yezh orin e vez troet an termen-se,
elfenn da elfenn, er yezh tal. Da skouer eo ar ger \emph{dibenn-sizhun}
ur c'halk eus \emph{week-end} saoznek
(\href{/wiki/Darbelnet_\%26_Vinay_1993}{Darbelnet \& Vinay 1993}, 6),
erruet e brezhoneg dre \emph{fin de semaine} gallek.

Hervez Vinay \& Darbelnet ez eus daou seurt kalkoù. \emph{Ar
c'halk ezteurel} a ginnig un doare nevez da frazennañ an
traoù, en ur zoujañ avat ouzh urzh ar gerioù boutin er yezh tal. Ur
skouer eo \emph{chemin de fer}, krouet diwar \emph{railroad} e saozneg.
Ar ger \emph{hent-houarn} brezhonek a zo ur c'halk eeun eus an termen
gallek. \emph{Ar c'halk framm}, diouzh un tu all, a
ginnig ur patrom nevez a-fed urzh ar gerioù er yezh tal, evel
\emph{skiant-faltazi} erruet e brezhoneg dre ar saozneg.

E degouezhioù zo e implijer ar c'halk dre fazi. Hervez
\href{/wiki/Gourmelon_2012}{Gourmelon (2012, 63)}, eo ur fazi boutin
implijout div araogenn dirak ar memes anv, e-giz e gallek: \emph{avec ou
sans son chien, pendant et après le marché, sur et sous la table}.
Hervezañ, ne vez ket lavaret e brezhoneg \textbf{gant pe hep e gi},
\emph{e-pad ha goude ar marc'had, war ha dindan an
daol}, abalamour ma ne c'heller ket implijout
\emph{gant}, \emph{war} hag \emph{e-pad} evel adverboù. Kinnig a ra
frammañ frazennoù gant an araogennoù-mañ kentoc'h
evel-se:

\begin{itemize}
\tightlist
\item
  Gwelet 'poa anezhañ gant e gi pe heptañ?
\item
  Klasket 'm eus war an daol ha dindan (dindani).
\item
  Komzet 'poa outañ e-pad ar marc'had pe
  goude?
\end{itemize}

Ur fazi boutin all eo treiñ \emph{un peu} diwar ar galleg evel \emph{un
tammig} e brezhoneg e degouezhioù ma ranker lavaret ur bannig. Da
skouer, \emph{un tamm bara}, met \emph{ur banne dour}.

Evit echuiñ, en droidigezh lennegel e c'haller implijout
a-dra-sur kalkennoù a zo anezho dija, met un droer evezhiek a rankfe
diwall pa ginnig ur c'halk nevez er yezh e-barzh un
droidigezh (\href{/wiki/Rottet_\%26_Moris_2018}{Rottet \& Moris 2018},
15).

\subsection{An droidigezh ger-ha-ger}\label{An_droidigezh_ger-ha-ger}

{{{[}}\href{/wiki/Ostilho\%C3\%B9_stil,_tekniko\%C3\%B9_trei\%C3\%B1?action=edit&section=4}{{modifier}}{{]}}}

En droidigezh ger-ha-ger e troer un destenn orin en ur yezh tal en un
doare reizh hag idiomatek, hep derc'hel e kont tra ebet
all en tu all d'ar frammoù yezhoniel. Ar skouerioù
kinniget aman-dindan a zo boutin a-walc'h:

\begin{itemize}
\tightlist
\item
  J'ai bu un café. \textgreater{} Evet em eus ur banne
  kafe.
\item
  Pelec'h emaoc'h? \textgreater{} Où
  êtes-vous?
\item
  Il a rencontré une personne parlant breton. \textgreater{} Kejet en
  deus gant un den hag a gomze brezhoneg.
\end{itemize}

Er skouerioù-mañ eo doujet holl reolennoù yezhadurel ar brezhoneg, ha
n'eus ket kalz doareoù all da dreiñ ar frazennoù
kinniget. Er c'hontrol d'ar
c'halkoù, pa seller ouzh un droidigezh ger-ha-ger a
seurt-se, n'eo ket anat eo un droidigezh an destenn
ginniget; ar frazennoù a seblant naturel er yezh tal.

Koulskoude, mard eo direizh an droidigezh ger-ha-ger, e ranker implijout
un doare all da dreiñ. Un droidigezh ger-ha-ger a vez sellet outi evel
dibosupl pa dreuzkas ur ster disheñvel pe eo dister, pa vez direizh
a-fet yezhadur, pa ne glot ket gant reolennoù ar yezh tarzh, pe pa
implij ul live yezh disheñvel da-geñver ar yezh orin.

Da skouer e c'haller treiñ an droienn
\emph{c'hoari mell-droad} e galleg diwar ar brezhoneg
evel \emph{jouer au foot}, met \emph{c'hoari biz meud}
ne c'heller ket treiñ evel \emph{jouer du pouce}. Un
droidigezh c'hallek natureloc'h evel
\emph{faire du stop} a dreuzkasfe ur ster kevatal, anavezet gant an
troer diouzh ur sav-poent diavaez d'ar yezh orin ha
d'ar yezh tal.

Dre vras ne implijer ket kalkoù evit treiñ elfennoù ar yezh frazeologel,
evel troiù-lavar pe krennlavaroù. Dre se, an droienn \emph{en tenue de
Noé} er skouer amañ-dindan a voe troet evel \emph{noazh-pilh} evit
treuzkas ar mennozh-se:

Le temps de reprendre les vieilles toiles et de laisser les bonshommes
en tenue de Noé, avec leurs bandeaux sur les yeux comme pour les
empêcher de souffrir dans leur pudeur, et ils filaient.

Queffélec 1970, 45

Amzer da adtapout al lien kozh ha da lezel ar baotred en noazh-pilh-ran,
mouchet o daoulagad evel evit mirout outo da gaout mezh, hag e oant
fustet.

Beyer 2016, 25

D'avoir vu et revu tout nus des chrétiens bien vivants.

Queffélec 1970, 45

Bezañ gwelet hag adwelet en o noazh-pilh daou gristen bev-mat.

Beyer 2016, 25

Koulskoude, e degouezhioù zo, e c'haller kalkiñ
troioù-lavar zo war-eeun eus ar galleg d'ar brezhoneg.
Da skouer e lavarer \emph{kouezhañ en avaloù} e brezhoneg, pehini a zo
ur c'halk diwar \emph{tomber dans les pommes} gallek.

\subsection[Trañspozadur]{\texorpdfstring{\protect\hypertarget{Tra.C3.B1spozadur}{}{}Trañspozadur}{Trañspozadur}}\label{Trauxf1spozadur}

{{{[}}\href{/wiki/Ostilho\%C3\%B9_stil,_tekniko\%C3\%B9_trei\%C3\%B1?action=edit&section=5}{{modifier}}{{]}}}

An trañspozadur a zo un teknik treiñ ma vez erlec'hiet
unan eus kevrennoù ar prezeg pe eus rummadoù gerioù gant egile etre ar
yezh orin hag ar yezh tal hep cheñch ster ar gemennadenn
(\href{/wiki/Rottet_\%26_Moris_2018}{Rottet \& Moris 2018}, 15).

Da skouer, er frazenn da heul e voe erlec'hiet verb
\emph{descendre} an destenn orin gant an anv-kadarn \emph{diskenn} er
yezh tal:

La mer avait fini de descendre, qu'on entendait frapper
les récifs dans des heurts rythmés et vaillants, ou glisser contre des
bancs de pierres.

Queffélec 1970, 51

Echu e oa e ziskenn gant ar mor a veze klevet o skeiñ ouzh ar
c'herreg gant stokadennoù kellusket ha kadarn, pe o
ruzañ ouzh bankoù mein.

Beyer 2016, 31

\subsection{Moduladur}\label{Moduladur}

{{{[}}\href{/wiki/Ostilho\%C3\%B9_stil,_tekniko\%C3\%B9_trei\%C3\%B1?action=edit&section=6}{{modifier}}{{]}}}

Ar moduladur a zo un teknik treiñ hag a ziskouez ur sav-poent disheñvel
war un darvoud evit ezteurel ar memes mennozh en ur
zerc'hel e ster
(\href{/wiki/Rottet_\%26_Moris_2018}{Rottet \& Moris 2018}, 19).

Er skouerioù kinniget amañ e weler e voe cheñchet ar rener etre ar yezh
orin hag ar yezh tal, met chomet eo digemm ster ar gemennadenn:

Ils rient des vaches qui courent.

Queffélec 1970, 6

Ar saout o redek a ra dezho c'hoarzhin.

Beyer 2016, 8

Il ne pouvait pas dire que les îliens manquaient de foi.

Queffélec 1970, 29

Ne c'helle ket lavarout e fazie ar feiz
d'an enezourien.

Beyer 2016, 12

\subsection{Kevatalded}\label{Kevatalded}

{{{[}}\href{/wiki/Ostilho\%C3\%B9_stil,_tekniko\%C3\%B9_trei\%C3\%B1?action=edit&section=7}{{modifier}}{{]}}}

Hervez \href{/wiki/Rottet_\%26_Moris_2018}{Rottet ha Moris (2018, 25)} e
servij ar gevatalded dreist-holl da dreuzkas troiennoù eus ar yezh
frazeologel, evel ar c'hrennlavaroù, ar
c'hlichedoù hag an troioù-lavar. En degouezhioù-se eo
kefridi an troer kavout un droienn a glot a-fet ster gant ar yezh tal.
Sellomp ouzh skouerioù kevatalded tennet eus ar c'horpus
troidigezhioù.

Amañ e voe troet \emph{aux trois quarts sourd} evel
\emph{bouzar-kloc'h}:

Thomas s'approchait du lit clos de son grand-père, le
paralytique, aux trois quarts sourd, et qui s'était
gardé d'arrêter sa besogne, la confection
d'un panier d'osier, pour une scène à
quoi il ne comprenait rien.

Queffélec 1970, 35

Tostaet e oa Tomaz ouzh gwele-kloz e dad-kozh. Seizet ha
bouzar-kloc'h en doa hennezh eveshaet a baouez gant e
labour fardañ ur baner evit heuliañ ur gaoz na gomprene seurt ebet
dezhi.

Beyer 2016, 15

Er frazenn da heul e voe treuzkaset mennozh \emph{pleurer à chaudes
larmes} gant an droienn \emph{gouelañ dourek}:

Une heure ensuite, lorsque Guillaume Gourvennec, le père de Thomas,
prenait la tête des iliens et remerciait le prêtre du bonheur
qu'il leur apportait à tous, Thomas, qui pleurait à
chaudes larmes, avait oublié qu'il appartenait à une
mauvaise famille.

Queffélec 1970, 35

Un eurvezh diwezhatoc'h, pa voe Gwilhom Gourvenneg, tad
Tomaz, e penn an enezourien da drugarekaat ar beleg evit an eurvad a
zegase dezho-holl, en doa disoñjet Tomaz, o ouelañ dourek, e oa ezel eus
un tiegezh fall.

Beyer 2016, 15-16

\emph{Ouvert à tous les vents} a voe troet amañ evel \emph{digor
d'ar pevar avel}:

Sur l'île ouverte à tous les vents et que la nuit calme
de septembre récompensait si bien, il tâchait de transformer son âme en
un lieu d'accueil, en une chambre basse et familière où
entrerait et se reposerait l'esprit de Dieu.

Queffélec 1970, 52

War an enez digor d'ar pevar avel ha digollet ken kaer
gant noz sioul miz Gwengolo edo o klask lakaat e ene da vezañ ul
lec'h degemer, ur gambr izel hag anavezet ma teufe ha ma
tiskuizhfe spered Doue.

Beyer 2016, 32

Evit treiñ an droienn c'hallek \emph{de but en blanc} e
lakas an droerez an droienn gevatal \emph{krak-ha-berr}:

Le jeudi, Thomas se rendit chez Guillerm et, de but en blanc, lui
demanda les pièces d'or.

Queffélec 1970, 53

D'ar Yaou ez eas Tomaz betek ti Gwilherm ha krak-ha-berr
e c'houlennas ar pezhioù aour digantañ.

Beyer 2016, 33

\emph{Regarder à la dérobée} a c'haller treiñ evel
\emph{sellet dre laer}:

Dehors, il se rendit tout de suite chez son père, qui lui répétait~:
«~Où vas-tu chercher tout ça~? Mon Dieu, où vas-tu chercher tout ça~?~»
et n'osait le regarder qu'à la dérobée.

Queffélec 1970, 58

''Va Dougwe, da belea e yez da busu toud an traou-ze?'' ha na grede
sellout outañ nemet dre laer.

Beyer 2016, 37

\subsection{Azasadur}\label{Azasadur}

{{{[}}\href{/wiki/Ostilho\%C3\%B9_stil,_tekniko\%C3\%B9_trei\%C3\%B1?action=edit&section=8}{{modifier}}{{]}}}

An azasadur a zo ur strategiezh treiñ a implijer evit treuzkas elfennoù
n'eus ket anezho e sevenadur ar yezh tal
\href{/wiki/Rottet_\%26_Moris_2018}{(Rottet ha Moris 2018, 26)}.

Ar skouer amañ-dindan a ziskouez penaos e c'haller
treuzkas unan eus ar modoù boutin da saludiñ an dud e Breizh en ur
implijout an azasadur:

Bonjour, monsieur Thomas...

Queffélec 1970, 71

Mont a ra ganeoc'h, Aotrou Tomaz...

Beyer 2016, 85

Implijout an azasadur a c'haller ivez evit treuzkas
elfennoù lec'hel pe rannyezhel:

"N'eo ket gwelet mat ar Mabichigoù Mareoñ, en ti-mañ", a
vousc'hoarzhas Paol Tirili, en ur droiñ e benn ouzh tu
ar beleg bihan.

Drezen 2012, 230

--- Les minets sont mal vus à la maison, sourit Paol Tirili, en se
tournant vers le petit prêtre.

Drezen 2002, 146

\section[Teknikoù treiñ
all]{\texorpdfstring{\protect\hypertarget{Tekniko.C3.B9_trei.C3.B1_all}{}{}Teknikoù
treiñ all}{Teknikoù treiñ all}}\label{Teknikouxf9_treiuxf1_all}

{{{[}}\href{/wiki/Ostilho\%C3\%B9_stil,_tekniko\%C3\%B9_trei\%C3\%B1?action=edit&section=9}{{modifier}}{{]}}}

Ouzhpenn ar strategiezhioù treiñ diazez kinniget gant Darbelnet ha Vinay
e plijfe din eztaoler un nebeud teknikoù all a gavan talvoudus.

\subsection{Astenn}\label{Astenn}

{{{[}}\href{/wiki/Ostilho\%C3\%B9_stil,_tekniko\%C3\%B9_trei\%C3\%B1?action=edit&section=10}{{modifier}}{{]}}}

An teknik-mañ a servij da resisaat ster ur ger en destenn troet en ur
ouzhpennañ gerioù war e lerc'h. Talvoudus e
c'hall bezañ pa vez liesster ur ger er yezh tal ha neket
er yezh orin. Da skouer, evit treiñ ar ger \emph{réservoire} e saozneg
diwar ar galleg, eo gwelloc'h lavaret \emph{water tank}
hag argas an amsterder a zegasfe ar ger \emph{tank} e- unan, pe hini a
lakafe da soñjal er c'harr milourel.

\subsection[Treiñ dre nac'hañ an
enep-ster]{\texorpdfstring{\protect\hypertarget{Trei.C3.B1_dre_nac.27ha.C3.B1_an_enep-ster}{}{}Treiñ
dre nac'hañ an
enep-ster}{Treiñ dre nac'hañ an enep-ster}}\label{Treiuxf1_dre_nacux27hauxf1_an_enep-ster}

{{{[}}\href{/wiki/Ostilho\%C3\%B9_stil,_tekniko\%C3\%B9_trei\%C3\%B1?action=edit&section=11}{{modifier}}{{]}}}

Treiñ dre nac'hañ an enep-ster a zo un doare da dreiñ un
elfenn gadarn diwar ar yezh orin gant un elfenn nac'h er
yezh tal, pe er c'hontrol, hep cheñch ster ar frazenn
orin.

\subsection[An
didroc'hañ]{\texorpdfstring{\protect\hypertarget{An_didroc.27ha.C3.B1}{}{}An
didroc'hañ}{An didroc'hañ}}\label{An_didrocux27hauxf1}

{{{[}}\href{/wiki/Ostilho\%C3\%B9_stil,_tekniko\%C3\%B9_trei\%C3\%B1?action=edit&section=12}{{modifier}}{{]}}}

An didroc'hañ a zo un teknik treiñ lec'h ma ranner ar
frazenn orin e div frazenn pe muioc'h frazennoù
bihanoc'h.

Ar frazenn-mañ zo bet didroc'het e div evit resisaat al
liamm etre ar rener \emph{le prêtre} hag ar renadenn dra ameeun \emph{la
piété de ses ouailles}:

Le prêtre songeait à la piété de ses ouailles qui fumerait dans la brume
et dans le soleil, qui s'élèverait, libre, lointaine et
pure.

Queffélec 1970, 29

Edo ar beleg o soñjal e deoliezh e zeñved o tivogediñ er vrumenn ha
dindan an heol. O sevel, dizalc'h, diamen ha glan.

Beyer 2016, 11

Ma vije dalc'het stumm ar frazenn orin en droidigezh
vrezhonek ne vefe ket sklaer petra zo o sevel
\emph{dizalc'h, diamen ha glan} - \emph{deoliezh e
zeñved} pe \emph{an heol}.

E galleg eo anat e komzer diwar-benn \emph{deoliezh} a-drugarez da reizh
gwregel ar ger \emph{piété}~: ne vijet ket bet posupl lavaret \emph{le
soleil lointaine}.

E brezhoneg n'eo ket anat reizh ar ger \emph{deoliezh} e
kenarroud-mañ, rak n'eus kemmadur ebet goude \emph{o
sevel}.

An doare disheñvel da ziskouez al liamm-se a vefe ouzhpennañ \emph{hag}:

Edo ar beleg o soñjal e deoliezh e zeñved o tivogediñ er vrumenn ha
dindan an heol {[}hag{]} o sevel, dizalc'h, diamen ha
glan.

Marteze he deus dibabet an droerez troc'hañ ar frazenn e
div evit chom hep adlavaret \emph{ha} teir gwech~: {[}\ldots{]} ha
dindan an heol hag o sevel, dizalc'h, diamen ha glan.

\hfill\break
E brezhoneg e c'hall ar ger \emph{seizet} bezañ un anv
pe un anv-gwan. Evit diskouez eo \emph{seizet ha
bouzar-kloc'h} an tad-kozh ha n'eo ket
ar gwele e voe krouet div frazenn en droidigezh vrezhonek:

Thomas s'approchait du lit clos de son grand-père, le
paralytique, aux trois quarts sourd, et qui s'était
gardé d'arrêter sa besogne, la confection
d'un panier d'osier, pour une scène à
quoi il ne comprenait rien.

Queffélec 1970, 35

Tostaet e oa Tomaz ouzh gwele-kloz e dad-kozh. Seizet ha
bouzar-kloc'h en doa hennezh eveshaet a baouez gant e
labour fardañ ur baner evit heuliañ ur gaoz na gomprene seurt ebet
dezhi.

Beyer 2016, 15

\subsection[Ar
stagañ]{\texorpdfstring{\protect\hypertarget{Ar_staga.C3.B1}{}{}Ar
stagañ}{Ar stagañ}}\label{Ar_stagauxf1}

{{{[}}\href{/wiki/Ostilho\%C3\%B9_stil,_tekniko\%C3\%B9_trei\%C3\%B1?action=edit&section=13}{{modifier}}{{]}}}

Ar stagañ eo kontrol an didroc'hañ.

Amañ eo bet kendeuzet un nebeud frazennoù gallek en ur frazenn nemeti en
droidigezh vrezhonek.

Des vieillards monologuaient et se récitaient à eux-mêmes ce
qu'ils auraient mis s'ils avaient su
écrire comme il le fallait. Qu'ils étaient des enfants
soumis et respectueux. Qu'au terme d'une
longue vie ils étaient heureux de voir un prêtre dans
l'ile. Qu'ils feraient leur possible
pour qu'il füt fier de sa paroisse. Que les communions
seraient fréquentes...

Queffélec 1970, 36

Bez\textquotesingle{} e oa kozhidi o komz en o-unan hag o tibunañ ar
pezh o dije lakaet m'o dije gouezet skrivañ evel ma oa
dav: e oant bugale sentus ha doujus, e oant eürus e dibenn un hir a
vuhez o welet ur beleg war an enez, e rafent o seizh gwellañ evit ma
vefe lorc'h ennañ gant e barrez, e vefe sakramantet
puilh.

Beyer 2016, 16

Strizh ar c'henarroud evit kregiñ ur frazenn e brezhoneg
gant ur rannig-verb.

Posupl e vije lakaat ur pik ha kregiñ al lavarenn nevez gant an anv, an
anv-gwan hag ar verb:

{[}\ldots{]} bugale sentus ha doujus e oant, eürus e oant {[}\ldots{]},
ober a rafent o seizh gwellañ {[}\ldots{]}.

\section[Ostilhoù
stil]{\texorpdfstring{\protect\hypertarget{Ostilho.C3.B9_stil}{}{}Ostilhoù
stil}{Ostilhoù stil}}\label{Ostilhouxf9_stil}

{{{[}}\href{/wiki/Ostilho\%C3\%B9_stil,_tekniko\%C3\%B9_trei\%C3\%B1?action=edit&section=14}{{modifier}}{{]}}}

Ar bazenn gentañ da astenn ar pennad-mañ a vefe kinnig ostilhoù stil
disheñvel e brezhoneg, evel ledanvad hag all. Pouezus-bras eo studiañ
pennad \href{/wiki/Le_Berre_2011}{Le Berre 2011} war al litote.

\subsection{Ledanvad}\label{Ledanvad}

{{{[}}\href{/wiki/Ostilho\%C3\%B9_stil,_tekniko\%C3\%B9_trei\%C3\%B1?action=edit&section=15}{{modifier}}{{]}}}

Er skouer kinniget amañ dindan e komzer diwar-benn an dud e-barzh an
iliz, ha neket ar savadur:

D'un seul et même courant, l'église se
vidait.

Queffélec 1970, 44

Gant ur froud nemetken e tileunie an ilizad.

Beyer 2016, 24

Hag en degouezh-mañ ez eus un hollekadur pa reer kaoz eus strollad an
dud a zo perzh d'ar barrez:

La paroisse l'écoutait sans surprise, avec une attention
fidèle.

Queffélec 1970, 46

E selaou a rae ar barreziad hep souezhiñ, gant un evezh feal.

Beyer 2016, 26
